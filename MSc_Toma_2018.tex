%% ---------------------------------------------------------------
%% $URL: https://repository.cs.ru.is/svn/thesis-template/trunk/ruthesis/latex/DEGREE-NAME-YEAR.tex $
%% $Id: DEGREE-NAME-YEAR.tex 350 2018-03-07 22:41:33Z foley $
%% This is a template LaTeX file for dissertations, theses, or reports at Reykjavík University
%% 
%% Comments and questions can be sent to the RU LaTeX group (latex AT list.ru.is) 
%% ---------------------------------------------------------------

%% METHOD:
%% 0) Read ruthesis/thesis-instructions.pdf
%%    If it is missing, goto https://repository.cs.ru.is/svn/thesis-template/trunk/ruthesis/thesis-instructions.pdf
%% 0.2) Subscribe to the announcements email list at
%%    https://list.ru.is/mailman/listinfo/latex-announcements
%% 1 LaTeX instructions.tex or goto http://afs.rnd.ru.is/project/thesis-template/trunk/ruthesis/latex/instructions.pdf
%% 2) Copy the template files (or unzip) to your working area
%% 3) Rename this file (if needed) with your information e.g. MSC-FOLEY-2007.tex
%% 4) Modify this file to fit your needs (please follow all comments below in the text)
%% 5) For making bibliographies, run "biber".  You can also change
%%    this back to "bibtex".  See below in "Bibliography options".

%%%%%%% CHOOSE ONE OF THESE %%%%%%%%%%%%%%%
%% projectreport: Project report (CS)
%% bachelors: Bachelor of Science thesis
%% masters: Master of Science thesis
%% doctorate: Doctor of Philosophy dissertation
%
%%%%%%% CHOOSE ONE OF THESE %%%%%%
%% 
%% draft: speed up processing by skipping graphics and adding useful
%%     information for editing.  Also sets spacing to double so that it is easier to
%%     write editing marks on paper copy.
%% proof:  proofreading version (final formatting with warnings)
%% final: generate document for submission, removing FIXMEs, and
%%     other markup.  Throw error if any fatal FIXMEs still in document.
%%
%%%%%%% CHOOSE ONE OF THESE IF APPLICABLE %%%%%%
%%
%% deptsse: School of Science and Engineering
%% deptscs: School of Computer Science
%%
%%%%%%%% CHOOSE ANY COMBINATION OF THESE %%%%%%%%%%%%
%%
%% forcegraphics: force graphics, etc. to be included, even in draft mode
%% debug:  writes more messages to the log file, adds debugging output 
%%     and sizing boxes
%% icelandic: thesis is in Icelandic
%% oldstyle:  use the PhD headers and footers from the old CS template
%% online: for online versions (skip blank pages)
\documentclass[table,xcdraw,online,masters,deptsse,forcegraphics,draft]{ruthesis}

%%%%%%%%%%%%%%%%%%%% TeXStudio Magic Comments %%%%%%%%%%%%%%%%%%%%%
%% These comments that start with "!TeX" modify the way TeXStudio works
%% For details see http://texstudio.sourceforge.net/manual/current/usermanual_en.html   Section 4.10
%%
%% What encoding is the file in?
% !TeX encoding = UTF-8
%% What language should it be spellchecked?
% !TeX spellcheck = en_US
%% What program should I compile this document with?
% !TeX program = xelatex

%%%%%%%%%%%%%%%%%%%% Bibliography options %%%%%%%%%%%%%%%%%%%%%
%% We suggest switching from bibtex to biblatex/biber because it is better able
%% to deal with Icelandic characters and other bibliography issues
%% As long as you use biblatex instead of bibtex by itself, it will at least
%%  generate a document without errors.
%% !!!If you are using TeXStudio, don't forget to update the bibliography setting!!!
\usepackage[backend=biber,bibencoding=utf8,style=ieee]{biblatex}
%\DeclareLanguageMapping{american}{american-apa}  
% need to declare mapping for style=apa to alphabetize properly
% If you set backend=bibtex, it will use bibtex for processing (old way)
%    this can work with Icelandic characters, but you may get weird results.
%    bibtex does not know how to sort Þ and ð
% if you set backend=biber, you can use UTF8 characters such as Þ and
%     ð  but you will have to remember to switch from using bibtex to 
%     biber in your client
% If you use JabRef, make sure the file is encoded in UTF-8 which is
%    not the default.

%% This tells TeXStudio to use biber
% !TeX TXS-program:bibliography = txs:///biber
%% This also sets the bibliography program for TeXShop and TeXWorks
% !BIB program = biber

% Where is your reference library?
\addbibresource{references.bib}

%%%%%%%%%%%%%%%%%%% CUSTOMIZATIONS %%%%%%%%%%%%%%%%%%%%%%%%%%%%%
%% It is not recommended that you customize this file nor
%% ruthesis.cls.  Just fill in the necessary fields.  You should put
%% your macros and packages into a separate file so that it is easier
%% to use updates to the template.  The custom.sty file was created
%% for this reason.  We load this much later so that it can overrite
%% any existing settings
\IfFileExists{custom.sty}{\usepackage{custom}}{}

%%%% My user packages-Toma%%%%%%
\usepackage{multirow}
\usepackage{longtable}
%\usepackage[table,xcdraw]{xcolor} %% this was added to the \documentclass[..here..]
% \usepackage{refcheck}
% \usepackage{showlabels}% !!!remove this in final version!!!


%%%%%%%%%%%%%%% INFORMATION %%%%%%%%%%%%%%%%%%5
%% University information must be multilingual to deal with the
%%  required cover pages and abstract on thesis
%% NOTE: This may not be required for other reports!!!

%% Babel Icelandic macros are setup  on RedHat at
%% /usr/share/texlive/texmf-dist/tex/generic/babel/icelandic.sty
%% /usr/share/texlive/texmf-dist/tex/generic/babel-icelandic/icelandic.ldf


%% Multilingual macros
%\newML{macroname}{englishword}{icelandicword}
%  creates \macronameML
%    \MLmacroname[english] - returns the english word
%    \MLmacroname[icelandic] - returns the icelandic word
%    \MLmacroname  - uses the current language setting
% Some useful ones have already been defined, but can be redefined
%% Predefined: \MLIceland \MLReykjavikUniversity \MLUniversityIceland

%% What institute?  Default is RU.
%\setInstitution{\MLReykjavikUniversity}
% \newML{InstitutionAddress}{Menntavegur 1\\101 Reykjavík, Iceland}
% {Menntavegi 1\\101 Reykjavík, Ísland}
% \setInstitutionAddress{\MLInstitutionAddress}
% \newML{Tel}{Tel.}{Sími}
% \setInstitutionPhone{\MLTel{} +354 599 6200\\
% Fax +354 599 6201}
% \setInstitutionURL{www.ru.is}


%% ONLY SET DEPARTMENT IF YOU HAVE NOT USED THE deptsse or deptscs OPTION!
%% Department and degree program
%\newML{ND}{New Department}{Nytt deild}
%\setSchool{\MLND}

%% Set your program of study
\newML{program}{Mechanical Engineering}{vélaverkfræði}
\program{\MLprogram}

%% Degree long name.  If not already defined, you can create a macro
%\newML{DEGREE}{English Degree Name}{Icelandic Degree Name}
%% Default is set based upon doctorate vs masters option
%% Predefined: \MLMSc \MLPhd
%\setDegreelong{\MLMSc}

%% Degree abb, change if default is not right
%% Default is set based upon doctorate vs masters option
%\degreeabbrv{Sc.D.} 

%\setFrontLogo{reyst-logo}
%% Use this if you need a different front logo on the first page
%% e.g. reyst-logo

%% Date in english and icelandic
%% NOTE: THIS IS THE DATE OF THE SUPERVISOR'S SIGNATURE!!!!!!
%% Predefined: \MLjan, \MLfeb, \MLmar, ... \MLdec
%\whensigned{day}{month}{year} %day is only used on some formats, but you must put something.
\whensigned{10}{\MLmay}{2019}

%% Title in English and Icelandic
%% You need to put both a normal case and ALL CAPS version into the macros.
%%
%% Note that title length is limited to what can fit in three lines in the inside page
\newML{Title}{Wake Turbulence Separation Re-categorization Requirements (RECAT) for Keflavik International Airport}{Titll verkefnis}
%\newML{Title}{Working Title}
\newML{TITLE}{WAKE TURBULENCE SEPARATION RE-CATEGORIZATION REQUIREMENTS (RECAT) FOR KEFLAVIK INTERNATIONAL AIRPORT}{TITLL VERKEFNIS}
%% If you have the title in just one language, put it into \setTitle{} directly e.g.
%%  \setTitle{Working Title}
%%
\setTitle{\MLTitle}{\MLTITLE}
%% ***** Special Titles ******
%% If the title must be formatted specifically for the cover page or internal pages
%% (typically via line-breaks using the \newline command) then the following commands must be used 
%%
%\setTitleCover{\MLTITLE}
%% These two for the internal cover pages, usually not needed
%\newML{TitleInternal}{Internal Title}{Icelandic Internal Title}
%\setTitleInternal{\MLTitleInternal}

%% Author name (should be the same in any language, if not use \newML)
%% If you are writing a Project report with multiple authors, separate them with \\:
%% To keep the names typeset together, you want to use non-breaking spaces: ~
%% TODO: fix formatting for TVD BSc multi-author (x4) --foley
%\author{Firstname1~Lastname1\\Firstname2~Lastname2}
\author{Toma~Emilov~Tomov}

%% If the name must be formatted specifically for the signature page
%% (typically via line-breaks) then the following command must be used 
%\setAuthorSignature{Student\\Name}
%% This macro adjusts the author name in the headers of the oldstyle formatting
%\setAuthorHeader{StudentLast}

%%% TODO:  Move the bachelor's form separately -- it confuses people. --foley
%%%%%%%%%%%%%%%%%%%%%%%%%% Project Report or Bachelor's Only!!! %%%%%%%%%%%%%%%%%%%%%%%%%%%%%%%%%%%%%%%%%%%
\setCourse{VT LOK 1012}

%%%%%%%%%%%%%%%%%%%%%%%%%% Bachelors Only!!! %%%%%%%%%%%%%%%%%%%%%%%%%%%%%%%%%%%%%%%%%%%
% \setID{010105--9999}%kennitala
% \setSemester{2016--1}
% \setShortSignedDate{1.1.2016}

% \setOrganization{Marel ehf.\\Austurhrauni 9\\210 Garðabær}
% \setSubProgram{Tæknifræði}

% %% If the thesis is confidential, uncomment this with the date it can be released
% %\setClosedDistribution{10.1.2016}%

% %% Put your keywords here in English, then Icelandic.  Separate them with commas.
% \newML{keywords}{Keyword1, Keyword2, Keyword3}{Lykliorð1, Lykliorð2, Lykliorð3}
% \setKeywords{\MLkeywords}

%%%%%%%%%%%%%%%%%%%%%%%%%%% Masters Only!! %%%%%%%%%%%%%%%%%%%%%%%%%%%%%%%%%%%%%%%%%%%%
%% How many credits (ECTS) on Master's degree
%% Usually 30 or 60
\ects{30}

%%%%%%%%%%%%%%%%%%%%%%%%%%% Doctorate Only!! %%%%%%%%%%%%%%%%%%%%%%%%%%%%%%%%%%%%%%%%%%
%% Some Computer Science Thesis have an ISSN number.
%% Most other documents do not.
%\bookidnumber{ISSN: 1670-8539} 
%% ID numbers are optional, but nice for sorting in libraries

%% International Standard Book Number (ISBN)
%% This is what most people should use if the thesis is being published.

%% International Standard Serial Number (ISSN)
%% This is usually only for a PhD dissertation as part of a series when published
%%   Computer Science: 1670-8539 

%% Additional degrees?  (optional, usually not needed)
%\adddegree{(list of degrees in appendix)}{(sjá lista yfir prófgraður í viðauka)}
%%%%%%%%%%%%%%%%%%%%%%%%%%%%%%%%%%%%%%%%%%%%%%%%%%%%%%%%%%%%%%%%%%%%%%%%%%%%%%%%%%%%%%%%


%% List the entire committee.  Each member has a name (degree should be omitted, unless it is not PhD),
%% Supervisor(s) must appear first
%% On a Bachelors, there is usually only one supervisor and one examiner.

%% Format for each entry:
%%  \personinfo{Name}{Role}{Job Title}{Company/institution}{Country}
%% Predefined macros: \MLSupervisor \MLSupervisors \MLExaminer \MLExaminers

%% Change these to singular/plural as needed.
%% Just uncomment and change the plurality of the macro.
\setSupervisorHeading{\MLSupervisors}
\setExaminerHeading{\MLExaminer}

%% Predefined macros:
%% \MLSeniorProfessor \MLProfessor \MLAssociateProfessor \MLAdjunctProfessor \MLEmeritusProfessor \Iceland
%% \MLReykjavikUniversity \MLUniversityIceland

%% Bachelors: primary advisor (Umsjónarkennari), ONLY ONE!
%% All others: As many as you want
\supervisors{
  \personinfo{Þorgeir~Pálson}{\MLSupervisor}{\MLEmeritusProfessor}{\MLReykjavikUniversity}{\MLIceland}
  \personinfo{Ármann~Gylfason}{\MLSupervisor}{\MLAssociateProfessor}
  {\MLReykjavikUniversity}{\MLIceland}
%  \personinfo{Helper A. Bunch}{Co-advisor}{\MLAssistantProfessor}{\MLUniversityIceland}{\MLIceland}
%  \personinfo{Ian M. Great}{Co-advisor}{\MLProfessor}{Hochschule Düsseldorf}{Germany}
}

%% Bachelors: secondary advisor (Leiðbeinandi), ONLY ONE
%% All others: As many as you want
\examiners{
  \personinfo{Tough E. Questions}{\MLExaminer}{Associate Professor}{Institute of Technology}{Iceland}

}

%% An abstract is required to be in both Icelandic and English for most degrees.
%% It is considered good form to limit the abstract to a single paragraph in each language,
%%   at 300 words.  Refer to your degree's instructions.
%% Note: Icelandic quotation marks cannot be typeset using "` and "'.  You should use \enquote{}
%% this is probably due to interactions with the MultiLingual macros.
%% TODO: turn this into more sensible macros to avoid confusion --foley
\newML{AbstractText}{

English abstract

}  
% ipsum generaes text text
{

útdráttur á íslensku

} % Icelandic abstract goes here
\setAbstract{\MLAbstractText}


%%%%%%%%%%%%%%INDEX SETUP %%%%%%%%%%%%%%%%%%%%%%%%%%%%%%%%%%%%%%%%%%%%%%%%%%%%
%% Indexes, and other auto-generated material
%% The Memoir package (which we use) automatically generates the index
%% See section 17.2 on page 302 of the guide
%% http://texdoc.net/texmf-dist/doc/latex/memoir/memman.pdf
%% This means you have to run "makeindex DEGREE-NAME-YEAR"
%% !!!Do not load any of the index packages, they cause problems with Memoir!!!
%% !!!You have been warned!!!
%% Note that memoir changes the [] options to only be for filenames, not other options!
\makeindex{}
\indexintoc{}

%% For abbreviations, you may want to try
%% Watch out though, each new index writes another external file and 
%% latex can only write a limited number of them
%%\usepackage[intoc]{nomencl} % intoc: In Table of Contents
%% remember to run:
%% makeindex filename.nlo  -s nomencl.ist -o filename.nls

\finalifforcegraphics{hyperref} %hyperlinks even in draft mode
\usepackage[hidelinks]{hyperref} 
%% !!!Must be the last package loaded except otherwise mentioned!!!!
%% \usepackage{hypcap}  %% puts link at top of figure, must be after hyperref

%%%%%%%%%%%%%%%%%%%%%%%%%%%%%%%%%%%%%%%%%%%%%%%%%%%%%%%%%%%%%%%%%%%%%%%%%%
%%%%%%%%%%%%%%%%%%%%%%% DOCUMENT START %%%%%%%%%%%%%%%%%%%%%%%%%%%%%%%%%%%
\begin{document}
%% Some elements have different names on the RU Masters rules
%% They will be annotated with RUM: "name"
\frontmatter{} % setup formatting at beginning

%\frontcover{}%%If you want to see what it looks like with the printed cover
%% TODO:  link to fill-in PDF file on RU website

\frontrequiredpages{}%% the various signaturepages and abstract
%%% WARNING:  if you get an error on the previous line, it is probably because
%%% you put a bad macro or something strange in a title, author, or abstract.

\ifdraft{\coverchapter{Important!!!  Read the Instructions!!!} If you
  have not already done so, \LaTeX{} the \path{instructions.tex} to
  learn how to setup your document and use some of the features.  You
  can see a (somewhat recent) rendered PDF of the instructions included in this folder at\path{instructions-publish.pdf}.
  There is also more information on working with \LaTeX{} at
  \url{http://afs.rnd.ru.is/project/htgaru/trunk/how-to-get-around-projects.pdf}.
  This includes common problems and fixes.

  This page will disappear in anything other than draft mode.}{}



%% Dedication is optional, comment out if it is absent
%% RUM: Not mentioned
% \begin{dedications}
%   I dedicate this to my two cats, Sid Vicious and Nina Hagen.
% \end{dedications}

\enableindents{}% turn on/off paragraph indents
% RUM: "Acknowledgements (optional)"
\coverchapter{Acknowledgements} 
% \begin{quotation}
% So long, and thanks for all the fish.
% \end{quotation}\sourceatright{Douglas Adams\cite{adams84fish}}
% \vspace{\baselineskip}

\draftnote{Acknowledgements are optional; comment this chapter out if they are absent
  Note that it is important to acknowledge any funding that helped in the work.}
  
The work on this thesis was supported by 2018~Isavia grant.
Additionally the Isavia company provided a workplace and the means to obtain the necessary information from the company servers. Special appreciation goes to Hjalti Pálsson, Head of Research and Development Department and his team of professionals: Atli Norðmann Sigurðarson, Specialist, for valuable insight into the manner of retrieving and analysing the data, Víkingur Logi Ásgeirsson, Project Manager, for his beneficial assistance in managing the specifics of the Python code.

I would like to give thanks to my supervisors Þorgeir Pálsson and Ármann Gylfason for the support and valuable advice throughout the process of developing this study.

% \coverchapter{Preface}
% % RUM: "Preface (optional)"
% This dissertation is original work by the author, Firstname~Lastname.
% Portions of the introductory text are used with permission from
% Student et al.\cite{student2015awesome} of which I am an author.

  
% \draftnote{The preface is an optional element
%   explaining a little who performed what work.  See
%   \url{https://www.grad.ubc.ca/sites/default/files/materials/thesis_sample_prefaces.pdf}
%   for suggestions.
  
%   List of publications as part of the preface is
%   optional unless elements of the work have already been published.
%   It should be a comprehensive list of all publications in which
%   material in the thesis has appeared, preferably with references to
%   sections as appropriate.  This is also a good place to state
%   contribution of student and contribution of others to the work
%   represented in the thesis.}

%\coverchapter{Publications}
%% RUM: Not mentioned, this was found in the CS thesis template.  
%% Maybe more applicable to PhD dissertations?
%%% Probably a duplication from before Preface became standard.

\starttables{}% setup formatting
%% TOC, list of figures and list of tables are required
\tableofcontents{}\clearpage%%RUM: "Table of contents"
\listoffigures{}\clearpage%%RUM: "List of figures"
\listoftables{}\clearpage%%RUM: "List of tables"

%\coverchapter{List of drawings and enclosed material}
%RUM: "List of drawings and enclosed material, e.g. CD(as appropriate)"

\listoffixmes{}
% if using fixme package, lists what needs to be done

%% The list of abbreviations is an example of a special list
%% Other lists may be added, such as lists of algorithms, symbols, theorems, etc.
%% IN CS PhD, this is sometimes centered.
\coverchapter{List of Abbreviations}%%RUM: Not mentioned
\begin{longtable}{ll}

% \begin{tabular}{ll}
% MSc     &   Masters of Science\\
% PhD   &   Doctor of Philosophy\\
 ADS-B   &   Automatic Dependent Surveillance-Broadcast\\
 AROT    &   Arrival Runway Occupancy Time\\
 ATC     &   Air Traffic Control\\
 ATM     &   Air Traffic Management\\
 B-C     &   CAT-B leader - CAT-C follower RECAT pair\\
 B-D     &   CAT-B leader - CAT-D follower RECAT pair\\
 BIKF    &   Keflavík International Airport ICAO code name\\
 CFMU   &   Central Flow Management Unit\\
 C-C     &   CAT-C leader - CAT-C follower RECAT pair\\
 C-D     &   CAT-C leader - CAT-D follower RECAT pair\\
 C-F     &   CAT-C leader - CAT-F follower RECAT pair\\
 D-C     &   CAT-D leader - CAT-C follower RECAT pair\\
 D-D     &   CAT-D leader - CAT-D follower RECAT pair\\
 D-F     &   CAT-D leader - CAT-F follower RECAT pair\\
 EAA    &   European Economic Area\\
 EASA    &   European Aviation Safety Agency\\
 EUROCONTROL    &   European Organisation for the Safety of Air Navigation\\
 FAA     &   Federal Aviation Administration\\
 H-H     &   Heavy leader - Heavy follower ICAO pair\\
 H-M     &   Heavy leader - Medium follower ICAO pair\\
 IATA    &   International Air Transport Association\\
 ICAO    &   International Civil Aviation Organisation\\
 IFR     &   Instrument Flight Rules\\
 IQR     &   Interquartile range\\
 KEF     &   Keflavík International Airport IATA code name\\
 LTI     &   Landing Time Interval\\
 M-H     &   Medium leader - Heavy follower ICAO pair\\
 M-M     &   Medium leader - Medium follower ICAO pair\\
 MOPS    &   Minimum Operational Performance Standards\\
 MRS     &   Minimum Radar Separation\\
 MTOM    &   Maximum Take-off Mass\\
 MTOW    &   Maximum Take-off Weight\\
 NACp    &   Navigation Accuracy Code for position\\
 NIC     &   Navigation Integrity Code\\
 NM      &   Nautical mile\\
 NUC     &   Navigation Uncertainty Code\\
 PIC     &   Position Integrity Code\\
 RECAT-EU&   European Wake Turbulence Categorisation and Separation Minima\\
        &    on Approach and Departure\\
 ROT     &   Runway Occupancy Time\\
 RWY     &   Runway\\
 SIL     &   Surveillance Integrity Level\\
 TWY     &   Taxiway\\
 VFR     &   Visual Flight Rules\\
 WTC     &   Wake Turbulence Category\\
 

% \end{tabular}
\end{longtable}

% \coverchapter{List of Symbols}%%RUM: Not mentioned
% \begin{tabular}{lll}
% Symbol &Description &Value/Units\\
% $E$ &Energy &\si{\joule}\\
% $m$ &Mass &\si{gram}\\
% $c$ &Speed of Light &\SI{2.99E8}{\meter\per\second\square}\\
% \end{tabular}

%% This command prepares for the actual text, e.g. by 
%% calling \mainmatter{}
\starttext{}

%% ---------------------------------------------------------------
%% From this point on, it is standard Latex, except the very end.
%% This is a "report"-based template, so the top-level heading 
%% is \chapter{}

%% WARNING: Make sure that all of these files (and any new ones)
%% are UTF-8 otherwise you will get weird encoding errors.
% \part{The First Part} % Parts optional but useful in longer documents

%% The default division is IMRAD, you may want to divide differently
%% See the introduction for guidance.

\chapter{Introduction\label{cha:introduction}}

Keflavík International Airport (IATA: KEF, ICAO: BIKF) is the major gateway to Iceland and is serviced by Isavia~\cite{Isavia_about}. The airport currently has two functioning runways operating in both directions and are designated as RWY-01 (South-North), RWY-19 (North-South), RWY-10 (West-East) and RWY-28 (East-West). 
There has been an unforeseen increase of passengers travelling through Keflavík Airport in recent years, with over eight million passengers and 62.300 air transport movements in 2017~\cite{isavia_facts_2017}, where a movement is defined as an aircraft take-off or landing at an airport~\cite{aircraft_movement}. For 2018 the passenger numbers show a yearly growth of 13,44\% (at the time of writing) for the same time period from the previous year~\cite{isavia_pass_statistics_2018}. The increase in traffic at Keflavík Airport can and does experience constrained capacity of the runways during peak traffic periods. Airport capacity is generally defined as the number of movements per unit of time that can be accepted~\cite{airport_capacity_methodology}.  

A significant constraint of airport capacity is caused by distance separation regulations between aircraft in flight due to wake turbulence. The term "wake turbulence" describes the effect of the rotating air masses generated behind the wing tips of large jet aircraft~\cite{doc4444full}. A smaller aircraft can experience loss of control in close proximity behind a larger aircraft because of those swirling air masses. The general flight safety requirements for maintaining safe distance between aircraft are established by the International Civil Aviation Organisation (ICAO)~\cite{doc4444full}. The European Aviation Safety Agency (EASA) provides oversight of those requirements and their application by the Air Traffic Management (ATM). 

The distance separation due to wake turbulence is especially critical during take-off and on approach for landing. ICAO has specified wake turbulence separation minima requirements that are applied to aircraft in the flight-path of another aircraft. Those requirements classify each aircraft in a wake turbulence category (WTC) and the wake turbulence separation is applied to a following aircraft associated with the wake category of a leading aircraft. Thus the aircraft form a pair composed of a "leader" and a "follower", that observe a certain spacing in nautical miles (NM). Generally the ICAO wake categories are three - Heavy, Medium and Light. The classification is based primarily on the weight of the aircraft, but this can vary slightly for different states.

The ICAO WTC categorisation is based on observations and calculations performed on aircraft models that are some 50 years old. Even though those separations are considered reliable, they are often conservative and outdated in view of newer aircraft models and as such generate unnecessary over-separation in some cases.
In recent years, a wake turbulence re-categorisation scheme known as RECAT-EU~\cite{rooseleer2015recat} has been devised by European Organisation for the Safety of Air Navigation (EUROCONTROL)~\cite{EUROCONTROL_recat_eu}, an intergovernmental organisation assisting in safe and efficient air traffic operations. 

RECAT-EU is based on the ICAO wake turbulence longitudinal separation minima on approach and departure, but takes into account the wake specifications of current aircraft models. The RECAT-EU scheme splits aircraft into six wake categories, generally referred to as CAT-A, CAT-B, CAT-C, CAT-D, CAT-E and CAT-F. Re-categorisation leads mostly to decreased wake turbulence separation minima requirements for aircraft pairs, which in turn can lead to accommodating more landing and departures during peak traffic periods and increase airport capacity (throughput).

Apart from wake turbulence separation, the other major factor that affects capacity is runway occupancy time (ROT), which is the time interval an aircraft occupies a runway during landing or departure. The rule of thumb is that no aircraft can use a runway before the previous aircraft has vacated it and runway occupancy, per se, does limit the throughput of a runway. The emphasis in this project will be on runway occupancy times for arriving aircraft (AROT) during peak traffic hours at the airport. The peak traffic hours are determined by the time intervals when the airport operates at high capacity loads.

The arrival runway occupancy time and the wake turbulence separation requirements are the two main metrics that directly influence throughput. The wake turbulence distance separation is transferred into the time domain and defined as the landing time interval (LTI), to be able to compare the two metrics and their ranges. The landing time interval quantifies the time separation between aircraft in a pair (derived from the distance separation between aircraft in a pair and the final approach speed of the follower). 

The aim of this study is to estimate the constraints for implementing the RECAT-EU wake turbulence categorisation and separation minima for Keflavík International Airport, the effect this can produce for airfield throughput and when is the runway occupancy a limiting factor for increased capacity.

\section{Runway Capacity\label{sec:runway_capacity}}

ICAO generally defines airport capacity as the number of movements per unit of time that can be accepted during different meteorological conditions~\cite{airport_capacity_methodology}. 
Several measures are currently used to estimate the amount of aircraft movements on the runways of an airport in a specified time interval.

R. de Neufville~\cite{de_neufville_airport_2013} states that the capacity of runway systems determines the ultimate capacity of an airport and that its principal measure is maximum throughput capacity. It indicates the average number of movements that can be performed on the runway system in one hour in the presence of continuous demand, while adhering to all the separation requirements imposed by the (ATM) system~\cite{de_neufville_airport_2013}.

The Capacity Envelope method used by Isavia to estimate the throughput capacity at BIKF takes into consideration the number of arrivals and departures within a selected time frame. This approach allows to measure the throughput capacity and model the maximum possible values or the potential of the airfield. A summary of the capacity envelope for a period of one year is shown in Figure~\ref{fig:capacity_evnelope} for a 15~minute time interval. 

\begin{figure}[h]
    \centering
    \includegraphics[width=0.8\textwidth]{graphics/fig_Capacity_Envelope_2017-10-04_to_2018-11-30_15min_occurrences_limits.png}
    \caption[Capacity envelope for BIKF]{Capacity envelope at BIKF since October 2017 during peak hours. The measured time interval is 15 minutes and includes both the morning and afternoon peaks. The capacity limits of the airfield are also defined and noted on the figure for various sequence modes. The mixed modes indicate: equal arrivals and departures (Mixed~Mode~A) or two times more arrivals than departures (Mixed Mode B), the latter being a tendency during the afternoon peaks. The figure is from Isavia's internal GUI Víkingaskipið.}  \label{fig:capacity_evnelope}
\end{figure}

This approach indicates a maximum measured capacity of four arrivals to four departures in a "Mixed~Mode~A". This equilibrium mode is defined as the primary measure for the throughput capacity of BIKF. The values are in accord with the defined peak hour intervals in section~\ref{sec:arot_and_study_objective}

There are many aspects of a runway system that can affect the number of aircraft that can land or depart from an airfield and some of those are listed below~\cite{de_neufville_airport_2013, kim_validation_2010}.
\begin{itemize}
    \item Number and geometric layout of the runways.
    \item State and performance of the ATM system.
    \item Wake Separation requirements between aircraft pairs imposed by the ATM.
    \item Weather conditions (visibility, precipitation, cloud ceiling). 
    \item Wind direction and strength.
    \item Mix of aircraft using the airport.
    \item Sequence of movement on each runway (arrivals -- departures split)
    \item Type and location of taxiway exits from the runway.
    \item Runway occupancy time.
    \item Controller workload.
    \item Noise-related constraints and other environmental considerations.
\end{itemize}

Two of the above mentioned factors are principal in determining airfield capacity, namely the wake turbulence separation requirements and runway occupancy time~\cite{kolos2013influence}. Those are in turn influenced by traffic mix, seasonal factors and runway exits, to mention a few. The following Chapter~\ref{cha:methods} on Methods will deal with some of those aspects. 


\section{Wake Turbulence Separation}
This section briefly reviews the necessity of adopting a wake turbulence separation scheme for aircraft from the perspective of fluid dynamics and the overall benefits of the RECAT-EU wake turbulence separation scheme in comparison to the ICAO WTC.

\subsection{Current Understandings of Wake Vortex Behaviour}
Wake vortices describe the nature of the air flow generated behind the wing tips of a jet aircraft~\cite{doc4444full}.
The lift effect on an aircraft wing is created by the differential pressure between the lower and the upper surface of the wing. 
At the wing tips the high pressure flow from the lower side leaks around the tip and to the upper side of the wing. Thus the streamlines over the wing are pushed inwards, while the streamlines under the wing are pushed outwards. When the two streams combine at the trailing edge of a lifting wing, the difference in span-wise velocity causes the air to roll up into a number of small stream-wise vortices distributed along the span of the wing that eventually mix together and are combined into two main counter-rotating vortices aft of the wing as illustrated in Figure~\ref{fig:vortex_develop} ~\cite{houghton2012aerodynamics,magazine_aibus_safety, Breitsamter2011Feb, gerz_commercial_2002}. Between the vortices the induced flow is downwards (downwash) while outside the air moves upwards.

The effect of wake vortices on air masses is known as wake turbulence. 
The kinetic energy contained in the vortices is dependent on the weight and aerodynamics of the generating aircraft. The cross flow velocities in the core region of the trailing vortices can reach $360$ km/h and the vortices can stay effective up to hundred wing spans, which can result in wake vortices lasting for several minutes and up to $30$ km behind larger aircraft~\cite{Breitsamter2011Feb, gerz_commercial_2002}. 
\begin{figure}[h]
    \centering
    \includegraphics[width=0.8\textwidth]{graphics/WakeVortexPlane.png}
    \caption[Wake vortex roll-up process]{Wing vortex evolution and roll-up process. Two main vortices form behind the aircraft turning in opposite directions, clockwise behind the left wing (seen from behind) and anti-clockwise behind the right one~\cite[p.~043]{magazine_aibus_safety}} \label{fig:vortex_develop}
\end{figure}

\subsection{Ground Effect and Vortex Decay Rate}
Wake vortices tend to descend slowly to an altitude of about one half of the initial separation, near ground on approach for landing.
Upon reaching this point the descent will stop and then ascend slowly. This "re-bounce" effect is caused by the presence of the ground and carries with it the formation of a second pair of induced vortices outside and below the main vortex. 

In the presence of stable cross-wind conditions the decay of the downwind vortex is not identical to the upwind vortex and the descent rate may vary significantly~\cite{Hallock2018Apr}. 
The trajectories of the vortex pair may be modified and the upwind vortex could stall over the runway while the downwind vortex ascends and decays faster (Figure~\ref{fig:vortex_ground_effect}).

\begin{figure}[h]
    \centering
    \includegraphics[width=0.8\textwidth]{graphics/Hallock_vortex_evolution.jpg}
    \caption[Wake vortex and ground effect]{Vortex evolution and ground effect~\cite[p.~29]{Hallock2018Apr}.} \label{fig:vortex_ground_effect}
\end{figure}

Measurements have shown that the lateral/sideways motion of a vortex in an airport environment could range between $229$~m and $518$~m and in some cases even $762$~m, which is the basis for the "2500-foot" rule separation in parallel runway configurations~\cite{Hallock2018Apr, hallock2004summary, hallock2003wake}.
The vertical descent rate of vortices for medium commercial aircraft in stable atmosphere is shown to be around $1.5-2.5$~m/s for the first $30$~seconds after which the descent slows and eventually approaches zero at $152-274$ m below the flight path, and for heavier aircraft decaying vortices have been observed at $305$~m below of the flight path~\cite{lissaman1973aircraft, Hallock2018Apr}. 

Aerodynamic properties of the aircraft govern the vortex roll-up process and ambient atmospheric conditions dominate the behaviour of the vortices forcing instability and eventual decay of the vortices~\cite{Hallock2018Apr}.
Several factors that drive the decay rate of the vortex pair have been formulated:
\begin{itemize}
    \item Atmospheric turbulence extracts energy from the vortex and reduces its strength leading to a faster wake decay ~\cite{Hallock2018Apr}.
    \item Viscosity of the atmosphere also draws out energy from the vortex but at a slower rate than the atmospheric turbulence. The so called dissipative action of viscosity  effectively removes energy from a disturbance, in this case a vortex, thereby causing it to decay~\cite{Hallock2018Apr, houghton2012aerodynamics}.
    \item Buoyancy force acts on the vortex in a thermally stable stratified environment as a result of the lesser air density inside the vortex system and may causes stall or rebound to the flight level~\cite{Holzapfel2001Feb, gerz_commercial_2002}.
    \item Vortex instability in the form of long wave sinusoidal fluctuations of the vortex core (Crow instability) may occur due to light turbulence in the atmosphere and may cause the vortices to link and decay faster~\cite{dup._donamdson_vortex_1975, Hallock2018Apr, crow2003stability}.
    \item Secondary vorticity structures, vertical rib-like counter rotating flow formations may appear between the vortex pair and eventually wrap around the main vortices, leading to turbulence build inside the system and rapid circulation decay~\cite{Holzapfel2001Feb, Holzapfel2003Jun}.
\end{itemize}

Wake vortex characteristics of conventional aircraft have been studied for several decades now and "it is generally acknowledged that the near wake of a large jet airliner poses a significant hazard to any smaller aircraft that follows it into of out of an air terminal"~\cite[p.~5]{dup._donamdson_vortex_1975}.

Distance separation between aircraft is necessary to mitigate the effect of the wake vortices. This is especially true for aircraft in approach for landing or take-off, "since practically all serious problems associated with wake hazard occur in the vicinity of airports"~\cite[p.~5]{dup._donamdson_vortex_1975}. 
The wake vortices generated by a leading aircraft (leader) can cause loss of lift or an induced rolling moment and velocity fluctuations in a following aircraft (follower), if the follower enters the wake turbulence region near the leader~(Figure~\ref{fig:vortex_encounter}).

\begin{figure}[h]
    \centering
    \includegraphics[width=0.8\textwidth]{graphics/reaction_in_wake.jpg}
    \caption[Wake vortex encounter]{Effect of wake vortex flow field on an aircraft: induced rolling moment (A), upward motion (B), loss of lift (C), yaw motion (D)~\cite[p.~33]{Hallock2018Apr}} 
    \label{fig:vortex_encounter}
\end{figure}

This can lead to the up-set and potentially loss of control of an aircraft following the flight path of a preceding aircraft.
To diminish the effect of such vortices the trailing aircraft must be maintained at a safe distance behind the leading aircraft as the vortices spread laterally to either side of the flight path and are dissipated~\cite{Breitsamter2011Feb}.
The distance between aircraft pairs in view of safety is referred to as wake vortex separation. 

\subsection{Wake Turbulence Categorisation}
Traditional separation standards, introduced in the 1970s, are defined by ICAO based on certificated maximum take-off mass (MTOM) of aircraft. The MTOM criteria allocates each aircraft in a wake turbulence category. The prescribed categories are three i.e. Heavy, Medium and Light.
The Heavy category includes aircraft with MTOM larger than $136.000$~kg. The aircraft in the Medium category have MTOM between $7.000$~and~$136.000$~kg and the Light category includes the aircraft with maximum take-off mass less than or equal to $7.000$~kg.
In addition, for aircraft in the order of $560.000$~kg, ICAO provides a subcategory to the Heavy types called Super Heavy (Table~\ref{tab:WTC}). 

\begin{table}[ht]
    \centering
    \resizebox{1\textwidth}{!} {
    \begin{tabular}{l|c|c|c|c|c|c}
    ~    & \multicolumn{6}{c}{Category} \\ \hline
    ICAO & Super Heavy & Heavy (H) & \multicolumn{2}{c|}{Medium (M)} & \multicolumn{2}{c}{Light (L)} \\
    
    ~    & A380-800    & > 136.000  & \multicolumn{2}{c|}{ 136.000 -- 7.000 } & \multicolumn{2}{c}{$ \leq 7.000$} \\ \hline
    
    ICAO (UK)   & ~  & Heavy (H) & Upper Medium (UM) & Lower Medium (LM) & Small (S)  & Light (L) \\
    
    ~    & A380-800    & > 136.000  & 136.000 -- 104.000     & 104.000 -- 40.000      & 40.000 -- 17.000 & $ \leq 17.000 $   \\ \hline
    
    FAA (US)   & Super      & Heavy & B757   & Large   & \multicolumn{2}{c}{Small} \\
    
    ~    & A380        &  > 136.000   & ~                 &  136.000 -- 18.600       & \multicolumn{2}{c}{$ \leq 18.600 $} \\ 
    \end{tabular}}
    \caption[ICAO wake turbulence categories based on maximum take-off mass]{ICAO wake turbulence categories of transport aircraft based on maximum take-off mass (MTOM) in~kg~\cite{doc4444full, uk_aeronautical_information_services_wake_2017, kolos2013influence}.} \label{tab:WTC}
\end{table}

Some states have chosen to further increase the number of the ICAO wake turbulence categories in light of safe airport operational experience. Aerodromes in the UK split the Medium and Light categories each into two subcategories (Table~\ref{tab:WTC}), while the Federal Aviation Administration (FAA) places Boeing~B757 into a separate category and increases the upper weight limit of the Small (Light) category to $18.600$~kg. The Boeing~757s are known to generate unusually high core vortex speed.~\cite{icao_wtc, uk_aeronautical_information_services_wake_2017, noauthor_recat_2018}

Under standard ICAO criteria, each aircraft is allocated in a wake category and the categories form pairs: a combination of a leader and a follower aircraft. Each pair is assigned a distance separation minima, depending on the wake turbulence generated by the leader and applied to the follower. Conventionally the separations are expressed in nautical miles (NM). The leader-follower combinations can be charted as a wake turbulence separation matrix~(Table~\ref{tab:ICAO_WTC}), in which the elements indicate the minimum spacing for each pair.

% Please add the following required packages to your document preamble:
% \usepackage{multirow}
% \usepackage{graphicx}
\begin{table}[h]
\centering
\resizebox{.8\textwidth}{!}{%
\begin{tabular}{|c|c|c|c|c|c|}
\hline
\multicolumn{2}{|c|}{\multirow{2}{*}{ICAO WTC scheme}} & \multicolumn{4}{c|}{Follower}              \\ \cline{3-6} 
\multicolumn{2}{|c|}{}                                 & Super (A380-800) & Heavy & Medium & Light \\ \hline
\multirow{4}{*}{\rotatebox[origin=c]{90}{Leader}}       & Super (A380-800)       & (*)                & 6 NM  & 7 NM   & 8 NM   \\ \cline{2-6} 
                              & Heavy                  & (*)                & 4 NM  & 5 NM   & 6 NM   \\ \cline{2-6} 
                              & Medium                 & (*)                & (*)     & (*)      & 5 NM   \\ \cline{2-6} 
                              & Light                  & (*)                & (*)     & (*)      & (*)      \\ \hline
\end{tabular}%
}
\caption[ICAO wake turbulence categories and separation minima]{ICAO wake turbulence categories and separation minima to avoid wake vortex encounter.(*) indicate radar separation minimum (MRS)~\cite{noauthor_recat_2018, rooseleer2015recat}.} \label{tab:ICAO_WTC}
\end{table}

The wake turbulence separation must be observed, for arrival and departure, when the airport operates under instrument flight rules (IFR). The IFR allow properly equipped aircraft to be flown under impaired visibility conditions. When visual flight rules (VFR) are in effect (flight visibility 5 km, clear of clouds and in sight of the surface), the separation may be decided by the pilot within observed safety limits.~\cite{gerz_commercial_2002, icao_annex_2005} 

When wake turbulence separation requirements are not specified and a surveillance system like radar is used, the longitudinal separation minimum may be reduced by the air traffic management (ATM), if the surveillance system's capabilities at a given location permit this~\cite{MRS_separation_standard}. The radar separation minimum (MRS) for aircraft established on the same final approach track is 3~NM (or 2,5~NM under given conditions) as prescribed by ICAO Doc 4444 PANS-AM~\cite{doc44444}.

\begin{figure}[h]
    \centering
    \includegraphics[width=1\textwidth]{graphics/Criteria_RECAT.png}
    \caption[RECAT-EU categorisation criteria]{Categorisation process and criteria for assigning an existing aircraft type into RECAT-EU scheme~\cite[p.~15]{rooseleer2015recat}} \label{fig:RECAT_criteria}
\end{figure}

The ICAO wake turbulence separation rules outline the "worst case" for each category and thus generate over-separation in many cases~\cite{noauthor_recat_2018, rooseleer2015recat}. The categorisation was implemented over 40 years ago and has become out-of-date in view of the newer aircraft models. Consequently EUROCONTROL has developed the European Wake Vortex Re-categorisation (RECAT-EU) as a more precise categorisation of aircraft compared to ICAO WTC. The objective was to safely help increase airport capacity and reduce delays by redefining wake turbulence categories and their associated separation 
minima. RECAT-EU is based on the ICAO scheme and takes into account maximum take-off weight (MTOW) and the wing span of the aircraft (Figure~\ref{fig:RECAT_criteria}).

According to EUROCONTROL the expected immediate benefits from RECAT-EU deployment are: increased runway throughput around 5\% and operational efficiency during peak periods. The runway capacity increase with RECAT-EU "comes from the categorisation and separation reduction for aircraft types which are predominant in European traffic"~\cite[p.~22]{rooseleer2015recat}. Implementation of RECAT-EU will mean a minimum ATM system update and does not require deployment of new technologies.

The resulting categories following the re-categorisation criteria from Figure~\ref{fig:RECAT_criteria} are six:

\begin{tabular}{l}
    \textbullet \space CAT A -- "Super Heavy" \\
    \textbullet \space CAT B -- "Upper Heavy"\\
    \textbullet \space CAT C -- "Lower Heavy"\\
    \textbullet \space CAT D -- "Upper Medium"\\
    \textbullet \space CAT E -- "Lower Medium"\\
    \textbullet \space CAT F -- "Light"\\ 
\end{tabular}

The RECAT-EU wake turbulence separation matrix combining leader and follower into pairs with assigned minimum spacing for each pair is shown in Table~\ref{tab:RECAT-dist}. The application of instrumental or visual flight rules and the MRS procedures remain identical to the ICAO WTC scheme. The introduction of RECAT-EU alters only the spacing between 
certain aircraft in a pair, which is applied to arrivals as well as departures.

\begin{table}[h]
\centering
\resizebox{.8\textwidth}{!}{%
\begin{tabular}{|c|c|c|c|c|c|c|c|}
\hline
\multicolumn{2}{|c|}{\multirow{2}{*}{RECAT-EU scheme}} & \multicolumn{6}{c|}{Follower}                   \\ \cline{3-8} 
\multicolumn{2}{|c|}{}                                 & CAT-A & CAT-B & CAT-C & CAT-D & CAT-E & CAT-F \\ \hline
\multirow{6}{*}{\rotatebox[origin=c]{90}{Leader}}            & CAT-A            & 3 NM   & 4 NM  & 5 NM  & 5 NM  & 6 NM  & 8 NM  \\ \cline{2-8} 
                                    & CAT-B            &    (*)    & 3 NM  & 4 NM  & 4 NM  & 5 NM  & 7 NM  \\ \cline{2-8} 
                                    & CAT-C            &    (*)    & (*)   & 3 NM  & 3 NM  & 4 NM  & 6 NM  \\ \cline{2-8} 
                                    & CAT-D            &    (*)    &   (*)    &    (*)   &   (*)    &   (*)    & 5 NM  \\ \cline{2-8} 
                                    & CAT-E            &    (*)    &   (*)    &   (*)    &   (*)    &   (*)    & 4 NM  \\ \cline{2-8} 
                                    & CAT-F            &    (*)    &   (*)    &    (*)   &   (*)    &   (*)    & 3 NM  \\ \hline
\end{tabular}%
}
\caption[RECAT-EU distance-based separation minima]{RECAT-EU wake turbulence distance-based separation minima on approach and departure. (*) indicates minimum radar separation (MRS), set at 3~NM (2,5~NM under given conditions), applicable as per current ICAO doc 4444 provisions \cite{doc44444, rooseleer2015recat, noauthor_recat_2018}.}
\label{tab:RECAT-dist}
\end{table}

The noticeable changes in wake turbulence separation under RECAT-EU from ICAO WTC are systematised in Table~\ref{tab:delta_distance_wtc2recat}. A beneficial change in wake turbulence separation experiences for example a Heavy leader -- Medium follower aircraft pair, which after re-categorisation form a CAT-C -- CAT-D pair. In this case, the separation from the leader can be reduced by up to 2~NM. 

\begin{table}[h]
\centering
\resizebox{.9\textwidth}{!}{%
\begin{tabular}{|c|c|c|c|c|c|c|c|}
\hline
\multicolumn{2}{|c|}{}                                          & \multicolumn{6}{c|}{Follower}                                                                                                                                                                                                   \\ \cline{3-8} 
\multicolumn{2}{|c|}{\multirow{-2}{*}{RECAT-EU scheme}} & CAT-A                             & CAT-B                                & CAT-C                         & CAT-D                         & CAT-E                         & CAT-F                                                \\ \hline
                                                        & CAT-A & \cellcolor[HTML]{FD6864}(+0,5 NM) & \cellcolor[HTML]{67FD9A}-2 NM        & \cellcolor[HTML]{67FD9A}-1 NM & \cellcolor[HTML]{67FD9A}-2 NM & \cellcolor[HTML]{67FD9A}-1 NM &                                                      \\ \cline{2-8} 
                                                        & CAT-B &                                   & \cellcolor[HTML]{67FD9A}-1 NM        &                               & \cellcolor[HTML]{67FD9A}-1 NM &                               & \cellcolor[HTML]{FD6864}+1 NM                        \\ \cline{2-8} 
                                                        & CAT-C &                                   & \cellcolor[HTML]{67FD9A}-1 (-1,5) NM & \cellcolor[HTML]{67FD9A}-1 NM & \cellcolor[HTML]{67FD9A}-2 NM & \cellcolor[HTML]{67FD9A}-1 NM &                                                      \\ \cline{2-8} 
                                                        & CAT-D &                                   &                                      &                               &                               &                               &                                                      \\ \cline{2-8} 
                                                        & CAT-E &                                   &                                      &                               &                               &                               & \cellcolor[HTML]{67FD9A}{\color[HTML]{000000} -1 NM} \\ \cline{2-8} 
\multirow{-6}{*}{\rotatebox[origin=c]{90}{Leader}}                                & CAT-F &                                   &                                      &                               &                               &                               & \cellcolor[HTML]{FD6864}(+0,5 NM)                    \\ \hline 
\end{tabular}%
}
\caption[Difference in wake turbulence separation minima between ICAO and RECAT-EU schemes]{Difference in wake turbulence separation minima on approach between reference ICAO and RECAT-EU schemes (full proposal)~\cite{rooseleer2015recat}}
\label{tab:delta_distance_wtc2recat}
\end{table}

The different category code names in this study will be referred to on a single letter basis for simplification e.g. "CAT-C" will be abbreviated to "C", "CAT-D" to "D", "Heavy" to "H", Medium to "M" and so forth. This applies to pair categories as well by abbreviating e.g. "CAT-C leader -- CAT-D follower"  to "C-D pair" or just "C-D", and "Medium--Medium" pair to "M-M".

\section{Arrival Runway Occupancy Time and Landing Time Interval}\label{sec:arot_and_study_objective}

The runway occupancy time (ROT) is defined by EUROCONTROL as the length of time that each aircraft occupies the runway~\cite{ROT_definition}. 
This project focuses on arrivals runway occupancy time (AROT) during peak traffic hours, that is the time interval defined by an aircraft crossing the threshold and its tail vacating the runway~\cite{AROT_definition}. The threshold is the beginning of that portion of the runway that is available for landing.

The peak traffic hours are characterised by the time intervals when the airport operates at high loads. High load interval or window is at least 15 minutes in length and has five or more flights arriving or departing from Keflavík Airport. The time between flights during the high load interval is set at $\leq$4~minutes (Figure~\ref{fig:Peak_Diagram}). This value is based on statistical estimations by Isavia in order to achieve two distinct peak hours: one in the morning and one in the afternoon. The last flight in a high load window that fulfils these requirements is not considered to be a part of the peak as its behaviour is not affected by an instantaneous next flight after it. 

\begin{figure}[h]
    \centering
    \includegraphics[width=1\textwidth]{graphics/Peak_Diagram.png}
    \caption[Rules defining a peak hour]{Rules used by Isavia to identify a high load interval as peak or not. The time separating each two aircraft in a peak cluster (red) is set as less than or equal to 4 minutes. The last red aircraft in the peak cluster is not counted as part of the peak. Cluster separators are defined as time intervals larger than four minutes.}
    \label{fig:Peak_Diagram}
\end{figure}

The AROT metric is essential for the project because a following aircraft is not allowed to land on the same runway before it has been vacated by the leading aircraft. The interval to the next landing aircraft is specified as the landing time interval (LTI) which is directly linked to the inter-arrival distance or in other words the wake turbulence separation. The LTI is a metric used to quantify the time separation between aircraft in a pair, derived from the distance separation between the aircraft and the final approach speed of the following aircraft. One way the conflict between the AROT and LTI can manifest itself is through a missed approach (ICAO: bulked landing), when an aircraft landing cannot be completed due to runway incursion \cite{doc44444}: the incorrect presence of another aircraft on the runway designated for landing.% (time travelled is distance travelled divided by velocity).

The goal of examining the arrival runway occupancy times and the landing time intervals is to determine whether the implementation of the RECAT-EU scheme at BIKF under certain conditions, would affect the throughput of the airfield. Additionally the analysis could indicate whether the limiting factor is the runway occupancy time or the separation requirement. The proposed transition by EUROCONTROL from ICAO WTS to RECAT-EU scheme suggests reduced distance separation for certain aircraft pairs (Table~\ref{tab:delta_distance_wtc2recat}). This reduction, revealed by shifting the wake turbulence separation minima for arrival pairs, creates potential for reduced spacing between aircraft, hence reducing the time interval between landings (LTI) and facilitating increased throughput capacity of the runways. 

\begin{figure}[h]
    \centering
    \includegraphics[width=0.8\textwidth]{graphics/fig_rot_landig_time_interval_RWY19_leader_H_follower_M_peak-hour_BAR_20171004_20181130.png}
    \caption[AROT and LTI of H-M pairs on RWY~19]{Arrival runway occupancy time (blue) and landing time intervals (green) for Heavy leader - Medium follower pairs on BIKF runway RWY-19. The observed time period is from October 2017 to November 2018. The red line indicates the ICAO WTC reference time separation for the selected H-M aircraft pairs. The figure is from the Isavia's internal GUI Víkingaskipið.}\label{fig:AROT_LTI_rwy19_H_M}
\end{figure}

Figure~\ref{fig:AROT_LTI_rwy19_H_M} may be used to illustrate the hypothesis. The current ICAO time reference separation, indicated by the red vertical line for BIKF RWY-19 will be relocated to the left for some of the aircraft pairs under the RECAT-EU scheme, following a decreased wake separation requirement. This relocation will be more noticeable for C-C and C-D pairs formed from the ICAO H-M pairs, in which case the separation is reduced from 5~NM to 3~NM. A shift of the reference separation line creates the potential for a shift in the distribution of the LTI, as long as it does not overlap with the AROT. Furthermore, study on the runway capacity of significantly larger airports~\cite{kolos2013influence} suggests that the frequency distribution of LTI tends to compress or squeeze to the right of the reference line with less standard deviation about the mean when the air traffic is intensified.






%%% Local Variables: 
%%% mode: latex
%%% TeX-master: "DEGREE-NAME-YEAR"
%%% End: 
%%RUM: Introduction


\chapter{Methods\label{cha:methods}}
Data collection on flights through the airspace monitored and controlled by Isavia has been going on for several years now. This data is filtered and stored on company servers for reference and analysis. The statistical analysis for this project was done in Python and used to determine the fleet mix at BIKF, runway occupancy and landing intervals.

\section{Data Collecting and Data Filtering}
More detailed data collection commences on 21 November 2014 with the implementation of the Automatic Dependent Surveillance-Broadcast (ADS-B) system by Isavia. ADS-B was preferred to radar at Keflavik Internetional Airport because of improved signal accuracy~\cite{isavia_wiki}. The ADS-B equipped aircraft through Keflavik Airport are estimated to be around~$90\%$~\cite{isavia-rounardeild_rannsoknir_2018}. The aircraft surveillance data specifications used for measurements are in the Eurocontrol ASTERIX Category 21 standard~\cite{ASTERIX_ADS-B_specs}.
The reading from the position of the ADS-B antenna of the aircraft is used instead of the tail and nose positions. This simplification could result in a few seconds difference. Certhttps://www.overleaf.com/read/whpxcswcrgytain filtering of the ADS-B data is applied before commencing measurements and calculations~\cite{isavia_wiki}: 
\begin{itemize}
    \item All records with no latitude, longitude and/or time data are omitted.
    \item The records are filtered with regards to quality, i.e. filtered with regards to Target Surveillance status, MOPS version, NUC/NIC, NACp, SIL and PIC.
    \item An ADS-B data record within 0.5 second from another is omitted, i.e. if there are less than 0.5 seconds between two successive locations one of the records is omitted.
    \item An algorithm is used to analyse the data and determine if it originates from two different flights (the flight stops for 15 minutes or more after/before taxiing). If so the data for the second flight is omitted
    \item The velocity is calculated from ADS-B location data and ADS-B time data (velocity = distance travelled/time). The ADS-B velocity record is optional in the ASTERIX category 21 standard and is therefore unreliable in measurements. Since the velocity is calculated from measured data, it can exhibit spikes. To get a more realistic velocity curve, a Savitzky-Golay filter is applied to smooth the data.
    \item An aircraft main gear lift-off is considered to be the point where the ADS-B ground bit is removed, i.e. the ADS-B ground bit goes from a value of 1 to a value of 0.
    \item An aircraft touchdown is considered to be the point where the ADS-B ground bit is set, i.e. the ADS-B ground bit goes from a value of 0 to a value of 1.
    \item An aircraft is considered to be stationary when the velocity is below 0.5~knots and to be moving if the velocity is greater than 0.5~knots.
    
\end{itemize}

\fxnote{The ROT measurements have been checked by: comparing them to measurements made by hand.
Who measures and who checks them and how???}

\section{Data Manipulation}
The data was obtained from the database using SQL server and stored as csv files. The manipulations on the data frames were performed using the functionality of Pandas and NumPy data analysis tools and computing packages for Python. A key table containing information about 2300 aircraft models, including ICAO and RECAT-EU wake turbulence categories for each model, was also used as reference. 
The reference table was provided by Eurocontrol. 
The Isavia aircraft data was cross-referenced with the Eurocontrol key table based on ICAO aircraft type labels, the mutual characteristic in both data sets. 
After cross-referencing a RECAT-EU category was assigned to each of the aircraft arriving at BIKF in peak hours. Outliers bellow the 0,003~quantile and above the 0,997~quantile were regarded as anomalies removed from data set based on AROT values. 
The resulting data frame was used for the analysis in this project. 
It contained over 11500 arrivals for the time period of almost four years (from 01.01.2015 until 30.11.2018). The data frame contained unique information about the AROT of each aircraft, landing time and runway, along with the ICAO wake category. Consequently the time frame for the analysis was reduced to 13 months (from 04.10.2017 to 30.11 2018) because of the effect of fast-exits on AROT, which is explained in the coming section \ref{sssec:factors_arot}


\subsection{Aircraft Traffic Mix\label{ssec:traffic_mix}}
The aircraft traffic mix is one of the components that affect runway capacity as mentioned in Section~\ref{sec:runway_capacity}. Sorting the aircraft fleet arriving at BIKF into ICAO WTC reveals that the majority of flights~(85,4\%) are in the Medium wake category (Figure~\ref{fig:post_fast_exit_mix_pie_v2}) and the rest are mainly in the Heavy category~(14,2\%) with less than one percent Light aircraft. The small portion of Light flights can be explained with the policy of ATM at BIKF to avoid servicing those flights during the morning and afternoon peaks. This distribution of the three categories implies that combining the aircraft into arriving pairs would produce primarily Medium-Medium pairs. This is confirmed by the analysis of the number of arrival pairs in the ICAO categories (Table~\ref{tab:mix_to_wtc}).\\
When this same fleet is presented using the RECAT categories, the prevailing category is CAT-C (73,7\%), followed by CAT-D (22,1\%). The percentage of each of the remaining categories varies within 2\%. The expectation that this distribution of the RECAT categories would result in primarily in C-C pairs is confirmed later on by the analysis presented in Table~\ref{tab:mix_to_recat}. \\
The method of of re-categorising the traffic mix is a prerequisite for assembling the RECAT pairs later on and determining the main aircraft categories that will be considered for analysis along with estimating the inter arrival distances characteristic of each pair.
\begin{figure}[h]
    \centering
    \includegraphics[width=1\textwidth]{graphics/fig_post_fast_exit_mix_pie_v2.png}
    \caption[Traffic mix in RECAT-EU and ICAO WTC]{The traffic mix at Keflavik Airport represented in RECAT-EU categories alongside ICAO WTC categories.}
    \label{fig:post_fast_exit_mix_pie_v2}
\end{figure}
Another approach for classification of the traffic fleet mix at BIKF in peak hours is to identify the aircraft types or models. The traffic data contains ICAO aircraft type designator, that is a two-, three- or four-character code comprising of numbers and letters. This designator is unique for each aircraft type. The top fifteen ICAO types of the aircraft mix are presented in Figure~\ref{fig:traffic_mix_by_model}. Dominating the scene (56.8\% of all arrivals) is the Boeing~757-200 model (ICAO:B752), followed by the Boeing~767-300 (ICAO:B763).
\begin{figure}[h]
    \centering
    \includegraphics[width=1\textwidth]{graphics/fig_traffic_mix_by_model.png}
    \caption[Traffic mix by aircraft model.]{The traffic mix at Keflavik Airport grouped by aircraft type are shown alongside the ICAO WTC and RECAT-EU designators. The top three models comprise the major part of the Icelandair fleet.}
    \label{fig:traffic_mix_by_model}
\end{figure}
 Those models represent a major part of the Icelandair fleet and the share of the Boeing~737~MAX~8 (ICAO:B38M) is likely to increase as the Icelandair company plans to gradually add sixteen new B38M and B39M models from the beginning of 2018~\cite{icelandair_fleet}. The aircraft types are shown with their ICAO and RECAT categories and the tendency is towards increasing the share of the Medium-Medium pairs, or the C-D and D-C RECAT pairs respectively, with the addition of the new Icelandair aircraft.




% -----------------------


\subsection{Arrival Runway Occupancy Times} 

The ICAO Doc 4444 PANS-AM~\cite{doc44444} dictates that the radar separation minimum (MRS) between succeeding aircraft which are established on the same final approach track, may be reduced from 3~NM to 2,5~NM under certain conditions. One of the requirements is that the average runway occupancy time of landing aircraft is proven, by means such as data collection and statistical analysis and methods based on theoretical model, not to exceed 50 seconds. \\
Several approaches were used to look at the AROT at the airfield. First the runway occupancy was inspected based on the RECAT category of the aircraft (Figure~\ref{fig:RECAT_AROTs_boxplot}). 
\begin{figure}[h]
    \centering
    \includegraphics[width=1\textwidth]{graphics/fig_RECAT_AROTs_boxplot.png}
    \caption[AROTs boxplot for RECAT categories, all runways]{Arrival Runway Occupancy Times for the different RECAT-EU categories based on data gathered for a period of one year since October 2017. The box plot shows the AROTs for all runways at BIKF.}
    \label{fig:RECAT_AROTs_boxplot}
\end{figure}
The analysis showed that none of the average runway occupancy values fulfils the 50~seconds limit for reduced MRS. Closest to the required time were the aircraft from the CAT-D and CAT-C with mean values of 75 and 78 seconds respectively (Table~\ref{tab:AROT_RECAT_stats}). Those results point to the necessity of setting the MRS reference value at 3 NM and also suggest that the runway occupancy will be a limiting factor for the cases in which MRS is applicable (refer to Table~\ref{tab:RECAT-dist}).


\subsubsection{Factors affecting AROT\label{sssec:factors_arot}}
\fxnote{Airfield surface condition, weather conditions, fast exits, season. The effect of Specific weather conditions on ROT were not investigated in this project, Airfield surface conditions ref from Doc 4444.}







% Please add the following required packages to your document preamble:
% \usepackage{graphicx}
\begin{table}[h]
\centering
\resizebox{0.8\textwidth}{!} & \multicolumn{1}{l|}{50\%} & \multicolumn{1}{l|}{75\%} & \multicolumn{1}{l|}{max} \\ \hline
\multicolumn{1}{|l|}{SUMMER} & \multicolumn{1}{r|}{3727} & 76  & 17 & 46 & 63  & 72 & 87  & 155 \\ \hline
\multicolumn{1}{|l|}{WINTER} & \multicolumn{1}{r|}{937}  & 84  & 20 & 45 & 69  & 84 & 96  & 153 \\ \hline
\end{tabular}%
}
\caption[AROTs for the air traffic mix by season]{AROT statistics for the air traffic mix at KEF by season. The count is the number of landings in peak hours since October 2017}
\label{my-label2}
\end{table}

% Please add the following required packages to your document preamble:
% \usepackage{graphicx}
\begin{table}[h]
\centering
\resizebox{0.8\textwidth}{!} & \multicolumn{1}{l|}{50\%} & \multicolumn{1}{l|}{75\%} & \multicolumn{1}{l|}{max} \\ \hline
\multicolumn{1}{|l|}{RWY 01} & 963 & 85 & 21 & 46 & 68 & 88 & 100 & 152 \\ \hline
\multicolumn{1}{|l|}{RWY 10} & 852 & 85 & 11 & 61 & 78 & 84 & 91 & 155 \\ \hline
\multicolumn{1}{|l|}{RWY 19} & 1511 & 67 & 10 & 46 & 61 & 66 & 71 & 155 \\ \hline
\multicolumn{1}{|l|}{RWY 28} & 401 & 68 & 14 & 49 & 60 & 66 & 72 & 155 \\ \hline
\end{tabular}%
}
\caption[AROTs for the air traffic mix by runway for the summer]{AROT statistics for the air traffic mix at KEF by runway for the summer of 2018. The count is the number of landings in peak hours.}
\label{my-label3}
\end{table}
\fxnote{Include a schematic of the airport, overview for runway and taxiway reference}


% Please add the following required packages to your document preamble:
% \usepackage{graphicx}
\begin{table}[h]
\centering
\resizebox{0.8\textwidth}{!} & \multicolumn{1}{l|}{50\%} & \multicolumn{1}{l|}{75\%} & \multicolumn{1}{l|}{max} \\ \hline
\multicolumn{1}{|l|}{RWY 01} & 284 & 88 & 26 & 47 & 63 & 90 & 106 & 153 \\ \hline
\multicolumn{1}{|l|}{RWY 10} & 333 & 93 & 12 & 70 & 85 & 91 & 99 & 144 \\ \hline
\multicolumn{1}{|l|}{RWY 19} & 251 & 72 & 12 & 51 & 65 & 70 & 75 & 140 \\ \hline
\multicolumn{1}{|l|}{RWY 28} & 69 & 73 & 13 & 49 & 63 & 72 & 79 & 105 \\ \hline
\end{tabular}%
}
\caption[AROTs for the air traffic mix by runway for the winter]{AROT statistics for the air traffic mix at KEF by runway for the winter season. The count is the number of landings in peak hours since October 2017}
\label{my-label4}
\end{table}


% Please add the following required packages to your document preamble:
% \usepackage{graphicx}
% \usepackage[table,xcdraw]{xcolor}
% If you use beamer only pass "xcolor=table" option, i.e. \documentclass[xcolor=table]{beamer}
\begin{table}[h]
\centering
\resizebox{0.8\textwidth}{!} & \multicolumn{1}{l|}{50\%} & \multicolumn{1}{l|}{75\%} & \multicolumn{1}{l|}{max} \\ \hline
\rowcolor[HTML]{DAE8FC} 
\multicolumn{1}{|l|}{\cellcolor[HTML]{DAE8FC}January} & 95 & 86 & 20 & 54 & 70 & 85 & 96 & 136 \\ \hline
\rowcolor[HTML]{DAE8FC} 
\multicolumn{1}{|l|}{\cellcolor[HTML]{DAE8FC}February} & 80 & 87 & 16 & 52 & 75 & 86 & 93 & 144 \\ \hline
\rowcolor[HTML]{DAE8FC} 
\multicolumn{1}{|l|}{\cellcolor[HTML]{DAE8FC}March} & 215 & 84 & 18 & 47 & 70 & 85 & 95 & 150 \\ \hline
\rowcolor[HTML]{DAE8FC} 
\multicolumn{1}{|l|}{\cellcolor[HTML]{DAE8FC}April} & 221 & 82 & 17 & 47 & 69 & 81 & 93 & 139 \\ \hline
\rowcolor[HTML]{FFFC9E} 
\multicolumn{1}{|l|}{\cellcolor[HTML]{FFFC9E}May} & 386 & 75 & 13 & 46 & 67 & 74 & 83 & 140 \\ \hline
\rowcolor[HTML]{FFFC9E} 
\multicolumn{1}{|l|}{\cellcolor[HTML]{FFFC9E}June} & 647 & 73 & 16 & 47 & 62 & 68 & 80 & 155 \\ \hline
\rowcolor[HTML]{FFFC9E} 
\multicolumn{1}{|l|}{\cellcolor[HTML]{FFFC9E}July} & 729 & 72 & 16 & 46 & 61 & 68 & 77 & 155 \\ \hline
\rowcolor[HTML]{FFFC9E} 
\multicolumn{1}{|l|}{\cellcolor[HTML]{FFFC9E}August} & 724 & 80 & 18 & 46 & 64 & 81 & 91 & 148 \\ \hline
\rowcolor[HTML]{FFFC9E} 
\multicolumn{1}{|l|}{\cellcolor[HTML]{FFFC9E}September} & 632 & 78 & 18 & 46 & 64 & 75 & 88 & 152 \\ \hline
\rowcolor[HTML]{FFFC9E} 
\multicolumn{1}{|l|}{\cellcolor[HTML]{FFFC9E}October} & 609 & 79 & 17 & 46 & 66 & 76 & 91 & 155 \\ \hline
\rowcolor[HTML]{DAE8FC} 
\multicolumn{1}{|l|}{\cellcolor[HTML]{DAE8FC}November} & 220 & 84 & 23 & 48 & 66 & 82 & 99 & 151 \\ \hline
\rowcolor[HTML]{DAE8FC} 
\multicolumn{1}{|l|}{\cellcolor[HTML]{DAE8FC}December} & 106 & 87 & 23 & 51 & 69 & 86 & 99 & 153 \\ \hline
\end{tabular}%
}
\caption[AROTs for the air traffic mix by month]{AROT statistics for the air traffic mix at BIKF by month. The count is the number of landings in peak hours from October 2017 til November 2018. The colour fields indicate a subjective separation of the data into summer and winter season, based on mean AROT value. Months with mean AROT $\leq$ 80 seconds are classified as summer, and the remaining as winter.}
\label{my-label5}
\end{table}



\fxnote{Looked ar ROT, boxplot, seasonal and by RWY, fast exit effect}\\



% ---------------------------
\subsection{Landing Time Intervals}


% Please add the following required packages to your document preamble:
% \usepackage{multirow}
% \usepackage{graphicx}
\begin{table}[h]
\centering
\resizebox{\textwidth}{!}{%
\begin{tabular}{|c|c|c|c|c|c|c|c|}
\hline
\multicolumn{2}{|c|}{\multirow{2}{*}{RECAT-EU scheme}} & \multicolumn{6}{c|}{Follower}                   \\ \cline{3-8} 
\multicolumn{2}{|c|}{}                                 & CAT-A & CAT-B & CAT-C & CAT-D & CAT-E & CAT-F \\ \hline
\multirow{6}{*}{\rotatebox[origin=c]{90}{Leader}}             & CAT-A            &        & 100s  & 120s  & 140s  & 160s  & 180s  \\ \cline{2-8} 
                                    & CAT-B            &        &       &       & 100s  & 120s  & 140s  \\ \cline{2-8} 
                                    & CAT-C            &        &       &       & 80s   & 100s  & 120s  \\ \cline{2-8} 
                                    & CAT-D            &        &       &       &       &       & 120s  \\ \cline{2-8} 
                                    & CAT-E            &        &       &       &       &       & 100s  \\ \cline{2-8} 
                                    & CAT-F            &        &       &       &       &       & 80s   \\ \hline
\end{tabular}%
}
\caption[RECAT-EU time-based separation minima]{RECAT-EU WT time-based separation minima on approach and departure~\cite{rooseleer2015recat}\fxnote{This table is not referenced, maybe unnecessary, maybe move to LTI chapter!}}
\label{tab:RECAT-time}
\end{table}


% Please add the following required packages to your document preamble:
% \usepackage{multirow}
% \usepackage{graphicx}
% \usepackage[table,xcdraw]{xcolor}
% If you use beamer only pass "xcolor=table" option, i.e. \documentclass[xcolor=table]{beamer}
\begin{table}[h]
\centering
\resizebox{0.4\textwidth}{!}{%
\begin{tabular}{cc|r|r|r|}
\cline{3-5}
\multicolumn{1}{l}{} & \multicolumn{1}{l|}{} & \multicolumn{3}{c|}{Follower} \\ \cline{3-5} 
\multicolumn{1}{l}{} & \multicolumn{1}{l|}{} & \multicolumn{1}{c|}{H} & \multicolumn{1}{c|}{M} & \multicolumn{1}{c|}{L} \\ \hline
\multicolumn{1}{|c|}{} & H & \cellcolor[HTML]{FFCC67}56 & \cellcolor[HTML]{FE996B}332 & \cellcolor[HTML]{FFFFC7}2 \\ \cline{2-5} 
\multicolumn{1}{|c|}{} & M & \cellcolor[HTML]{FE996B}367 & \cellcolor[HTML]{FD6864}1795 & \cellcolor[HTML]{FFFFC7}7 \\ \cline{2-5} 
\multicolumn{1}{|c|}{\multirow{-3}{*}{\rotatebox[origin=c]{90}{Leader}}} & L & \cellcolor[HTML]{FFFFC7}3 & \cellcolor[HTML]{FFFFC7}9 & \cellcolor[HTML]{FFFFC7}1 \\ \hline
\end{tabular}%
}
\caption[BIKF traffic mix sorted into ICAO WTC]{Number of ICAO pairs from the traffic mix at BIKF arranged into the corresponding wake categories. It is apparent that the majority of arrival pairs are classified as Medium-Medium}
\label{tab:mix_to_wtc}
\end{table}


% Please add the following required packages to your document preamble:
% \usepackage{multirow}
% \usepackage{graphicx}
% \usepackage[table,xcdraw]{xcolor}
% If you use beamer only pass "xcolor=table" option, i.e. \documentclass[xcolor=table]{beamer}
\begin{table}[h]
\centering
\resizebox{\textwidth}{!}{%
\begin{tabular}{cc|c|c|c|c|c|c|}
\cline{3-8}
\multicolumn{1}{l}{} & \multicolumn{1}{l|}{} & \multicolumn{6}{c|}{Follower} \\ \cline{3-8} 
\multicolumn{1}{l}{} & \multicolumn{1}{l|}{} & CAT-A & CAT-B & CAT-C & CAT-D & CAT-E & CAT-F \\ \hline
\multicolumn{1}{|c|}{} & CAT-A &  &  &  & \cellcolor[HTML]{FFFFC7}1 &  &  \\ \cline{2-8} 
\multicolumn{1}{|c|}{} & CAT-B &  & \cellcolor[HTML]{FFFFC7}1 & \cellcolor[HTML]{FFFC9E}17 & \cellcolor[HTML]{FFFC9E}16 & \cellcolor[HTML]{FFFFC7}2 &  \\ \cline{2-8} 
\multicolumn{1}{|c|}{} & CAT-C &  & \cellcolor[HTML]{FFFC9E}19 & \cellcolor[HTML]{FD6864}1690 & \cellcolor[HTML]{FE996B}240 & \cellcolor[HTML]{FFCE93}42 & \cellcolor[HTML]{FFFC9E}13 \\ \cline{2-8} 
\multicolumn{1}{|c|}{} & CAT-D & \cellcolor[HTML]{FFFFC7}1 & \cellcolor[HTML]{FFFC9E}10 & \cellcolor[HTML]{FE996B}229 & \cellcolor[HTML]{FE996B}198 & \cellcolor[HTML]{FFFC9E}10 & \cellcolor[HTML]{FFFFC7}5 \\ \cline{2-8} 
\multicolumn{1}{|c|}{} & CAT-E &  & \cellcolor[HTML]{FFFFC7}1 & \cellcolor[HTML]{FFCE93}43 & \cellcolor[HTML]{FFFFC7}9 &  &  \\ \cline{2-8} 
\multicolumn{1}{|c|}{\multirow{-6}{*}{\rotatebox[origin=c]{90}{Leader}}} & CAT-F &  &  & \cellcolor[HTML]{FFFC9E}16 & \cellcolor[HTML]{FFFFC7}9 & \cellcolor[HTML]{FFFFC7}1 &  \\ \hline
\end{tabular}%
}
\caption[BIKF traffic mix sorted into RECAT-EU categories]{Number of RECAT pairs from the traffic mix at BIKF arranged into the corresponding wake categories. The majority of arrival pairs are classified as C-C.}
\label{tab:mix_to_recat}
\end{table}




\begin{figure}[h]
    \centering
    \includegraphics[width=1\textwidth]{graphics/fig_dist_separ_HH_HM_MH_MM_pairs.png}
    \caption[list of figures caption]{Caption}
    \label{fig:dist_separ_HH_HM_MH_MM_pairs}
\end{figure}

\begin{figure}
    \centering
    \includegraphics[width=1\textwidth]{graphics/fig_time_separ_HH_HM_MH_MM_pairs.png}
    \caption[list of figures caption]{Caption}
    \label{fig:time_separ_HH_HM_MH_MM_pairs}
\end{figure}




%\lipsum[14-20]
%%% Local Variables: 
%%% mode: latex
%%% TeX-master: "DEGREE-NAME-YEAR"
%%% End: 
%%RUM: "Methods"
% \part{The Second Part}
\chapter{Results and Analysis\label{cha:results_analysis}}

The results chapter deals with presenting the different metrics discussed so far, within the RECAT-EU scheme. The data subset for the analysis was limited to four pair categories from the ICAO scheme: H-H, H-M, M-H and M-M. This simplification was done in order to filter out the small number of data points for other pairs, insufficient for any valid estimations of the distribution of inter-arrival distance or the landing time intervals. 

The distribution of inter-arrival distances for the RECAT-EU pairs were examined in agreement with the RECAT-EU reference separations. The AROT and LTI for particular pairs were compared to identify the limiting factor for throughput capacity, using the method already introduced in section LTI \fxnote{expoainagain how is lti and rot limiting. this secton comes later} .

\section{Aircraft Traffic Mix\label{sec:traffic_mix}}

The aircraft traffic mix is one of the components that affect runway capacity as mentioned in the Introduction section~\ref{sec:runway_capacity}. Analysing of the traffic mix outlines the main aircraft categories that will be considered for analysis, the portion of arrivals that will experience reduced wake turbulence separation minima along with estimate of the inter arrival distance characteristic of each pair. A simplified overview of the re-categorisation mechanism for BIKF is shown in Table~\ref{tab:wtc2recat_division}.

\begin{table}[h]
\centering
\resizebox{0.7\textwidth}{!}{%
\begin{tabular}{clcllc}
\multicolumn{3}{c}{\cellcolor[HTML]{34CDF9}HEAVY} &  &  & \cellcolor[HTML]{FD6864}LIGHT \\ \hline
\multicolumn{1}{|c|}{\cellcolor[HTML]{34CDF9}CAT-A} & \multicolumn{1}{c|}{\cellcolor[HTML]{34CDF9}CAT-B} & \multicolumn{1}{c|}{\cellcolor[HTML]{32CB00}CAT-C} & \multicolumn{1}{c|}{\cellcolor[HTML]{F8FF00}CAT-D} & \multicolumn{1}{c|}{\cellcolor[HTML]{F8FF00}CAT-E} & \multicolumn{1}{c|}{\cellcolor[HTML]{FFC702}CAT-F} \\ \hline
\multicolumn{1}{l}{} &  & \multicolumn{4}{c}{\cellcolor[HTML]{F8FF00}MEDIUM}
\end{tabular}%
}
\caption[Transition from ICAO WTC to RECAT-EU categories]{Transition from ICAO WTC to RECAT-EU categories. CAT-C combines aircraft from the ICAO Heavy and Medium categories and CAT-F combines aircraft from the Medium and Light categories. The categorisation process and criteria for assigning an existing aircraft type into RECAT-EU scheme is illustrated in detail in Figure~\ref{fig:RECAT_criteria}.}
\label{tab:wtc2recat_division}
\end{table}

Sorting the aircraft fleet arriving at BIKF into ICAO WTC reveals that the majority of flights~(85,4\%) are in the Medium wake category and the rest are mainly in the Heavy category~(14,2\%) with less than one percent Light aircraft~(Figure~\ref{fig:post_fast_exit_mix_pie_v2}). The small portion of Light flights can be explained with the policy of air traffic control at BIKF to avoid servicing those flights during the morning and afternoon peaks. This distribution of the three categories implies that combining the aircraft into arriving pairs would produce primarily Medium-Medium (M-M) pairs. This is confirmed later by the analysis of the number of arrival pairs in the ICAO categories (Table~\ref{tab:pairs_mix_to_wtc}).

\begin{figure}[h]
    \centering
    \includegraphics[width=1\textwidth]{graphics/fig_post_fast_exit_mix_pie_v2.png}
    \caption[Traffic mix in RECAT-EU and ICAO WTC]{The traffic mix at Keflavik Airport during peak hours, represented in RECAT-EU categories alongside ICAO WTC. The observed time period is 13 months starting October 2017.}
    \label{fig:post_fast_exit_mix_pie_v2}
\end{figure}

When this same fleet mix is presented using the RECAT-EU categories, the prevailing category is CAT-C (73,7\%), followed by CAT-D (22,1\%)~(Figure~\ref{fig:post_fast_exit_mix_pie_v2}). The percentage of each of the remaining categories varies but does not exceed 2\%. The expectation that this distribution of the RECAT-EU categories would result in primarily in C-C pairs is confirmed later on by the numbers presented in Table~\ref{tab:pairs_mix_to_recat}. 

Another approach for classification of the traffic fleet mix at BIKF during peak hours was to identify the aircraft types or models. The traffic data contains ICAO aircraft type designator, which is a two-, three- or four-character code composed of numbers and letters. This designator is unique for each aircraft type. The top fifteen aircraft types of the aircraft mix with their ICAO and RECAT-EU categories are presented in Figure~\ref{fig:traffic_mix_by_model}. Dominating the scene (56,8\% of all arrivals during peak hours) is the Boeing~757-200 model (ICAO: B752), followed by the Boeing~767-300 (ICAO: B763).

\begin{figure}[h]
    \centering
    \includegraphics[width=1\textwidth]{graphics/fig_traffic_mix_by_model.png}
    \caption[Traffic mix by aircraft type.]{The traffic mix at Keflavik Airport during peak hours, grouped by aircraft type are shown alongside the ICAO WTC and RECAT-EU designators. The top three types comprise the major part of the Icelandair fleet. The observed time period is 13 months starting October 2107.}
    \label{fig:traffic_mix_by_model}
\end{figure}

 Those aircraft types represent a major part of the Icelandai fleet ~\cite{icelandair_fleet} and the share of the Boeing~737~MAX~8 (ICAO: B38M) is likely to increase as the Icelandair company plans to gradually add sixteen new B38M and B39M models from the beginning of 2018. The tendency is towards increasing the share of the Medium-Medium pairs, or the C-D and D-C RECAT-EU pairs respectively, with the addition of the new Icelandair aircraft.\fxnote{share of icelandair flights for the last year from all the arruivals in peaks.}


\section{Arrival Runway Occupancy Time Considerations}\label{sec:AROT_considerations} 

\subsection{Traffic Mix and AROT\label{ssec:mix_effect_arot}}

Several approaches were used to look at the AROT at the airfield. First the runway occupancy was inspected based on the RECAT-EU category of the arrival aircraft in peak hours (Figure~\ref{fig:RECAT_AROTs_boxplot}).

\begin{figure}[h]
    \centering
    \includegraphics[width=1\textwidth]{graphics/fig_RECAT_AROTs_boxplot.png}
    \caption[AROTs box-plot for RECAT-EU categories, all runways]{Arrival Runway Occupancy Times for the different RECAT-EU categories based on data gathered for a period 13 months since October 2017. The coloured blocks indicate where 50\% of the data are located, or the inter-quartile range (IQR); lower edge is the 25\textsuperscript{th}~percentile (Q1), upper edge is the 75\textsuperscript{th}~percentile (Q3). The whiskers are at Q1-1,5$\times$IQR and Q3+1,5$\times$IQR.  The box plot shows the AROTs for all runways at BIKF during peak hours.}
    \label{fig:RECAT_AROTs_boxplot}
\end{figure}

The analysis showed that the shortest AROTs relate to aircraft from the CAT-D and CAT-C with mean values of 75 and 78 seconds respectively, as seen from the statistical analysis in (Table~\ref{tab:AROT_RECAT_stats}). These are also the two main categories that are considered in this study.

\begin{table}[h]
\centering
\resizebox{0.8\textwidth}{!}{%
\begin{tabular}{lr|r|r|r|r|r|r|r|}
\cline{3-9}
                          & \multicolumn{1}{c|}{}   & \multicolumn{7}{c|}{AROT [s]} \\ \hline
\multicolumn{1}{|l|}{RECAT-EU} & count & mean & std & min & 25\% & 50\% & 75\% & max \\ \hline
\multicolumn{1}{|l|}{CAT-A}    & 2     & 119  & 19  & 105 & 112  & 119  & 125  & 132 \\ \hline
\multicolumn{1}{|l|}{CAT-B}    & 53    & 81   & 20  & 51  & 65   & 77   & 91   & 136 \\ \hline
\multicolumn{1}{|l|}{CAT-C}    & 3436  & 78   & 17  & 46  & 65   & 75   & 89   & 155 \\ \hline
\multicolumn{1}{|l|}{CAT-D}    & 1029  & 75   & 18  & 46  & 62   & 71   & 87   & 155 \\ \hline
\multicolumn{1}{|l|}{CAT-E}    & 95   & 88   & 24  & 46  & 71   & 85   & 101  & 151 \\ \hline
\multicolumn{1}{|l|}{CAT-F}    & 49    & 84   & 22  & 48  & 72   & 81   & 91   & 155 \\ \hline
\end{tabular}%
}
\caption[AROTs for the air traffic mix by RECAT]{AROT statistics for the air traffic mix at BIKF by RECAT-EU categories. The count is the number of landings during peak hours since October 2017}
\label{tab:AROT_RECAT_stats}
\end{table}

The numbers also reveal that half of the aircraft from the CAT-C and CAT-D categories have AROTs in the range 60 to 90 seconds approximately, which outlines to some extent the anticipated values for AROT for the airfield. 

AROT is a metric that is expected to have minimal value, within certain safety limits, for increased throughput of the runway. One of the requirements for reducing the radar separation minimum (MRS) from 3~NM to 2,5~NM is AROT $\leq50$ seconds. ICAO Doc 4444 PANS-AM~\cite{doc44444} dictates that MSR may be reduced if the AROT of landing succeeding aircraft, which are established on the same final approach track is proven, by means such as data collection and statistical analysis and/or theoretical models, not to exceed 50 seconds. None of the average AROT values fulfils the 50~seconds limit for reduced MRS. This constraint fixes the MRS reference value at 3 NM and also suggest that the runway occupancy will be a limiting factor for the cases in which MRS is applicable (Table~\ref{tab:RECAT-dist}). This limitation is also confirmed by the statistical analysis for each of the runways in the following sections.% \ref{sssec:seasonal_arot}, \ref{sssec:runway_usage_arot}. 

\subsection{Seasonal Variation of AROT\label{ssec:seasonal_arot}}
Another approach was to analyse the seasonal variations of the runway occupancy time. The differentiation between summer and winter months was based on AROT values for a period of one year (Table~\ref{tab:month2season_arot}). Months with average AROT~$\leq$80~seconds formed the summer season and the rest were selected as winter months. This separation criteria is purely subjective but succeeds in forming two seasons with equal number of months. On average the seasonal difference of runway occupancy times was only eight seconds as seen in Table~\ref{tab:summer_winter_arot}. Still the seasonal variation should be taken into consideration as it can affect the AROT significantly, especially in the winter months when adverse weather conditions may impair the runway surface by accumulated slush, snow or ice, thus diminishing braking action. Good braking action due to runway surface condition is also one of the requirements for reduced MRS as provided by the procedures for navigation services of Air Traffic Management~\cite{doc44444}.

\begin{table}[h]
\centering
\resizebox{0.8\textwidth}{!} & \multicolumn{1}{l|}{50\%} & \multicolumn{1}{l|}{75\%} & \multicolumn{1}{l|}{max} \\ \hline
\multicolumn{1}{|l|}{SUMMER} & \multicolumn{1}{r|}{3727} & 76  & 17 & 46 & 63  & 72 & 87  & 155 \\ \hline
\multicolumn{1}{|l|}{WINTER} & \multicolumn{1}{r|}{937}  & 84  & 20 & 45 & 69  & 84 & 96  & 153 \\ \hline
\end{tabular}%
}
\caption[AROTs for the air traffic mix by season]{AROT statistics for the air traffic mix at BIKF by season. The count is the number of landings during peak hours over a 13 month period starting October 2017.}
\label{tab:summer_winter_arot}
\end{table}

\subsection{Runway Usage and AROT\label{ssec:runway_usage_arot}}
Runway occupancy for each of the runways was also examined both with regard to seasonal variations and rapid-exit taxiway usage. The usage of the four runways during peak hours is shown in Figure~\ref{fig:runway_usage_peak}. 

\begin{figure}[h]
    \centering
    \includegraphics[width=0.8\textwidth]{graphics/fig_runway_usage_peak.png}
    \caption[Runway usage at BIKF during peak hours]{Runway usage at BIKF during peak hours for a period of 13 months starting October 2107. RWY-19 is the most frequently used runway, followed by RWY-01 and RWY-10. The RWY-01 is connected to rapid-exit TWY~A-1 and RWY-28 to rapid-exit TWY~B-1.}
    \label{fig:runway_usage_peak}
\end{figure}

BIKF airfield is currently equipped with two rapid-exit taxiways designated as TWY~A-1 and TWY~B-1. The first was completed on 26~July~2017 and the latter on 4~October~2017. The day that A-1 became operational was chosen as the starting time for the data set considered for analysis in this project. The reason behind this is the beneficial effect that rapid-exits have on reducing arrival runway occupancy time. Taxiway A-1 provides a rapid-exit track to the left for RWY-01, in the north landing direction, and B-1 serves RWY-28 exiting to the right in the west direction. The statistical analysis points to decreased AROT on average for RWY-01 after the start of A-1 (Table~\ref{tab:season_AROT_stats_RWY01_pre_fast_exit},~\ref{tab:season_AROT_stats_RWY01_post_fast_exit}). This decrease was primarily during the winter season (12 seconds) but trivial for the summer months. The data for RWY-28 presented a different picture. The average AROT has been reduced by 31 seconds for the summer months and by 37 seconds for the winter, after the implementation of the rapid-exit (Table~\ref{tab:season_AROT_stats_RWY28_pre_fast_exit},~\ref{tab:season_AROT_stats_RWY28_post_fast_exit}). The minor gain of RWY~01 with TWY~A-1 can be explained with its layout and the fact that A-1 exits into TWY~E-3, meeting taxiing aircraft in the opposite direction (Figure~\ref{fig:BIKF_schematic}), so the rapid-exit was avoided altogether during peak hours.

\begin{figure}[h]
    \centering
    \includegraphics[width=1\textwidth]{graphics/BIKF_schematic.png}
    \caption[BIKF schematic]{BIKF schematic with four marked runways in perpendicular configuration (dashed yellow lines) and two rapid-exit taxiways (red arrows): TWY~A-1 on RWY~01 and TWY~B-1 on RWY~28 (source:~ Isavia).}
    \label{fig:BIKF_schematic}
\end{figure}

Despite the reduced AROT, RWY~28 remained the least used runway, servicing only 10,1$\%$ share of landing aircraft (Figure~\ref{fig:runway_usage_peak}). The preferred runway was RWY~19, servicing 37,8$\%$ of the arrivals. A statistical summary for all the runways is shown in Table~\ref{tab:all_RWY_AROT_stats}. The average AROT for the BIKF airfield amounted to 77,5 seconds.

\begin{table}[]
\centering
\resizebox{0.8\textwidth}{!} & \multicolumn{1}{l|}{50\%} & \multicolumn{1}{l|}{75\%} & \multicolumn{1}{l|}{max} \\ \hline
\multicolumn{1}{|l|}{RWY 01} & 1247 & 86 & 23 & 46 & 66 & 88 & 101 & 153 \\ \hline
\multicolumn{1}{|l|}{RWY 10} & 1185 & 87 & 12 & 61 & 79 & 86 & 93 & 155 \\ \hline
\multicolumn{1}{|l|}{RWY 19} & 1762 & 68 & 10 & 46 & 62 & 66 & 71 & 155 \\ \hline
\multicolumn{1}{|l|}{RWY 28} & 470 & 69 & 14 & 49 & 61 & 66 & 74 & 155 \\ \hline
\end{tabular}%
}
\caption[AROTs during peak hours by runway]{AROT statistics for the air traffic mix at BIKF during peak hours by runway. The count is the number of landings during peak hours from October 2017 to November 2018.}
\label{tab:all_RWY_AROT_stats}
\end{table}


% ---------------------------

\section{Inter-arrival Distance Separation}\label{sec:interarrival_dist_sep_RECAT}

The information for the arrival pairs was fitted into the ICAO WTC scheme in order to recognise the prevailing aircraft pair mix, which is a consequence of the traffic mix discussed previously in \ref{sec:traffic_mix}. Clearly the majority of arrival pairs were classified as Medium-Medium (M-M) as shown in Table~\ref{tab:pairs_mix_to_wtc}. The other noticeable pairs were variations of the Heavy and the Medium categories -- H-H, H-M and M-H. Those four pair types formed the subset of data to be further analysed and split into RECAT-EU categories. The rest of the pairs containing Light aircraft were discarded as being insignificant because of their limited number. 

% Please add the following required packages to your document preamble:
% \usepackage{multirow}
% \usepackage{graphicx}
% \usepackage[table,xcdraw]{xcolor}
% If you use beamer only pass "xcolor=table" option, i.e. \documentclass[xcolor=table]{beamer}
\begin{table}[h]
\centering
\resizebox{0.3\textwidth}{!}{%
\begin{tabular}{cc|r|r|r|}
\cline{3-5}
\multicolumn{1}{l}{} & \multicolumn{1}{l|}{} & \multicolumn{3}{c|}{Follower} \\ \cline{3-5} 
\multicolumn{1}{l}{} & \multicolumn{1}{l|}{} & \multicolumn{1}{c|}{H} & \multicolumn{1}{c|}{M} & \multicolumn{1}{c|}{L} \\ \hline
\multicolumn{1}{|c|}{} & H & \cellcolor[HTML]{FFCC67}56 & \cellcolor[HTML]{FE996B}334 & \cellcolor[HTML]{FFFFC7}2 \\ \cline{2-5} 
\multicolumn{1}{|c|}{} & M & \cellcolor[HTML]{FE996B}368 & \cellcolor[HTML]{FD6864}1809 & \cellcolor[HTML]{FFFFC7}7 \\ \cline{2-5} 
\multicolumn{1}{|c|}{\multirow{-3}{*}{\rotatebox[origin=c]{90}{Leader}}} & L & \cellcolor[HTML]{FFFFC7}3 & \cellcolor[HTML]{FFFFC7}9 & \cellcolor[HTML]{FFFFC7}1 \\ \hline
\end{tabular}%
}
\caption[BIKF traffic mix sorted into ICAO WTC]{Number of ICAO pairs from the traffic mix at BIKF during peak hours, arranged into the corresponding wake categories. The observation period is from October 2017 to November 2018.}
\label{tab:pairs_mix_to_wtc}
\end{table}

The ICAO WTC scheme specifies a distance separation minima as prescribed in  Table~\ref{tab:ICAO_WTC}. The distributions of the distance separations from the selected four pair-types were examined and presented in Figure~\ref{fig:dist_separ_HH_HM_MH_MM_pairs} along with the ICAO reference separation minima. 

\fxnote{Why are some data to the left of the red line???}
\begin{figure}[h]
    \centering
    \includegraphics[width=1\textwidth]{graphics/fig_dist_separ_HH_HM_MH_MM_pairs.png}
    \caption[Distribution of distance separation for ICAO pairs]{Distribution of the WTC distance separation between selected ICAO pair categories. The red reference line indicates the separation minima for a particular pair category. The blue horizontal line indicates 1 standard deviation left and right of the mean.}
    \label{fig:dist_separ_HH_HM_MH_MM_pairs}
\end{figure}



Each of the ICAO pairs were re-categorised, the data-set was reduced by filtering out the pair categories with insufficient number of data points. The mix of arrival pairs from the peak hour traffic at BIKF after re-categorisation is presented in Table~\ref{tab:pairs_mix_to_recat}. As expected from the traffic fleet analysis in \ref{sec:traffic_mix}, most of the aircraft combined into C-C pairs. The rest of the more significant traffic pairs were variations from CAT-C, CAT-D and CAT-E categories.

% Please add the following required packages to your document preamble:
% \usepackage{multirow}
% \usepackage{graphicx}
% \usepackage[table,xcdraw]{xcolor}
% If you use beamer only pass "xcolor=table" option, i.e. \documentclass[xcolor=table]{beamer}
\begin{table}[h]
\centering
\resizebox{0.8\textwidth}{!}{%
\begin{tabular}{cc|c|c|c|c|c|c|}
\cline{3-8}
\multicolumn{1}{l}{} & \multicolumn{1}{l|}{} & \multicolumn{6}{c|}{Follower} \\ \cline{3-8} 
\multicolumn{1}{l}{} & \multicolumn{1}{l|}{} & CAT-A & CAT-B & CAT-C & CAT-D & CAT-E & CAT-F \\ \hline
\multicolumn{1}{|c|}{} & CAT-A &  &  &  & \cellcolor[HTML]{FFFFC7}1 &  &  \\ \cline{2-8} 
\multicolumn{1}{|c|}{} & CAT-B &  & \cellcolor[HTML]{FFFFC7}1 & \cellcolor[HTML]{FFFC9E}17 & \cellcolor[HTML]{FFFC9E}16 & \cellcolor[HTML]{FFFFC7}2 &  \\ \cline{2-8} 
\multicolumn{1}{|c|}{} & CAT-C &  & \cellcolor[HTML]{FFFC9E}19 & \cellcolor[HTML]{FD6864}1697 & \cellcolor[HTML]{FE996B}242 & \cellcolor[HTML]{FFCE93}41 & \cellcolor[HTML]{FFFC9E}14 \\ \cline{2-8} 
\multicolumn{1}{|c|}{} & CAT-D & \cellcolor[HTML]{FFFFC7}1 & \cellcolor[HTML]{FFFC9E}10 & \cellcolor[HTML]{FE996B}229 & \cellcolor[HTML]{FE996B}200 & \cellcolor[HTML]{FFFC9E}10 & \cellcolor[HTML]{FFFFC7}5 \\ \cline{2-8} 
\multicolumn{1}{|c|}{} & CAT-E &  & \cellcolor[HTML]{FFFFC7}1 & \cellcolor[HTML]{FFCE93}44 & \cellcolor[HTML]{FFFFC7}10 &  &  \\ \cline{2-8} 
\multicolumn{1}{|c|}{\multirow{-6}{*}{\rotatebox[origin=c]{90}{Leader}}} & CAT-F &  &  & \cellcolor[HTML]{FFFC9E}16 & \cellcolor[HTML]{FFFFC7}10 & \cellcolor[HTML]{FFFFC7}1 &  \\ \hline
\end{tabular}%
}
\caption[BIKF traffic mix sorted into RECAT-EU categories]{Number of RECAT-EU pairs from the traffic mix at BIKF during peak hours, arranged into the corresponding wake categories. The vast majority of arrival pairs are classified as C-C. The observation period is from October 2017 to November 2018.}
\label{tab:pairs_mix_to_recat}
\end{table}








The resulting pairs from the ICAO H-H pairs were re-categorised primarily as C-C (87,5\%) and C-B (10,7\%) into the RECAT-EU scheme (Table~\ref{fig:HH_to_RECAT_pairs_dist_separ}). It is apparent that for those two pairs the RECAT-EU scheme would decrease the required separation minima by one nautical mile (from 4 NM to 3 NM). The share of the H-H pairs was 2,2\% of all observed pairs at BIKF.

\begin{figure}[h]
    \centering
    \includegraphics[width=1\textwidth]{graphics/fig_HH_to_RECAT_pairs_dist_separ.png}
    \caption[Inter-arrival distance separation of ICAO H-H pairs into the RECAT-EU scheme]{Inter-arrival distance separation after re-categorisation of ICAO H-H pairs into the RECAT-EU scheme. The vertical lines indicate the separation minima in different schemes (ICAO - solid red line, RECAT-EU - dashed red line, MRS - dotted magenta line)}
    \label{fig:HH_to_RECAT_pairs_dist_separ}
\end{figure}

For the aircraft pairs of the ICAO H-M category, the transition to the RECAT-EU scheme will decrease the separation minima requirements more significantly. The reduction for aircraft pairs that were re-categorised as B-C or B-D category is from 5 NM to 4 NM, while for the C-C and C-D pairs this reduction is 2~NM, from 5~NM to 3~NM (Figure~\ref{fig:HM_to_RECAT_pairs_dist_separ}). Here again the C-C pairs were predominant with 73,8\% and C-D pairs with 10,8\%. The share of the B-C and B-D pairs was around 5\%. Provided that the MRS requirements allow for reduced minima, the D-D pairs from the ICAO H-M pair category would potentially result in 2,5~NM reduction, but only three of those pairs were present in the observed data subset (Figure~\ref{fig:HM_to_DD_pairs_dist_separ}). The share of the H-M pair category from the whole BIKF traffic during peak hours is 12,9\%.

\begin{figure}[h]
    \centering
    \includegraphics[width=1\textwidth]{graphics/fig_HM_to_RECAT_pairs_dist_separ.png}
    \caption[Inter-arrival distance separation of ICAO H-M pairs into the RECAT-EU scheme]{Inter-arrival distance separation after re-categorisation of ICAO H-M pairs into the RECAT-EU scheme.}
    \label{fig:HM_to_RECAT_pairs_dist_separ}
\end{figure}

The re-categorisation of  the ICAO M-H (Figure~\ref{fig:MH_to_RECAT_pairs_dist_separ}) and M-M (Figure~\ref{fig:MM_to_RECAT_pairs_dist_separ}) pairs from the data subset shows no apparent advantage for the newly formed RECAT-EU pairs. The traffic in both cases is concentrated in the C-C pairs category. The RECAT-EU reference separation minima does not change its position after re-categorisation to allow for any shift in the separation distribution between aircraft pairs. In both schemes the reference separation was specified as 3 NM, or 2,5 NM when the MRS was applicable. In the case of BIKF the MRS minima was established at 3 NM because of technical characteristics of the radar system, as mentioned earlier. The 2,5 NM dotted line in the figures was drawn just for comparison and not as a reference minima limit.  Pairs from the C-C category comprise 77,9\% of all the ICAO M-H pairs, followed by the D-C pair category with 11,4\%.

\begin{figure}[h]
    \centering
    \includegraphics[width=1\textwidth]{graphics/fig_MH_to_RECAT_pairs_dist_separ.png}
    \caption[Inter-arrival distance separation of ICAO M-H pairs into the RECAT-EU scheme]{Inter-arrival distance separation after re-categorisation of ICAO M-H pairs into the RECAT-EU scheme.}
    \label{fig:MH_to_RECAT_pairs_dist_separ}
\end{figure}

The portion of C-C pairs from the last category, the M-M pairs, is also prevailing with 61,8\%, followed by the C-D (11,4\%), D-D (10,8\%) and D-C (10,4\%) pairs. The lower percentage of the C-C pairs from the ICAO M-M category in comparison with the other ICAO categories after re-categorisation was accompanied by increased ratio of the other pair categories, as noticed in the percentage values and in Figure~\ref{fig:MM_to_RECAT_pairs_dist_separ}. The reference separation minima remains unchanged after re-categorisation, similar to the M-H pairs. This result was also evidence that most of the aircraft traffic mix for Keflavík Airport were in the Medium category.

\begin{figure}[h]
    \centering
    \includegraphics[width=1.0\textwidth]{graphics/fig_MM_to_RECAT_pairs_dist_separ.png}
    \caption[Inter-arrival distance separation of ICAO M-M pairs into the RECAT-EU scheme]{Inter-arrival distance separation after re-categorisation of ICAO M-M pairs into the RECAT-EU scheme.}
    \label{fig:MM_to_RECAT_pairs_dist_separ}
\end{figure}

The share of the M-H pairs from the whole BIKF traffic during peak hours is 14,3\%. The share of the M-M pairs on the other hand is the largest of all ICAO pairs - 69.8\%. The figures above were also evidence that incorporating the RECAT-EU scheme for the selected data subset, would potentially assist the aircraft pairs with a Heavy leader, but would not change the situation for pairs with a Medium leader. 

It is illustrative to mention also that some of the pair categories would suffer an increase in the separation minima in the RECAT-EU scheme. Such example from the traffic at Keflavík would be the C-F and D-F pairs from the ICAO M-M category (Figure~\ref{fig:MM_to_CF_and_DF_pairs_dist_separ}). The reference separation in the former case would double from 3 NM up to 6 NM and in the later case the increase would be from 3 NM to 5 NM. Those pairs were rare in the data set and insufficient to reach any accurate conclusion. Only two pairs from the M-M category for the selected time period were in the D-F category and eight pairs were in the C-F category.

\begin{figure}[h]
    \centering
    \includegraphics[width=1\textwidth]{graphics/fig_MM_to_CF_and_DF_pairs_dist_separ.png}
    \caption[Inter-arrival distance separation of ICAO M-M pairs into the RECAT-EU C-F and D-F pairs]{Inter-arrival distance separation after re-categorisation of ICAO M-M pairs into the RECAT-EU C-F and D-F pairs.}
    \label{fig:MM_to_CF_and_DF_pairs_dist_separ}
\end{figure}




\section{Landing Time Interval}\label{sec:LTI}
The method used to determine the landing time interval (LTI) requires the examination of aircraft pairs -- i.e. leader and follower. The data-set for the analysis contained the ICAO type and WTC for the leader and the follower along with distance separation and time separation between the aircraft in each pair during peak hours. The time frame was identical with the one used in the previous sections. Additionally every leader and follower were re-categorised and assigned a RECAT-EU designator.


Another important metric derived from the data, together with the distance separation between pairs, was the landing time interval (LTI) that indicates whether AROT or the separation requirement is the limiting factor for airfield capacity. As stated in the study objective \ref{sec:arot_and_study_objective} the LTI quantifies the time separation between aircraft in a pair, calculated from the distance separation and the final approach speed of the following aircraft. This measurement was computed for each of the observed pairs and is presented in Figure~\ref{fig:time_separ_HH_HM_MH_MM_pairs}, where the reference line indicates the time that a follower takes to travel the distance minima for the respective pair category. 

The final approach velocity for this calculation was set equal to the maximum average approach velocity from all four runways. This is done by finding the average velocity for each of four runways and using the maximum of the four values as the final approach velocity for the calculation of the LTI. Higher velocity value results in lower reference time. In this sense the reference time is a more liberally specified constraint, as opposed to using the minimum of the average approach velocities. Similar methodology is currently being used by Isavia in determining the runway capacity envelope.

\begin{figure}[h]
    \centering
    \includegraphics[width=1\textwidth]{graphics/fig_time_separ_HH_HM_MH_MM_pairs.png}
    \caption[Distribution of landing time interval (LTI) for ICAO pairs]{Distribution of of landing time interval (LTI) for selected ICAO pair categories. The red line indicates a liberal time reference estimate for a particular pair category. The blue horizontal line indicates 1 standard deviation left and right of the mean.}
    \label{fig:time_separ_HH_HM_MH_MM_pairs}
\end{figure}



\section{Constraints for Increased Capacity}% Runway Occupancy and Landing Time Interval for RECAT-EU pairs}

The following section presents the landing time intervals (LTI) for a subset of data points after re-categorisation from the ICAO wake turbulence scheme. The subset contained primarily C-C pairs. The C-C pairs were selected on the basis of being the more prominent subset with most of the data points. The LTI for the ICAO pair sets from \ref{sec:LTI} were modified to represent only the C-C pairs, where the effect on the LTI distribution after re-categorisation would be more accurate. The other metric used in contrast to the LTI was the arrival runway occupancy time for the respective aircraft pairs. The frequency distribution of the AROT included the runway occupancy for all the leaders of the selected ICAO category in order to make use of more data points and obtain a more general view on the AROT distribution. This decision was based on the assumption that the AROT of the leader in an aircraft pair is the constraint for the time separation of follower. This is illustrated in the result for the LTI and the AROT for the C-C pairs from H-H pair category in Figure~\ref{fig:CC_from_HH_pairs_time_sep}. 

\begin{figure}[h]
    \centering
    \includegraphics[width=1\textwidth]{graphics/fig_CC_from_HH_pairs_time_sep.png}
    \caption[AROT and LTI of C-C pairs originating from ICAO H-H pairs]{AROT and landing time intervals of C-C pairs originating from ICAO H-H pairs. The dashed red line indicates the RECAT-EU reference time separation for the C-C pairs. The blue horizontal line indicates 1 standard deviation left and right of the mean.}
    \label{fig:CC_from_HH_pairs_time_sep}
\end{figure}

The LTI frequency distribution contained all C-C pairs originating from the ICAO H-H category. The reference time separation (red dashed line) was calculated from the final approach speed and the RECAT-EU distance separation, as explained in \ref{sec:LTI}. The frequency distribution for the AROT, on the other hand, contained all pairs with CAT-C leader originating from the Heavy category, not only pairs with CAT-C leader and CAT-C follower. In this sense the distribution for the AROT was based on more data points than the distribution for the LTI and gave a more accurate presentation of the runway occupancy. Figure~\ref{fig:CC_from_HH_pairs_time_sep} represents the case for 1,9\% of all arrival pairs at BIKF for the observed period during peak hours.

The share of C-C pairs originating from the ICAO H-M category from the peak traffic at the airport was 9,5\%.  The LTI frequency distribution of those pairs is shown in Figure~\ref{fig:CC_from_HM_pairs_time_sep}. Here again the AROT distribution was based on the runway occupancy of CAT-C leaders, as in the previous case. 
 
\begin{figure}[h]
    \centering
    \includegraphics[width=1\textwidth]{graphics/fig_CC_from_HM_pairs_time_sep.png}
    \caption[AROT and LTI of C-C pairs originating from ICAO H-M pairs]{AROT and landing time intervals of C-C pairs originating from ICAO H-M pairs. The dashed red line indicates the RECAT-EU reference time separation for the C-C pairs. The blue horizontal line indicates 1 standard deviation left and right of the mean.}
    \label{fig:CC_from_HM_pairs_time_sep}
\end{figure}

These two cases: C-C pairs from ICAO H-H and H-M categories, would have more noticeable beneficial effect from the implementation of the RECAT-EU scheme (refer to~\ref{sec:interarrival_dist_sep_RECAT}). The decrease in distance separation in those cases is 1 NM or 2 NM. Translating the separation into the time domain resulted in a separation reference time line at 74 seconds and minimum LTI values close to 100 seconds. This shift of the reference line creates the potential for shift to the left in the frequency distribution of the LTI.
 
 \begin{figure}[h]
    \centering
    \includegraphics[width=1\textwidth]{graphics/fig_CD_from_HM_pairs_time_sep.png}
    \caption[AROT and LTI of C-D pairs originating from ICAO H-M pairs]{AROT and landing time intervals of C-D pairs originating from ICAO H-M pairs. The dashed red line indicates the RECAT-EU reference time separation for the C-D pairs. The blue horizontal line indicates 1 standard deviation left and right of the mean.}
    \label{fig:CD_from_HM_pairs_time_sep}
\end{figure}

Even greater potential for decrease of the inter-arrival time was observed in the case of C-D pairs from the ICAO H-M category (Figure~\ref{fig:CD_from_HM_pairs_time_sep}). Here the minimum LTI observed was 113 seconds, serving as a possible 39 seconds shift to the left of the frequency distribution. However the share of those pairs in the traffic was limited to 1.4\%.

The C-C pairs formed from the ICAO M-H category were 11,1\% from the BIKF traffic with frequency distribution of the LTI shown in Figure~\ref{fig:CC_from_MH_pairs_time_sep}. The reference line in this case is again at 74 seconds, and the LTI frequency distribution curve was closer to the reference than in the cases above.

\begin{figure}[h]
    \centering
    \includegraphics[width=1\textwidth]{graphics/fig_CC_from_MH_pairs_time_sep.png}
    \caption[AROT and LTI of C-C pairs originating from ICAO M-H pairs]{AROT and landing time intervals of C-C pairs originating from ICAO M-H pairs. The dashed red line indicates the RECAT-EU reference time separation for the C-C pairs. The blue horizontal line indicates 1 standard deviation left and right of the mean.}
    \label{fig:CC_from_MH_pairs_time_sep}
\end{figure}

The majority of the C-C pairs of the BIKF traffic were formed from the ICAO M-M pairs, or 43\%. Here the minimum inter-arrival time separation was 80 seconds, a mere 6 seconds above the reference limit, with half of all pairs having LTI between 132 and 185 seconds.

\begin{figure}[h]
    \centering
    \includegraphics[width=1\textwidth]{graphics/fig_CC_from_MM_pairs_time_sep.png}
    \caption[AROT and LTI of C-C pairs originating from ICAO M-M pairs]{AROT and landing time intervals of C-C pairs originating from ICAO M-M pairs. The dashed red line indicates the RECAT-EU reference time separation for the C-C pairs. The blue horizontal line indicates 1 standard deviation left and right of the mean.}
    \label{fig:CC_from_MM_pairs_time_sep}
\end{figure}

Nevertheless the potential for decrease of the landing time interval was hindered by the AROT, the other major factor affecting runway capacity (refer to \ref{sec:runway_capacity}). All of the distributions in the figures above witness an overlap of the reference time separation line and the AROT frequency distribution. The average runway occupancy time was estimated as 77,5 seconds (\ref{ssec:runway_usage_arot}) and compared to the reference time line for the predominant C-C pairs (74 seconds), it is apparent that the AROT is the limiting factor and a major constraint for shift in the LTI distribution.








% \lipsum[28-34]

%%% Local Variables: 
%%% mode: latex
%%% TeX-master: "DEGREE-NAME-YEAR"
%%% End: 
%%RUM: "Results"





\chapter{Summary}\label{cha:summary}
This study uses measured data from the ADS-B surveillance system at Keflavík International Airport~(BIKF). The analysis of the arrival runway occupancy time~(AROT) and the wake turbulence separation aims to validate the constraints of implementing the RECAT-EU wake  separation scheme for BIKF. The focus is on arrival aircraft pairs during peak hours and the observed time period is thirteen months (from 4 October 2017 to 30 November 2018). 
Airfield capacity is influenced by two major factors: runway occupancy and wake turbulence separation. Already, the analysis of those metrics has been performed by Isavia within the current ICAO wake turbulence scheme. Wake turbulence categories in use at BIKF and the arrival runway occupancy times show that in most of the cases the AROT is a limiting factor for increase of the airfield throughput.  

Different variables affect the AROT, such as the existence and usage of rapid-exit taxiways, weather conditions and the traffic mix. Two rapid-exit taxiways became operational in 2017 (taxiway TWY~A-1 and TWY~B-1). The usage of TWY~B-1 on runway RWY~28 has managed to decrease the AROT by more than 33\% (37 seconds) on average during the winter and 31\% (31 seconds) during the summer months. The seasonal variations of the AROT were also examined and showed a difference of almost 10\% on average. The weather conditions determine runway surface conditions that affect braking action, the strength and direction of winds and visibility. One explanation of the preferred usage of RWY~19 over the other runways could be the frequent winds in northern direction (\ref{fig:BIKF_wind_rose}), facilitating decreased AROT for the runway. The traffic fleet at BIKF was primarily from the CAT-C and CAT-D wake categories that combined represent 96\% of the arrivals and were so far the aircraft with the smallest AROT on average. Nevertheless the above conditions were insufficient to produce AROT~$\leq~50$ seconds, which is one requirement for reducing the radar separation minima (MRS) for aircraft on final approach to the 2,5 NM minimum. The average arrival runway occupancy time for BIKF was 77,5 seconds. 

The runway occupancy in turn has effect on the landing time intervals, which is also governed by the wake turbulence requirements.
The analysis of the re-categorisation requirements for BIKF show that for certain pairs the wake turbulence separation could be reduced by~1~-~2~NM, while for a few others the separation would increase by~2~-~3~NM. Reduced wake turbulence separation will affect most of the Heavy-Heavy and the Heavy-Medium pairs. Separation will increase for Medium-Medium pairs that are re-categorised as C-F and D-F pairs, but the number of those cases for Keflavík Airport is relatively small.

The majority of the fleet mix though will experience no change of wake turbulence separation with the implementation of the RECAT-EU scheme. The separation requirement for the most of the ICAO Medium-Medium pairs remains 3~NM after re-categorisation and the same is true for Medium-Heavy pairs. The combined share of the aircraft that will experience no change to required wake turbulence separation was around 84\%. 

The wake distance separation was transferred into time separation and defined as the landing time interval~(LTI), in order to compare it to arrival runway occupancy time.
The correlation of arrival runway occupancy and landing time intervals for the predominant C-C pairs revealed that even though a decrease of the LTI is apparently possible, the AROT is a limiting factor for increased runway throughput. Not only do the landing frequency distributions of the two metrics overlap to some extent but also the reference time separation is well within the AROT range. One way the conflict between the AROT and LTI can manifest itself is through a missed approach (ICAO: bulked landing), when an aircraft landing cannot be completed due to runway incursion \cite{doc44444}: the incorrect presence of another aircraft on the runway designated for landing.

\chapter{Conclusion\label{cha:conclusions}}
The analysis of landing aircraft at Keflavík International, when the airport operates at high loads, shows that rapid-exits on runways have a strong influence of arrival runway occupancy times. Optimised position of rapid-exits in context with the traffic fleet can reduce AROT by 1/3.% The reduction can be greater still if the weather conditions are favourable, even though its effect of is intermittent and not decisive for AROT.

Under the RECAT-EU scheme, the wake turbulence separation will remain unchanged for the majority of the aircraft arriving at BIKF. A predominant part of the traffic mix is currently classified as Medium within the ICAO wake turbulence categories and after re-categorisation the reference separation between most of the aircraft pairs remains 3 NM. The observation from the frequency distribution of the landing time intervals is clearly the wide range over which the times are stretched. If the frequency distribution is compressed, this will create opportunity for overall decrease of the inter-arrival time intervals by shifting the distribution closer to the required minimum separation. This possibility can potentially accommodate more landings per time interval and raise runway capacity.

Arrival runway occupancy times and wake turbulence separation requirements interact to formulate the capacity of an airfield. In the case with Keflavík Airport, high arrival runway occupancy times are an obstacle for increased runway capacity, despite the prospect of decreased wake turbulence separation within the RECAT-EU scheme. The observed arrival runway occupancy time is a limiting factor for increase of the runway throughput for the selected aircraft pairs from the current fleet mix at Keflavík Airport.

Although the study examines the constraints of the airfield in connection with RECAT, it only considers the metrics for arriving aircraft. The analysis becomes more complex and involved when arrival-departure sequencing is considered. Estimating the constraints on airfield capacity for each of the runways with regard to arrivals and departures is a subject for future work. Creating a simulation of the arrival-departure sequencing and assessing the effect of wind and weather conditions on runway occupancy times can also be considered a continuation of this work.

% \lipsum[35-41]


% \lipsum[42-43]
% \lipsum[44-50]
%%% Local Variables: 
%%% mode: latex
%%% TeX-master: "DEGREE-NAME-YEAR"
%%% End: 
%%RUM: "Discussion"

%% ---------------------------------------------------------------
\printbibliography{} %%RUM: "References"

%% If appendices are needed, uncomment the following line
%% and include the appendices in separate files
\appendix{}%%RUM: "Appendicies (as appropriate)



\chapter{Runway Usage}\label{app:RWY_usage}


\begin{figure}[h]
    \centering
    \includegraphics[width=0.8\textwidth]{graphics/fig_runway_usage_2017-10-04_to_2018-11-30.png}
    \caption[Runway usage at BIKF]{Overall runway usage at BIKF for a period of one year since October 2017. Almost half of all arrivals during that period (44,7\%) use RWY-19, followed by RWY-01 (25,7\%) and RWY-10 (17.6\%). The summer traffic favoured RWY-19 with 52.8\%~(Figure~\ref{fig:runway_usage_summer}), while during the winter season some of those arrivals were diverted towards RWY-01 and RWY-10~(Figure~\ref{fig:runway_usage_summer}). RWY-28 was least used despite its rapid-exit TWY B-1.}
    \label{fig:runway_usage}
\end{figure}

\begin{figure}[h]
    \centering
    \includegraphics[width=0.7\textwidth]{graphics/fig_runway_usage_summer}
    \caption[Summer runway usage at BIKF]{The summer traffic favours RWY-19 (52,8\%), followed by RWY-01 (23,3\%), while RWY-10 and RWY-28 are with 11,6\% and 12,3\% respectively.}
    \label{fig:runway_usage_summer}
\end{figure}

\begin{figure}[h]
    \centering
    \includegraphics[width=0.7\textwidth]{graphics/fig_runway_usage_winter}
    \caption[Winter runway usage at BIKF]{The winter traffic at BIKF is slightly more evenly distributed among runways but still favours RWY-19 (38,8\%),followed by RWY-01 with 28,9\%, RWY-10 (20,8\%) and RWY-28 (11,4\%).}
    \label{fig:runway_usage_winter}
\end{figure}

\clearpage
\chapter{Arrival Runway Occupancy Times\label{app:AROTs}}





% Please add the following required packages to your document preamble:
% \usepackage{graphicx}
\begin{table}[h]
\centering
\resizebox{0.8\textwidth}{!} & \multicolumn{1}{l|}{50\%} & \multicolumn{1}{l|}{75\%} & \multicolumn{1}{l|}{max} \\ \hline
\multicolumn{1}{|l|}{RWY 01} & 963 & 85 & 21 & 46 & 68 & 88 & 100 & 152 \\ \hline
\multicolumn{1}{|l|}{RWY 10} & 852 & 85 & 11 & 61 & 78 & 84 & 91 & 155 \\ \hline
\multicolumn{1}{|l|}{RWY 19} & 1511 & 67 & 10 & 46 & 61 & 66 & 71 & 155 \\ \hline
\multicolumn{1}{|l|}{RWY 28} & 401 & 68 & 14 & 49 & 60 & 66 & 72 & 155 \\ \hline
\end{tabular}%
}
\caption[AROTs for the air traffic mix by runway for the summer]{AROT statistics for the air traffic mix at KEF by runway for the summer of 2018. The count is the number of landings during peak hours.}
\label{my-label3}
\end{table}

% Please add the following required packages to your document preamble:
% \usepackage{graphicx}
\begin{table}[h]
\centering
\resizebox{0.8\textwidth}{!} & \multicolumn{1}{l|}{50\%} & \multicolumn{1}{l|}{75\%} & \multicolumn{1}{l|}{max} \\ \hline
\multicolumn{1}{|l|}{RWY 01} & 290 & 88 & 26 & 47 & 63 & 90 & 106 & 153 \\ \hline
\multicolumn{1}{|l|}{RWY 10} & 338 & 93 & 12 & 70 & 85 & 91 & 100 & 144 \\ \hline
\multicolumn{1}{|l|}{RWY 19} & 259 & 72 & 13 & 51 & 65 & 69 & 75 & 140 \\ \hline
\multicolumn{1}{|l|}{RWY 28} & 69 & 73 & 13 & 49 & 63 & 72 & 79 & 105 \\ \hline
\end{tabular}%
}
\caption[AROTs for the air traffic mix by runway for the winter]{AROT statistics for the air traffic mix at KEF by runway for the winter season. The count is the number of landings during peak hours since October 2017}
\label{my-label4}
\end{table}

% Please add the following required packages to your document preamble:
% \usepackage{graphicx}
% \usepackage[table,xcdraw]{xcolor}
% If you use beamer only pass "xcolor=table" option, i.e. \documentclass[xcolor=table]{beamer}
\begin{table}[h]
\centering
\resizebox{0.8\textwidth}{!} & \multicolumn{1}{l|}{50\%} & \multicolumn{1}{l|}{75\%} & \multicolumn{1}{l|}{max} \\ \hline
\rowcolor[HTML]{DAE8FC} 
\multicolumn{1}{|l|}{\cellcolor[HTML]{DAE8FC}January} & 95 & 86 & 20 & 54 & 70 & 85 & 96 & 136 \\ \hline
\rowcolor[HTML]{DAE8FC} 
\multicolumn{1}{|l|}{\cellcolor[HTML]{DAE8FC}February} & 80 & 87 & 16 & 52 & 75 & 86 & 93 & 144 \\ \hline
\rowcolor[HTML]{DAE8FC} 
\multicolumn{1}{|l|}{\cellcolor[HTML]{DAE8FC}March} & 215 & 84 & 18 & 47 & 70 & 85 & 95 & 150 \\ \hline
\rowcolor[HTML]{DAE8FC} 
\multicolumn{1}{|l|}{\cellcolor[HTML]{DAE8FC}April} & 221 & 82 & 17 & 47 & 69 & 81 & 93 & 139 \\ \hline
\rowcolor[HTML]{FFFC9E} 
\multicolumn{1}{|l|}{\cellcolor[HTML]{FFFC9E}May} & 386 & 75 & 13 & 46 & 67 & 74 & 83 & 140 \\ \hline
\rowcolor[HTML]{FFFC9E} 
\multicolumn{1}{|l|}{\cellcolor[HTML]{FFFC9E}June} & 647 & 73 & 16 & 47 & 62 & 68 & 80 & 155 \\ \hline
\rowcolor[HTML]{FFFC9E} 
\multicolumn{1}{|l|}{\cellcolor[HTML]{FFFC9E}July} & 729 & 72 & 16 & 46 & 61 & 68 & 77 & 155 \\ \hline
\rowcolor[HTML]{FFFC9E} 
\multicolumn{1}{|l|}{\cellcolor[HTML]{FFFC9E}August} & 724 & 80 & 18 & 46 & 64 & 81 & 91 & 148 \\ \hline
\rowcolor[HTML]{FFFC9E} 
\multicolumn{1}{|l|}{\cellcolor[HTML]{FFFC9E}September} & 632 & 78 & 18 & 46 & 64 & 75 & 88 & 152 \\ \hline
\rowcolor[HTML]{FFFC9E} 
\multicolumn{1}{|l|}{\cellcolor[HTML]{FFFC9E}October} & 609 & 79 & 17 & 46 & 66 & 76 & 91 & 155 \\ \hline
\rowcolor[HTML]{DAE8FC} 
\multicolumn{1}{|l|}{\cellcolor[HTML]{DAE8FC}November} & 239 & 84 & 23 & 48 & 66 & 82 & 99 & 151 \\ \hline
\rowcolor[HTML]{DAE8FC} 
\multicolumn{1}{|l|}{\cellcolor[HTML]{DAE8FC}December} & 106 & 87 & 23 & 51 & 69 & 86 & 99 & 153 \\ \hline
\end{tabular}%
}
\caption[AROTs for the air traffic mix by month]{AROT statistics for the air traffic mix at BIKF by month. The count is the number of landings during peak hours from October 2017 til November 2018. The colour fields indicate a subjective separation of the data into summer and winter season, based on mean AROT value. Months with mean AROT $\leq$ 80 seconds were classified as summer, and the remaining as winter.}
\label{tab:month2season_arot}
\end{table}

\clearpage
\chapter{Inter-arrival time and distance separation}\label{app:time_and_dist_sep}

\begin{figure}[h]
    \centering
    \includegraphics[width=0.5\textwidth]{graphics/fig_HM_to_DD_pairs_dist_separ.png}
    \caption[Inter-arrival distance separation of H-M pairs to D-D pairs into the RECAT-EU scheme]{Inter-arrival distance separation after re-categorisation of H-H pairs to D-D pairs into the RECAT-EU scheme.}
    \label{fig:HM_to_DD_pairs_dist_separ}
\end{figure}


% Please add the following required packages to your document preamble:
% \usepackage{multirow}
% \usepackage{graphicx}
\begin{table}[h]
\centering
\resizebox{0.8\textwidth}{!}{%
\begin{tabular}{|c|c|c|c|c|c|c|c|}
\hline
\multicolumn{2}{|c|}{\multirow{2}{*}{RECAT-EU scheme}} & \multicolumn{6}{c|}{Follower}                   \\ \cline{3-8} 
\multicolumn{2}{|c|}{}& CAT-A & CAT-B & CAT-C & CAT-D & CAT-E & CAT-F \\ \hline
\multirow{6}{*}{\rotatebox[origin=c]{90}{Leader}}& CAT-A&& 100s  & 120s  & 140s  & 160s  & 180s  \\ \cline{2-8}
& CAT-B&&&& 100s  & 120s  & 140s  \\ \cline{2-8} 
& CAT-C&&&& 80s   & 100s  & 120s  \\ \cline{2-8} 
& CAT-D&&&&&& 120s  \\ \cline{2-8} 
& CAT-E&&&&&& 100s  \\ \cline{2-8} 
& CAT-F&&&&&& 80s   \\ \hline
\end{tabular}%
}
\caption[RECAT-EU time-based separation minima]{RECAT-EU WT time-based separation minima on approach and departure~\cite{rooseleer2015recat}}
\label{tab:RECAT-time}
\end{table}



\begin{figure}[h]
    \centering
    \includegraphics[width=1.0\textwidth]{graphics/fig_best_CC_from_HM_pairs_time_sep.png}
    \caption[AROT and LTI of C-C pairs from ICAO H-M pairs on RWY~19]{AROT and landing time intervals of C-C pairs originating from ICAO H-M pairs for RWY-19 during the summer season. The dashed red line indicates the RECAT-EU reference time separation for the C-C pairs.}
    \label{fig:best_CC_from_HM_pairs_time_sep}
\end{figure}


\clearpage
\chapter{ADS-B Standard Data Items}\label{app:adsb_items}

\begin{figure}[h]
    \centering
    \includegraphics[width=0.7\textwidth]{graphics/ads_b_items.png}
    \caption[ADS-B Standard Data Items]{ADS-B Standard Data Items \cite[p. 8]{ASTERIX_ADS-B_specs}.}
    \label{fig:adsb_items}
\end{figure}


\clearpage
\chapter{BIKF Wind Rose}\label{app:wind_rose}

\begin{figure}[h]
    \centering
    \includegraphics[width=0.7\textwidth]{graphics/BIKF_windrose_2005-2014.pdf}
    \caption[BIKF Wind Rose]{Wind rose - frequency, direction and strength of winds at BIKF aerodrome \cite{wind_rose_2014}.}
    \label{fig:BIKF_wind_rose}
\end{figure}
% ----------------------------


% \chapter{Code}\label{cha:code}
% You can put code in your document using the listings package, which is
% loaded by default in \path{custom.tex}.  Be aware that the listings
% package does not put code in your document if you are in draft mode
% unless you set the \texttt{forcegraphics} option.

% There is an example java (Listing~\ref{src:Data_Bus.java}) and XML
% file (Listing~\ref{src:AndroidManifest.xml}).  Thanks to the
% \texttt{url} package, you can typeset OSX and unix paths like this:
% \path{/afs/rnd.ru.is/project/thesis-template}.  Windows paths:
% \path{C:\windows\temp\ }.  You can also typeset them using the menukey
% package, but it tends to delete the last separator and has other
% complications.\footnote{The menukey package has issues with biblatex,
%   read \path{custom.tex} for more information.}

% If you are trying to include multiple different languages, you should
% go read the documentation and set these up in \path{custom.tex}.  You
% will save yourself a lot of effort, especially if you have to fix
% anything.

% %I have put the source code in the \directory{src/} folder.
% \lstinputlisting[language=Java, firstline=1,
% lastline=40, caption={Data\_Bus.java: Setting up the class.},
% label={src:Data_Bus.java}]{src/Data_Bus.java}

% \lstinputlisting[language={[android]XML}, firstline=1, lastline=20,
% caption={AndroidManifest.xml: Configuration for the Android UI.},
% label={src:AndroidManifest.xml}]{src/AndroidManifest.xml}

%%% Local Variables: 
%%% mode: latex
%%% TeX-master: "DEGREE-NAME-YEAR"
%%% End: 
 % as an example, perhaps some of your code
% \input{app2}
%\backmatter{} % Sections after this don't get numbers
%% We prefer that all elements be numbered

%%%%%%%%%%%%% SHOW INDEX %%%%%%%%%%%%%%%%%%
%% Index, optional.  A good idea on longer documents

% You can put instructions at the beginning of the index:
%\renewcommand{\preindexhook}{%
%  The first page number is usually, but not always,
%  the primary reference to the indexed topic.\vskip\onelineskip}

%% You may have to run "makeindex <FILENAME>" to have it be generated
%% Depending upon which package you chose.
%% 
\clearforchapter{}
\printindex{}%%RUM: Not mentioned

%\backcover{}%%RUM: "Back cover (only Phd)
\end{document}

%% ---------------------------------------------------------------

%%% Local Variables:
%%% mode: latex
%%% TeX-master: t
%%% TeX-engine: xetex
%%% End:




\chapter{Summary}\label{cha:summary}

Keflavík International Airport (BIKF) has two functioning runways operating in both directions. The airport is serviced by Isavia and in recent years has experienced constrained runway capacity during peak hours because of intensified aircraft traffic. Currently the measured equilibrium capacity for the airfield is four arrivals to four departures per 15 minute interval. 

Two major factors affect airport capacity: wake separation requirements and runway occupancy time. The wake  separation between aircraft in flight due to wake turbulence is an essential safety measure because of the effect of wing generated wake vortices on air masses. ICAO has specified wake turbulence separation minima requirements that are applied to an aircraft in the flight-path of another aircraft. Those requirements classify each aircraft in a wake turbulence category and the wake turbulence separation is applied to a "follower" aircraft associated with the wake category of a "leader" aircraft, thus the aircraft form a leader--follower pair, that observe a certain spacing in nautical miles (NM). The present ICAO wake separation scheme used at Keflavík comprises three aircraft categories (Heavy, Medium and Light) and is generally considered outdated, leading to over-separation between aircraft in some cases and affecting runway throughput. 

A more recent RECAT-EU wake separation scheme developed by EUROCONTROL splits aircraft into six categories (CAT-A, CAT-B, CAT-C, CAT-D, CAT-E and CAT-F), for the most part shortens the separation requirements between some aircraft pairs and in general leads to accommodating more landing and departures during peak traffic periods and increased airport capacity. 

The other major factor influencing airfield capacity is runway occupancy time (ROT), which is the length of time an aircraft is present on the runway. In this study only the arrival runway occupancy time (AROT) is considered. The AROT is regarded as a limiting factor for capacity when a landing cannot be completed due to incursion -- the incorrect presence of another aircraft on the runway designated for landing. 

The distance separation between landing aircraft in a pair is transferred into time separation and defined as the landing time interval (LTI), in order to be able to compare it to the runway occupancy time, which is measured in units of seconds. Overlap of the distributions of AROT and LTI can lead to incursions, establishing the AROT as a limiting factor for increased capacity.
Already, the analysis performed by Isavia within the current ICAO wake turbulence scheme points to AROT as a limiting factor for increased airfield throughput in most of the cases. 

Different elements affect the arrival runway occupancy time, such as existence and usage of rapid-exit taxiways, weather conditions and the traffic mix. Two rapid-exit taxiways became operational in 2017 (taxiway TWY~A-1 and TWY~B-1). The usage of TWY~B-1 on runway RWY~28 has managed to decrease the AROT by more than 33\% (from 110 to 73 seconds) on average during the winter and 31\% (from 99 to 66 seconds) during the summer months. The seasonal variations of the AROT were also examined and show a difference of almost 10\% on average. The weather conditions determine runway surface conditions that affect braking action, the strength and direction of winds and visibility. The average AROT for the BIKF airfield is currently 78 seconds with standard
deviation about the mean of 18 second.

% The traffic fleet at BIKF was primarily from the CAT-C and CAT-D wake categories that combined represent 96\% of the arrivals and were so far the aircraft with the smallest AROT on average. Nevertheless the above conditions were insufficient to produce AROT~$\leq~50$ seconds, which is one requirement for reducing the radar separation minima (MRS) for aircraft on final approach to the 2,5 NM minimum. The average arrival runway occupancy time for BIKF was 78 seconds. 

% The runway occupancy in turn has effect on the landing time intervals, which is also governed by the wake turbulence requirements.
The task at hand is to evaluate whether adopting the RECAT-EU scheme is a feasible option for Keflavík Airport that could lead to increased capacity, as well as estimate the constraints of the airfield. 
The analysis of the RECAT-EU re-categorisation requirements for BIKF, show that for certain pairs the wake turbulence separation could be reduced by~1--2~NM, while for a few others the separation would increase by~2--3~NM. On the other hand, the re-categorisation will have no effect on the separation requirement for majority of the aircraft pairs (83,5\%) from the traffic mix at BIKF during peak hours, which suggests that for those aircraft the status is unchanged and AROT remains a limiting capacity factor. 

Reduced wake turbulence separation requirements apply mostly to pairs currently categorised as Heavy-Heavy and Heavy-Medium. 
The pairs that will encounter beneficial change in separation requirement comprise 14,9\% of the traffic during peak hours. For those pairs the reference time for landing aircraft pairs could be reduced by 36,2\%, or down to 74 seconds. However, this separation minima requirement is in conflict with the average AROT for the airfield, indicating that that the arrival runway occupancy time is still the limiting factor for potential increase of airfield capacity (Table~\ref{tab:limit_capacity}). The separation requirements cannot be reduced by adopting the RECAT-EU scheme without first reducing the arrival occupancy time.

\begin{table}[h]
\centering
\resizebox{0.7\textwidth}{!}{%
\begin{tabular}{cc|c|c|c|c|c|c|}
\cline{3-8}
 &  & \multicolumn{6}{c|}{Follower} \\ \cline{3-8} 
 &  & CAT-A & CAT-B & CAT-C & CAT-D & CAT-E & CAT-F \\ \hline
\multicolumn{1}{|c|}{} & CAT-A &  &  &  &  &  &  \\ \cline{2-8} 
\multicolumn{1}{|c|}{} & CAT-B & (*) &  &  &  &  &  \\ \cline{2-8} 
\multicolumn{1}{|c|}{} & CAT-C & (*) & (*) & \cellcolor[HTML]{FFEDCC}AROT & \cellcolor[HTML]{FFEDCC}AROT & \cellcolor[HTML]{FFEDCC}AROT &  \\ \cline{2-8} 
\multicolumn{1}{|c|}{} & CAT-D & (*) & (*) & \cellcolor[HTML]{FFEDCC}AROT & \cellcolor[HTML]{FFEDCC}AROT & (*) &  \\ \cline{2-8} 
\multicolumn{1}{|c|}{} & CAT-E & (*) & (*) & \cellcolor[HTML]{FFEDCC}AROT & (*) & (*) &  \\ \cline{2-8} 
\multicolumn{1}{|c|}{\multirow{-6}{*}{\rotatebox[origin=c]{90}{Leader}}} & CAT-F & (*) & (*) & (*) & (*) & (*) &  \\ \hline
\end{tabular}%
}
\caption[Limiting capacity factors]{Limiting airfield capacity factor under RECAT-EU for Keflavík Airport during peak hours is repeatedly the arrival runway occupancy time. Estimate for the AROT and LTI distributions reveal an overlap of the two metrics after re-categorisation. (*) indicates aircraft pairs observing minimum radar separation (MRS) -- a reference value set at 3 NM for BIKF, again due to AROT being a limiting factor.}
\label{tab:limit_capacity}
\end{table}




\chapter{Conclusion\label{cha:conclusions}}

This study uses measured data from the ADS-B surveillance system at Keflavík International Airport~(BIKF) to estimate the distribution of arrival runway occupancy time~(AROT) and the wake turbulence separation and aims to validate the constraints of implementing the RECAT-EU wake  separation scheme for BIKF. The focus is on arrival aircraft pairs during peak hours and the observed time period is fourteen months -- from 4 October 2017 to 30 November 2018. The wake distance separation was transferred into time separation and defined as the landing time interval~(LTI), in order to correlate with the arrival runway occupancy time, both measured in units of seconds.
From the results of this study the following conclusions can be made:
\begin{itemize}

    \item The traffic mix at the airport reveals that the majority of flights (85,4\%) are in the Medium wake category and the rest are mainly in the Heavy category (14,2\%) with less than one percent Light aircraft. The traffic mix is relevant as it determines the portion of aircraft pairs that experience change of the separation requirement under the RECAT-EU re-categorisation scheme.

    \item The analysis of landing aircraft at Keflavík International, when the airport operates at high loads, shows that rapid-exits on runways have a strong influence on arrival runway occupancy times. Optimised position of rapid-exits in context with the traffic fleet can reduce AROT by 1/3. The usage of TWY~B-1 on runway RWY~28 has managed to decrease the AROT from 110 to 73 seconds on average during the winter and from 99 to 66 seconds during the summer months.
    
    \item The majority of the fleet mix will experience no change of wake turbulence separation requirement with the implementation of the RECAT-EU scheme. The separation requirement for most of the ICAO Medium-Medium pairs remains unchanged (3~NM) after re-categorisation which is also true for Medium-Heavy pairs. The combined share of the aircraft that experience no change to required wake turbulence separation is 83,5\%. The pairs that encounter beneficial change in separation requirement comprise 14,9\% of the traffic during peak hours.
    
    \item The landing time interval distribution of re-categorised pairs revealed that even though a decrease of the LTI is apparently possible for some aircraft pairs, it is restrained by the AROT distribution. Not only do the landing frequency distributions of the two metrics overlap to some extent but also the reference time line is well within the AROT range. One way the conflict between the AROT and LTI can manifest itself is through a missed approach due to the incorrect presence of another aircraft on the runway designated for landing. One observation from the frequency distribution of the landing time intervals is the wide range over which the times are spread. If the frequency distribution is compressed, this will create opportunity for overall decrease of the inter-arrival time intervals by shifting the distribution closer to the required minimum separation. This possibility can potentially accommodate more landings per time interval and raise runway capacity.
    
    \item Arrival runway occupancy times and wake turbulence separation requirements interact to formulate the capacity of an airfield. In the case with Keflavík Airport, high arrival runway occupancy times are an obstacle for increased runway capacity, despite the prospect of decreased wake turbulence separation within the RECAT-EU scheme. The observed arrival runway occupancy time is a limiting factor for increase of the runway throughput for the selected aircraft pairs from the current fleet mix at Keflavík Airport.
    
    
    
\end{itemize}

% \chapter{Conclusion-WORK IN PROGRESS\label{cha:conclusions}}
% The reduction can be greater still if the weather conditions are favourable, even though its effect of is intermittent and not decisive for AROT.

% Under the RECAT-EU scheme, the wake turbulence separation requirement will remain unchanged for the majority of the aircraft pairs arriving at BIKF during peak hours. 

% A predominant part of the traffic mix (69,9\%) is currently classified as Medium-Medium within the ICAO wake turbulence categories and after re-categorisation the reference separation between most of those aircraft pairs (98\%) remains 3 NM.

Even though the study examines the constraints of the airfield in connection with RECAT, it only considers the data for arriving aircraft. The analysis becomes more complex and involved when arrival-departure sequencing is considered. Estimating the constraints on airfield capacity for each of the runways with regard to arrivals and departures is a subject for future work. Creating a simulation of the arrival-departure sequencing and assessing the effect of wind and weather conditions on runway occupancy times can also be considered a continuation of this study.

% \lipsum[35-41]


% \lipsum[42-43]
% \lipsum[44-50]
%%% Local Variables: 
%%% mode: latex
%%% TeX-master: "DEGREE-NAME-YEAR"
%%% End: 

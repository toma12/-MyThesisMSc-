


\chapter{Discussion}\label{cha:discussion}

\section{Summary}\label{sec:summary}
This study aims to validate the constraints of implementing the RECAT-EU scheme for Keflavík International Airport. The focus is on arrival aircraft pairs in peak hours and the observed time period is thirteen months (from 04 October 2017 to 30 November 2018). 
Airfield capacity is influenced by two major factors: runway occupancy and wake turbulence separation. Currently the wake turbulence categories in use at BIKF and the arrival runway occupancy times show that the AROT is a limiting factor for increased airfield throughput. Different variables affect the AROT, such as the existence and usage of fast-exit runways, weather conditions and the traffic mix. Two fast exit taxiways became operational in 2017 (TWY~A-1 and TWY~B-1). The usage of TWY~B-1 has managed to decrease the AROT by more than 30 seconds on average for RWY~28 during the winter months. The seasonal variations of the AROT were also examined and showed a difference of almost ten seconds on average. The aspects affected by the weather are: the condition of the runway surface that determines braking action, the strength and direction of winds, visibility. One explanation of the preferred usage of RWY~19 over the other runways could be the frequent winds in northern direction (\ref{fig:BIKF_wind_rose}), facilitating decreased AROT for the runway. The traffic fleet at BIKF was primarily from the CAT-C and CAT-D wake categories that combined represent 96\% of the arrivals and were so far the aircraft with the smallest AROT on average. Nevertheless the above conditions were insufficient to produce AROT $\leq 50$ seconds, which is required for reducing the MRS for aircraft on final approach to the 2,5 NM minimum. The average arrival runway occupancy time for BIKF was 77,5 secongs. The runway occupancy in turn has effect on the landing time intervals, which is also governed by the wake turbulence requirements.
The analysis of the re-categorisation requirements for BIKF 







\section{Conclusion\label{sec:conclusions}}



% \lipsum[35-41]


% \lipsum[42-43]
% \lipsum[44-50]
%%% Local Variables: 
%%% mode: latex
%%% TeX-master: "DEGREE-NAME-YEAR"
%%% End: 

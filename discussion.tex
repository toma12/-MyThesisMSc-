


\chapter{Discussion}\label{cha:discussion}

\section{Summary}\label{sec:summary}
This study aims to validate the constraints of implementing the RECAT-EU wake turbulense separation scheme for Keflavík International Airport (BIKF). The focus is on arrival aircraft pairs in peak hours and the observed time period is thirteen months (from 04 October 2017 to 30 November 2018). 
Airfield capacity is influenced by two major factors: runway occupancy and wake turbulence separation. Already, the analysis of those metrics has been performed by Isavia within the current ICAO wake turbulence scheme. Wake turbulence categories in use at BIKF and the arrival runway occupancy times (AROT) show that in most of the cases the AROT is a limiting factor for increase of the airfield throughput.  \\
Different variables affect the AROT, such as the existence and usage of fast-exit runways, weather conditions and the traffic mix. Two fast exit taxiways became operational in 2017 (taxiway TWY~A-1 and TWY~B-1). The usage of TWY~B-1 has managed to decrease the AROT by more than 30 seconds on average for RWY~28 during the winter months. The seasonal variations of the AROT were also examined and showed a difference of almost 10 seconds on average. The weather conditions affect: the runway surface condition that determines braking action, the strength and direction of winds, visibility. One explanation of the preferred usage of RWY~19 over the other runways could be the frequent winds in northern direction (\ref{fig:BIKF_wind_rose}), facilitating decreased AROT for the runway. The traffic fleet at BIKF was primarily from the CAT-C and CAT-D wake categories that combined represent 96\% of the arrivals and were so far the aircraft with the smallest AROT on average. Nevertheless the above conditions were insufficient to produce AROT $\leq 50$ seconds, which is required for reducing the radar separation (MRS) for aircraft on final approach to the 2,5 NM minimum. The average arrival runway occupancy time for BIKF was 77,5 seconds. \\
The runway occupancy in turn has effect on the landing time intervals, which is also governed by the wake turbulence requirements.
The analysis of the re-categorisation requirements for BIKF show that for certain pairs the wake separation could be reduced by 1 - 2 NM, while for a few others the separation would increase by 2 - 3 NM. Reduced wake separation will experience most of the Heavy-Heavy and the Heavy-Medium pairs. Separation will increase for in Medium-Medium pairs that are re-categorised as C-F and D-F pairs, in which cases the Medium follower aircraft is classified as CAT-F, but the number of those cases for Keflavík Airport is relatively small.\\
The majority of the fleet mix though will experience no change of wake separation with the implementation of the RECAT scheme. The separation requirement for the most of the ICAO Medium-Medium pairs remains 3 NM after re-categorisation and the same is true for Medium-Heavy pairs. The combined share of the aircraft that would experience no change to required wake separation was around 84\%. The wake distance separation was transferred into time separation and defined as the landing time interval (LTI), in order to compare it to arrival runway occupancy time.\\
The correlation of arrival runway occupancy and landing time intervals for the predominant C-C pairs revealed that even though a decrease of the LTI is apparently possible, the AROT is a limiting element for increased runway throughput. Not only do the frequency distributions or the two metrics overlap to some extent but also the reference time separation is well within the AROT range. One way the conflict between the AROT and LTI can manifest itself is through a missed approach (ICAO:bulked landing), when an aircraft landing cannot be completed due to runway incursion \cite{doc44444}: the incorrect presence of another aircraft on the runway designated for landing.

\section{Conclusion\label{sec:conclusions}}
The analysis of the  arrival runway occupancy time and the wake vortex separations, measured from the ADS-B data at Keflavík International Airport, show that  


% \lipsum[35-41]


% \lipsum[42-43]
% \lipsum[44-50]
%%% Local Variables: 
%%% mode: latex
%%% TeX-master: "DEGREE-NAME-YEAR"
%%% End: 

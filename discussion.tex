


\chapter{Summary-WORK IN PROGRESS}\label{cha:summary}

Airfield capacity is influenced by two major factors: runway occupancy and wake turbulence separation. Already, the analysis of those metrics has been performed by Isavia within the current ICAO wake turbulence scheme. Wake turbulence categories in use at BIKF and the arrival runway occupancy times show that in most of the cases the AROT is a limiting factor for increase of the airfield throughput.  

Different variables affect the AROT, such as the existence and usage of rapid-exit taxiways, weather conditions and the traffic mix. Two rapid-exit taxiways became operational in 2017 (taxiway TWY~A-1 and TWY~B-1). The usage of TWY~B-1 on runway RWY~28 has managed to decrease the AROT by more than 33\% (37 seconds) on average during the winter and 31\% (31 seconds) during the summer months. The seasonal variations of the AROT were also examined and showed a difference of almost 10\% on average. The weather conditions determine runway surface conditions that affect braking action, the strength and direction of winds and visibility. 

The traffic fleet at BIKF was primarily from the CAT-C and CAT-D wake categories that combined represent 96\% of the arrivals and were so far the aircraft with the smallest AROT on average. Nevertheless the above conditions were insufficient to produce AROT~$\leq~50$ seconds, which is one requirement for reducing the radar separation minima (MRS) for aircraft on final approach to the 2,5 NM minimum. The average arrival runway occupancy time for BIKF was 78 seconds. 

The runway occupancy in turn has effect on the landing time intervals, which is also governed by the wake turbulence requirements.
The analysis of the re-categorisation requirements for BIKF show that for certain pairs the wake turbulence separation could be reduced by~1--2~NM, while for a few others the separation would increase by~2--3~NM. Reduced wake turbulence separation will affect most of the Heavy-Heavy and the Heavy-Medium pairs. Separation will increase for Medium-Medium pairs that are re-categorised as C-F and D-F pairs, but the number of those cases for Keflavík Airport is relatively small.



\chapter{Conclusion-WORK IN PROGRESS\label{cha:conclusions}}

This study uses measured data from the ADS-B surveillance system at Keflavík International Airport~(BIKF) to estimate the arrival runway occupancy time~(AROT) and the wake turbulence separation and aims to validate the constraints of implementing the RECAT-EU wake  separation scheme for BIKF. The focus is on arrival aircraft pairs during peak hours and the observed time period is fourteen months (from 4 October 2017 to 30 November 2018). 

The majority of the fleet mix though will experience no change of wake turbulence separation with the implementation of the RECAT-EU scheme. The separation requirement for the most of the ICAO Medium-Medium pairs remains 3~NM after re-categorisation and the same is true for Medium-Heavy pairs. The combined share of the aircraft that will experience no change to required wake turbulence separation is 83,5\%. 

The wake distance separation was transferred into time separation and defined as the landing time interval~(LTI), in order to compare it to arrival runway occupancy time.
The correlation of arrival runway occupancy and landing time intervals for the predominant C-C pairs revealed that even though a decrease of the LTI is apparently possible, the AROT is a limiting factor for increased runway throughput. Not only do the landing frequency distributions of the two metrics overlap to some extent but also the reference time line is well within the AROT range. One way the conflict between the AROT and LTI can manifest itself is through a missed approach due to the incorrect presence of another aircraft on the runway designated for landing.

% \chapter{Conclusion-WORK IN PROGRESS\label{cha:conclusions}}
The analysis of landing aircraft at Keflavík International, when the airport operates at high loads, shows that rapid-exits on runways have a strong influence of arrival runway occupancy times. Optimised position of rapid-exits in context with the traffic fleet can reduce AROT by 1/3.% The reduction can be greater still if the weather conditions are favourable, even though its effect of is intermittent and not decisive for AROT.

Under the RECAT-EU scheme, the wake turbulence separation will remain unchanged for the majority of the aircraft pairs arriving at BIKF during peak hours. A predominant part of the traffic mix (69,9\%) is currently classified as Medium-Medium within the ICAO wake turbulence categories and after re-categorisation the reference separation between most of those aircraft pairs (98\%) remains 3 NM. One observation from the frequency distribution of the landing time intervals is the wide range over which the times are spread. If the frequency distribution is compressed, this will create opportunity for overall decrease of the inter-arrival time intervals by shifting the distribution closer to the required minimum separation. This possibility can potentially accommodate more landings per time interval and raise runway capacity.

Arrival runway occupancy times and wake turbulence separation requirements interact to formulate the capacity of an airfield. In the case with Keflavík Airport, high arrival runway occupancy times are an obstacle for increased runway capacity, despite the prospect of decreased wake turbulence separation within the RECAT-EU scheme. The observed arrival runway occupancy time is a limiting factor for increase of the runway throughput for the selected aircraft pairs from the current fleet mix at Keflavík Airport.

Although the study examines the constraints of the airfield in connection with RECAT, it only considers the data for arriving aircraft. The analysis becomes more complex and involved when arrival-departure sequencing is considered. Estimating the constraints on airfield capacity for each of the runways with regard to arrivals and departures is a subject for future work. Creating a simulation of the arrival-departure sequencing and assessing the effect of wind and weather conditions on runway occupancy times can also be considered a continuation of this work.

% \lipsum[35-41]


% \lipsum[42-43]
% \lipsum[44-50]
%%% Local Variables: 
%%% mode: latex
%%% TeX-master: "DEGREE-NAME-YEAR"
%%% End: 

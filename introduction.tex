\chapter{Introduction\label{cha:introduction}}
%% \ifdraft only shows the text in the first argument if you are in draft mode.
%% These directions will disappear in other modes.
% \ifdraft{State the objectives of the exercise. Ask yourself:
%   \underline{Why} did I design/create the item? What did I aim to
%   achieve? What is the problem I am trying to solve?  How is my
%   solution interesting or novel?}{}

Keflavík International Airport (IATA:KEF, ICAO:BIKF) is the major gateway to Iceland currently with two functioning runways. The runways operate in both directions and are designated as RWY-01 (South-North), RWY-19 (North-South) , RWY-10 (West-East) and RWY-28 (East-West). 
There has been a gradual increase of passengers travelling through Keflavík Airport in the recent years with over eight million passengers and 62,3 thousand air transport movements in 2017~\cite{isavia_facts_2017}. For 2018 the passenger numbers show a growth of 13.1\% on average (at the time of writing) from the previous year~\cite{isavia_pass_statistics_2018}. The increase in traffic at Keflavík Airport can encounter constrained capacity of the runways during peak traffic periods. 
A significant constraint of airport capacity is caused by separation regulations of aircraft due to wake-vortices. These are specified by rules established by the European Aviation Safety Agency (EASA) for each type of aircraft based primarily on the weight of the aircraft. The general flight safety requirements are established by the International Civil Aviation Organisation (ICAO) for maintaining safe distance between aircraft.\fxnote{Introduction needs more work, incomplete.}

\section{Background and Literature Review}
The research parameters for the literate review included but were not limited to the following concepts: runway occupancy time, runway capacity, wake vortex separation. The project addresses in particular the arrival runway occupancy and its seasonal fluctuations as well as the variations between the different runways at Keflavík International Airport.
%\ifdraft{Provide background about the subject matter (e.g. %How was morse code developed?  How is it used today?). 
%This is a place where there are usually many citations.
%It is suspicious when there is not.
%Include the purpose of the different equipment and your %design intent. 
%Include references to relevant scientific/technical work and %books.
%What other examples of similar designs exist?
%How is your approach distinctive?

%If you have specifications or related standards, these %mustscribed and cited also.  As an example, you might cite %the specific
%RoboSub competition website (and documents) if working on the %lighting system for an AUV

%%% Glossary is broken, do not use --foley
%% \gls{auv}\footnote{Autonomous Undersea Vehicle}.
%
%% Notice that there is now information on the AUV in the %Index %and Acronyms.
%% It isn't in the \gls{glossary} because we didn't put it %there.
%\index{AUV}
%}{}



\subsection{Current Understandings of Wake Vortex Behaviour}

The lift effect on wing is created by the differential pressure between the lower and the upper surface of the wing. At the wing tips the high pressure flow from the lower side leaks around the tip and to the upper side of the wing. Thus the streamlines over the wing are pushed inwards, while the streamlines under the wing are pushed outwards. When the two streams combine at the trailing edge of a lifting wing, the difference in span-wise velocity causes the air to roll up into a number of small stream-wise vortices distributed along the span of the wing that eventually mix together and combine into two main counter-rotating vortices aft of the wing as illustrated in Figure~\ref{fig:vortex_develop} ~\cite{houghton2012aerodynamics,magazine_aibus_safety, Breitsamter2011Feb, gerz_commercial_2002}. Between the vortices the induced flow is downwards (downwash) while outside the air moves upwards. Those vortices are also known as wake turbulence. 
The kinetic energy contained in the vortices is dependent on the weight and aerodynamics of the generating aircraft. The cross flow velocities in the core region of the trailing vortices can reach $360$ km/h and the vortices can stay effective up to hundred wing spans, which can result in wake vortices lasting for several minutes and up to $30$ km behind larger aircraft~\cite{Breitsamter2011Feb, gerz_commercial_2002}. 

\begin{figure}[ht]
    \centering
    \includegraphics[width=0.8\textwidth]{graphics/WakeVortexPlane.png}
    \caption[Wake vortex roll-up process]{Wing vortex evolution and roll-up process. Two main vortices form behind the aircraft turning in opposite directions, clockwise behind the left wing (seen from behind) and anti-clockwise behind the right one~\cite[p. 043]{magazine_aibus_safety}} \label{fig:vortex_develop}
\end{figure}

\subsection{Ground Effect and Decay Process}

Near ground on approach the wake vortices tend to descend slowly to an altitude of about one half of the initial separation.
Upon reaching this point the descent will stop and then ascend slowly. This "re-bounce" effect is caused by the presence of the ground and carries with it the formation of a second pair of induced vortices outside and bellow the main vortex. 
In the presence of stable cross-wind conditions the decay of the downwind vortex is not identical to the upwind vortex and the descent rate may vary significantly \cite{Hallock2018Apr}. 
The trajectories of the vortex pair may be modified and the upwind vortex could stall over the runway while the downwind vortex ascends and decays faster (Figure~\ref{fig:vortex_ground_effect}).
Measurements have shown that the lateral/sideways motion of a vortex in an airport environment could range between $229$~m and $518$~m and in some cases even $762$~m, which is the basis for the "2500-foot"  rule separation in parallel runways configuration~\cite{Hallock2018Apr, hallock2004summary, hallock2003wake}. The vertical descent rate of vortices for medium commercial aircraft in stable atmosphere  is shown to be around $1.5-2.5$~m/s for the first $30$~s after which the descent slows and eventually approaches zero at $152-274$ m below the flight path, and for heavier aircraft decaying vortices have been observed at $305$~m and below of the flight path~\cite{lissaman1973aircraft, Hallock2018Apr}. 
Aerodynamic properties of the aircraft govern the vortex roll-up process and ambient atmospheric conditions dominate the behaviour of the vortices forcing instability and eventual decay of the vortices~\cite{Hallock2018Apr}.
Several factors that drive the decay rate of the vortex pair have been formulated:
\begin{itemize}
    \item Atmospheric turbulence extracts energy from the vortex and reduces its strength leading to a faster wake decay ~\cite{Hallock2018Apr}.
    \item Viscosity of the atmosphere also draws out energy from the vortex but at a slower rate than the atmospheric turbulence. The so called dissipative action of viscosity  effectively removes energy from a disturbance, in this case a vortex, thereby causing it to decay~\cite{Hallock2018Apr, houghton2012aerodynamics}.
    \item Buoyancy force acts on the vortex in a thermally stable stratified environment as a result of the lesser air density inside the vortex system and may causes stall or rebound to the flight level~\cite{Holzapfel2001Feb, gerz_commercial_2002}.
    \item Vortex instability in the form of long wave sinusoidal fluctuations of the vortex core (Crow instability) may occur due to light turbulence in the atmosphere and may cause the vortices to link and decay faster~\cite{Hallock2018Apr, crow2003stability}.
    \item Secondary vorticity structures, vertical rib-like counter rotating flow formations may appear between the vortex pair and eventually wrap around the main vortices, leading to turbulence build inside the system and rapid circulation decay~\cite{Holzapfel2001Feb, Holzapfel2003Jun}.
\end{itemize}


\begin{figure}[h]
    \centering
    \includegraphics[width=0.8\textwidth]{graphics/Hallock_vortex_evolution.jpg}
    \caption[Wake vortex and ground effect]{Vortex evolution and ground effect~\cite[p. 29]{Hallock2018Apr}.} \label{fig:vortex_ground_effect}
\end{figure}


\subsection{Wake Vortex Separation}
An aircraft affected by the wake vortex can experience loss of lift or an induced rolling moment and velocity fluctuations (Figure~\ref{fig:vortex_encounter}).
This can lead to the up-set and potentially loss of control of an aircraft following the flight path of a preceding aircraft.
To diminish the effect of such vortices the trailing aircraft must be maintained at a safe distance behind the leading aircraft as the vortices spread laterally to either side of the flight path and are dissipated~\cite{Breitsamter2011Feb}.
The distance between aircraft pairs in view of safety is referred to as wake vortex separation. 

\begin{figure}[h]
    \centering
    \includegraphics[width=0.8\textwidth]{graphics/reaction_in_wake.jpg}
    \caption[Wake vortex encounter]{Effect of wake vortex flow field on an aircraft. (A) induced rolling moment, (B) upward motion, (C) loss of lift~\cite[p. 33]{Hallock2018Apr}} \label{fig:vortex_encounter}
\end{figure}

Traditional separation standards, introduced in the seventies, are defined by ICAO  on the basis of certificated maximum take-off mass (MTOM) as shown in Table~\ref{tab:WTC}. The prescribed categories are three i.e. Heavy, Medium and Light. In addition for aircraft in the order of $560000$~kg ICAO provides a subcategory to the Heavy types called Super Heavy. Some states have chosen to further increase the number of categories to adapt the wake turbulence categories (WTC) to specific airport requirements. Aerodromes in the UK split the Medium and Light categories each into two subcategories (Table~\ref{tab:WTC}), while the Federal Aviation Administration (FAA) places Boeing B757 into a separate category and increases the upper weight limit of the Light category.~\cite{icao_wtc, uk_aeronautical_information_services_wake_2017, noauthor_recat_2018}

\begin{table}[h]
    \centering
    \resizebox{1\textwidth}{!} {
    \begin{tabular}{l|c|c|c|c|c|c}
    ~    & \multicolumn{6}{c}{Category} \\ \hline
    ICAO & Super Heavy & Heavy (H) & \multicolumn{2}{c|}{Medium (M)} & \multicolumn{2}{c}{Light (L)} \\
    
    ~    & A380-800    & $ > 136000 $  & \multicolumn{2}{c|}{$ 136000-7000 $ } & \multicolumn{2}{c}{$ \leq 7000 $} \\ \hline
    
    ICAO (UK)   & ~  & Heavy (H) & Upper Medium (UM) & Lower Medium (LM) & Small (S)  & Light (L) \\
    
    ~    & A380-800    & $ > 136000 $  & $ 136000-104000 $     & $ 104000-40000 $      & $ 40000-17000 $ & $ \leq 17000 $   \\ \hline
    
    ICAO \& FAA (US)   & Super      & Heavy (H) & B757   & Large   & \multicolumn{2}{c}{Small} \\
    
    ~    & A380        & $ > 136000 $  & ~                 & $ 136000-18600 $      & \multicolumn{2}{c}{$ \leq 18600 $} \\ 
    \end{tabular}}
    \caption[ICAO wake turbulence categories based on maximum take-off mass]{ICAO wake turbulence categories based on maximum take-off mass (MTOM) in kg~\cite{icao_wtc, uk_aeronautical_information_services_wake_2017, kolos2013influence}.} \label{tab:WTC}
\end{table}

The ICAO wake turbulence categories (Table~\ref{tab:ICAO_WTC}) outline the worst case for each category and thus generate over-separation in many cases~\cite{noauthor_recat_2018}. The separation distances for instrumental flight rules are observed for the follower aircraft and are conventionally expressed in nautical miles (NM). The separation reverts to radar separation minimum (MRS) when wake turbulence restrictions are not required. As  prescribed  by  ICAO  the  horizontal MRS is 3~NM (or a reduced separation minimum of 2.5~NM may be applied under given conditions described in ICAO Doc 4444 PANS-AM~\cite{doc44444}), which translates to 79~s (66~s) time separation between leader and follower at final approach speed of 70~m/s.


% Please add the following required packages to your document preamble:
% \usepackage{multirow}
% \usepackage{graphicx}
\begin{table}[h]
\centering
\resizebox{\textwidth}{!}{%
\begin{tabular}{|c|c|c|c|c|c|}
\hline
\multicolumn{2}{|c|}{\multirow{2}{*}{ICAO WTC scheme}} & \multicolumn{4}{c|}{Follower}              \\ \cline{3-6} 
\multicolumn{2}{|c|}{}                                 & Super (A380-800) & Heavy & Medium & Light \\ \hline
\multirow{4}{*}{\rotatebox[origin=c]{90}{Leader}}       & Super (A380-800)       & (*)                & 6 NM  & 7 NM   & 8 NM   \\ \cline{2-6} 
                              & Heavy                  & (*)                & 4 NM  & 5 NM   & 6 NM   \\ \cline{2-6} 
                              & Medium                 & (*)                & (*)     & (*)      & 5 NM   \\ \cline{2-6} 
                              & Light                  & (*)                & (*)     & (*)      & (*)      \\ \hline
\end{tabular}%
}
\caption[ICAO wake turbulence categories and separation minima]{ICAO wake turbulence categories and separation minima to avoid wake vortex encounter~\cite{noauthor_recat_2018, rooseleer2015recat}.} \label{tab:ICAO_WTC}
\end{table}

It must be noted that the division of ICAO WTC into the RECAT scheme (Table~\ref{tab:RECAT-dist}) is not achieved by splitting each category in half.
The criteria used for categorisation of existing and new aircraft types are provided in detail in~\cite{rooseleer2015recat}. The case from the traffic mix at BIKF presents the arrangement shown in Table~\ref{tab:wtc2recat_division}. A more detailed illustration of the re-categorisation is presented in the upcoming section \ref{ssec:traffic_mix} on aircraft traffic mix (Figure~\ref{fig:post_fast_exit_mix_pie_v2}).

% Please add the following required packages to your document preamble:
% \usepackage{graphicx}
% \usepackage[table,xcdraw]{xcolor}
% If you use beamer only pass "xcolor=table" option, i.e. \documentclass[xcolor=table]{beamer}
\begin{table}[h]
\centering
\resizebox{0.7\textwidth}{!}{%
\begin{tabular}{clcllc}
\multicolumn{3}{c}{\cellcolor[HTML]{34CDF9}HEAVY} &  &  & \cellcolor[HTML]{FD6864}LIGHT \\ \hline
\multicolumn{1}{|c|}{\cellcolor[HTML]{34CDF9}CAT-A} & \multicolumn{1}{c|}{\cellcolor[HTML]{34CDF9}CAT-B} & \multicolumn{1}{c|}{\cellcolor[HTML]{32CB00}CAT-C} & \multicolumn{1}{c|}{\cellcolor[HTML]{F8FF00}CAT-D} & \multicolumn{1}{c|}{\cellcolor[HTML]{F8FF00}CAT-E} & \multicolumn{1}{c|}{\cellcolor[HTML]{FFC702}CAT-F} \\ \hline
\multicolumn{1}{l}{} &  & \multicolumn{4}{c}{\cellcolor[HTML]{F8FF00}MEDIUM}
\end{tabular}%
}
\caption[Transition from ICAO WTC to RECAT-EU categories]{Transition from ICAO WTC to RECAT-EU categories. The categorisation process and criteria for assigning an existing aircraft type into RECAT-EU scheme is illustrated in detail in~\cite{rooseleer2015recat}}.
\label{tab:wtc2recat_division}
\end{table}

The noticeable changes in Table~\ref{tab:RECAT-dist} from Table~\ref{tab:ICAO_WTC} affect a follower aircraft, whose relative position from the ICAO scheme is shifted to the left in the new table after re-categorisation. Such benefit experiences for example a Heavy follower aircraft that is re-categorised as CAT-C, in which case the separation from the leader is reduced with up to 3 NM. 

%There is an RU logo in Figure~\ref{fig:ru-logo}.
%This logo will scale according to the width of the text on the page.

% Please add the following required packages to your document preamble:
% \usepackage{multirow}
% \usepackage{graphicx}
\begin{table}[h]
\centering
\resizebox{\textwidth}{!}{%
\begin{tabular}{|c|c|c|c|c|c|c|c|}
\hline
\multicolumn{2}{|c|}{\multirow{2}{*}{RECAT-EU scheme}} & \multicolumn{6}{c|}{Follower}                   \\ \cline{3-8} 
\multicolumn{2}{|c|}{}                                 & CAT-A & CAT-B & CAT-C & CAT-D & CAT-E & CAT-F \\ \hline
\multirow{6}{*}{\rotatebox[origin=c]{90}{Leader}}            & CAT-A            & 3 NM   & 4 NM  & 5 NM  & 5 NM  & 6 NM  & 8 NM  \\ \cline{2-8} 
                                    & CAT-B            &    (*)    & 3 NM  & 4 NM  & 4 NM  & 5 NM  & 7 NM  \\ \cline{2-8} 
                                    & CAT-C            &    (*)    & (*)   & 3 NM  & 3 NM  & 4 NM  & 6 NM  \\ \cline{2-8} 
                                    & CAT-D            &    (*)    &   (*)    &    (*)   &   (*)    &   (*)    & 5 NM  \\ \cline{2-8} 
                                    & CAT-E            &    (*)    &   (*)    &   (*)    &   (*)    &   (*)    & 4 NM  \\ \cline{2-8} 
                                    & CAT-F            &    (*)    &   (*)    &    (*)   &   (*)    &   (*)    & 3 NM  \\ \hline
\end{tabular}%
}
\caption[RECAT-EU distance-based separation minima]{RECAT-EU wake turbulence distance-based separation minima on approach and departure. (*) means minimum radar separation (MRS), set at 2.5 NM, applicable as per current ICAO doc 4444 provisions \cite{doc44444, rooseleer2015recat, noauthor_recat_2018}.}
\label{tab:RECAT-dist}
\end{table}

The different category code names in this study will also be referred to on a single letter basis for simplification e.g. "CAT-C" will be abbreviated to "C", "CAT-D" to "D", "Heavy" to "H", Medium to "M" and so forth. This applies to pair categories by abbreviating e.g. "CAT-C leader - CAT-D follower"  to "C-D pair" or simply "C-D", and "Medium-Medium" pair to "M-M".

\section{Runway Occupancy Time\label{sec:ROT}}

The Runway Occupancy Time (ROT) is defined by Eurocontrol as the amount of time that each aircraft occupies the runway~\cite{ROT_definition}. \\
This project focuses on Arrivals Runway Occupancy Time (AROT) in peak traffic hours, that is the time interval between the aircraft crossing the threshold and its tail vacating the runway~\cite{AROT_definition}. The threshold is the beginning of that portion of the runway that is available for landing.
The peak traffic hours are characterised by the time intervals when the airport operates at high load. High load interval or window is at least 15 minutes in length and has four or more flights arriving or departing from Keflavík Airport. The time between flights during the high load interval is set at $\leq$4~minutes (Figure~\ref{fig:Peak_Diagram}). This value is based on statistical data by Isavia in order to achieve two distinct peak hours: one in the morning and one in the afternoon. The last flight in a high load window that fulfils these requirements is not considered to be a part of the peak as its behaviour is not affected by an instantaneous next flight after it. The AROT metric is essential for the project because a following aircraft is not allowed to land on the same runway before it has been vacated by the leading aircraft. The interval to the next landing aircraft is specified as the Landing Time Interval (LTI) which is directly linked to the inter-arrival distance or in other words the wake turbulence separation.

\begin{figure}[h]
    \centering
    \includegraphics[width=1\textwidth]{graphics/Peak_Diagram.png}
    \caption[Rules defining a peak hour]{Rules used by Isavia to identify a high load interval as peak or not. The time separating each two aircraft in a peak cluster (red) is set as less than or equal to 4 minutes. The last red aircraft in the peak cluster is not counted as part of the peak. Cluster separators are defined as time intervals larger than four minutes.}
    \label{fig:Peak_Diagram}
\end{figure}


\section{Runway Capacity\label{sec:runway_capacity}}

ICAO generally defines airport capacity as the number of movements per unit of time that can be accepted during different meteorological conditions~\cite{airport_capacity_methodology}. 
Several measures are currently used to estimate the amount of aircraft movements on the runways of an airport in a specified time interval.
R. de Neufville~\cite{de_neufville_airport_2013} states that the capacity of runway systems determines the ultimate capacity of an airport and that its principal measure maximum throughput capacity. It indicates the average number of movements that can be performed on the runway system in one hour in the presence of continuous demand, while adhering to all the separation requirements imposed by the air traffic management (ATM) system~\cite{de_neufville_airport_2013}.
The Capacity Envelope method used by Isavia to estimate the throughput capacity at BIKF takes into consideration the number of arrivals and departures within a selected time frame. This approach allows to measure the throughput capacity and model the maximum possible values or the potential of the airfield. A summary of the capacity envelope for a period of one year is shown in Figure~\ref{fig:capacity_evnelope} for a 15~minute time interval. 

\begin{figure}[h]
    \centering
    \includegraphics[width=0.8\textwidth]{graphics/fig_Capacity_Envelope_2017-10-04_to_2018-11-30_15min_occurrences_limits.png}
    \caption[Capacity envelope for BIKF]{Capacity envelope at BIKF since October 2017. The measured time interval is 15 minutes and includes both the morning and afternoon peaks. The capacity limits of the airfield are also defined and noted on the figure for various sequence modes. The mixed modes indicate equal arrivals and departures (Mixed~Mode~A) or two times more arrivals than departures (Mixed Mode B), the later being a tendency during the afternoon peaks. The figure is from Isavia´s internal GUI Víkingaskipið.}  \label{fig:capacity_evnelope}
\end{figure}

This approach indicates a maximum measured capacity of four arrivals to four departures in a "Mixed~Mode~A". This equilibrium mode is defined as the primary measure for the throughput capacity of BIKF. The values are in accord with the defined peak hour intervals in the previous section~\ref{sec:ROT}

There are many aspects of a runway system that can affect the number of aircraft that can land or depart from an airfield and some of those are listed bellow~\cite{de_neufville_airport_2013, kim_validation_2010}.
\begin{itemize}
    \item Number and geometric layout of the runways.
    \item State and performance of the ATM system.
    \item Separation requirements between aircraft pairs imposed by the ATM.
    \item Weather conditions (visibility, precipitation, cloud ceiling). 
    \item Wind direction and strength.
    \item Mix of aircraft using the airport.
    \item Sequence of movement on each runway (arrivals - departures split)
    \item Type and location of taxiway exits from the runway.
    \item Runway occupancy time.
    \item Controller workload.
    \item Noise-related constraints and other environmental considerations.
\end{itemize}

Two of the above mentioned factors are principal in determining airfield capacity, namely the wake vortex separation requirements and runway occupancy~\cite{kolos2013influence}. Those are in turn influenced by traffic mix, seasonal factors and runway exits, to mention a few. The following chapter~\ref{cha:methods} on Methods  will deal with some of those aspects. 

\fxnote{equipment specifications in the tower that might influence the WTC separtarion decision-talk to Haraldur, Talk to Hjalti about the 2,5~-~3~NM MRS issue!}

\section{Runway Occupancy and Landing Time Interval Study Objective}\label{sec:study_objective}
The goal of examining the arrival runway occupancy times and the landing time intervals (LTI) is to determine whether the implementation of the RECAT-EU scheme at BIKF under certain conditions, would affect the throughput of the airfield. 
LTI is the a measure of the time separation between aircraft pairs derived from the distance separation between the aircraft and the final approach speed (time travelled is distance travelled divided by velocity).
% The LTI is directly influenced by the distance separation of arrival pairs and the final approach speed of the aircraft .
Additionally the analysis could indicate whether the limiting element is the runway occupancy or the separation requirement. The proposed transition by Eurocontrol from ICAO WTS to RECAT-EU scheme suggests a decrease in distance separation for a number of the aircraft pairs (Table~\ref{tab:delta_distance_wtc2recat}).

% Please add the following required packages to your document preamble:
% \usepackage{multirow}
% \usepackage{graphicx}
\begin{table}[h]
\centering
\resizebox{\textwidth}{!}{%
\begin{tabular}{|c|c|c|c|c|c|c|c|}
\hline
\multicolumn{2}{|c|}{}                                          & \multicolumn{6}{c|}{Follower}                                                                                                                                                                                                   \\ \cline{3-8} 
\multicolumn{2}{|c|}{\multirow{-2}{*}{RECAT-EU scheme}} & CAT-A                             & CAT-B                                & CAT-C                         & CAT-D                         & CAT-E                         & CAT-F                                                \\ \hline
                                                        & CAT-A & \cellcolor[HTML]{FD6864}(+0,5 NM) & \cellcolor[HTML]{67FD9A}-2 NM        & \cellcolor[HTML]{67FD9A}-1 NM & \cellcolor[HTML]{67FD9A}-2 NM & \cellcolor[HTML]{67FD9A}-1 NM &                                                      \\ \cline{2-8} 
                                                        & CAT-B &                                   & \cellcolor[HTML]{67FD9A}-1 NM        &                               & \cellcolor[HTML]{67FD9A}-1 NM &                               & \cellcolor[HTML]{FD6864}+1 NM                        \\ \cline{2-8} 
                                                        & CAT-C &                                   & \cellcolor[HTML]{67FD9A}-1 (-1,5) NM & \cellcolor[HTML]{67FD9A}-1 NM & \cellcolor[HTML]{67FD9A}-2 NM & \cellcolor[HTML]{67FD9A}-1 NM &                                                      \\ \cline{2-8} 
                                                        & CAT-D &                                   &                                      &                               &                               &                               &                                                      \\ \cline{2-8} 
                                                        & CAT-E &                                   &                                      &                               &                               &                               & \cellcolor[HTML]{67FD9A}{\color[HTML]{000000} -1 NM} \\ \cline{2-8} 
\multirow{-6}{*}{\rotatebox[origin=c]{90}{Leader}}                                & CAT-F &                                   &                                      &                               &                               &                               & \cellcolor[HTML]{FD6864}(+0,5 NM)                    \\ \hline 
\end{tabular}%
}
\caption[Difference in wake separation minima between ICAO and RECAT-EU schemes]{Difference in wake separation minima on approach between reference ICAO and RECAT-EU schemes (full proposal)~\cite{rooseleer2015recat}}
\label{tab:delta_distance_wtc2recat}
\end{table}



% % Please add the following required packages to your document preamble:
% % \usepackage{multirow}
% % \usepackage{graphicx}
% \begin{table}[]
% \centering
% \resizebox{\textwidth}{!}{%
% \begin{tabular}{|c|c|c|c|c|c|c|c|}
% \hline
% \multicolumn{2}{|c|}{}                                          & \multicolumn{6}{c|}{Follower}                                                                                                             \\ \cline{3-8} 
% \multicolumn{2}{|c|}{\multirow{-2}{*}{RECAT-EU scheme}} & CAT-A & CAT-B                        & CAT-C & CAT-D                        & CAT-E                        & CAT-F                        \\ \hline
%                                                         & CAT-A &       & \cellcolor[HTML]{67FD9A}-20s &       & \cellcolor[HTML]{67FD9A}-40s & \cellcolor[HTML]{67FD9A}-20s &                              \\ \cline{2-8} 
%                                                         & CAT-B &       &                              &       & \cellcolor[HTML]{67FD9A}-20s &                              & \cellcolor[HTML]{FD6864}+20s \\ \cline{2-8} 
%                                                         & CAT-C &       &                              &       & \cellcolor[HTML]{67FD9A}-40s & \cellcolor[HTML]{67FD9A}-20s &                              \\ \cline{2-8} 
%                                                         & CAT-D &       &                              &       &                              &                              &                              \\ \cline{2-8} 
%                                                         & CAT-E &       &                              &       &                              &                              & \cellcolor[HTML]{67FD9A}-20s \\ \cline{2-8} 
% \multirow{-6}{*}{Leader}                                & CAT-F &       &                              &       &                              &                              & \cellcolor[HTML]{FD6864}+20s \\ \hline
% \end{tabular}
% }
% \caption[Time difference between reference ICAO and RECAT-EU schemes]{Time difference in wake separation on departure between reference ICAO and RECAT-EU schemes(full proposal) \cite{rooseleer2015recat}}
% \label{tab:delta_times_wtc2recat}
% \end{table}

The implementation of RECAT shifts the wake separation minima for arrival and departure pairs thus creating the potential for reduced time separation between the aircraft. The reduction of the time interval between landings facilitates increas in the throughput capacity of the runways. 
Figure~\ref{fig:AROT_LTI_rwy19_H_M} may be used to illustrate the hypothesis. The current ICAO time separation reference, indicated by the red vertical line for BIKF RWY-19 will be relocated to the left for some of the aircraft pairs under the RECAT scheme. This relocation will be more noticeable for RECAT C-C and C-D pairs formed from the ICAO HEAVY-MEDIUM pairs, in which case the separation is reduced from 5 NM to 3 NM. A shift of the reference separation line creates the potential for shift in the distribution of the LTI, as long as it does not overlap with the ROT. Furthermore study on the runway capacity of significantly larger airports~\cite{kolos2013influence} suggests that the frequency distribution of LTI tends to compress or squeeze to the right of the reference line with less standard deviation about the mean when the air traffic is intensified.

\begin{figure}[h]
    \centering
    \includegraphics[width=0.8\textwidth]{graphics/fig_rot_landig_time_interval_RWY19_leader_H_follower_M_peak-hour_BAR_20171004_20181130.png}
    \caption[Caption table]{Arrival runway occupancy time (blue) and landing time intervals (green) for Heavy leader - Medium follower pairs on BIKF runway RWY-19. The observed time period is from October 2017 to November 2018. The red line indicates the ICAO WTC reference time separation for the selected H-M aircraft pairs. The figure is from the Isavia´s internal GUI Víkingaskipið.}\label{fig:AROT_LTI_rwy19_H_M}
\end{figure}




%%% Local Variables: 
%%% mode: latex
%%% TeX-master: "DEGREE-NAME-YEAR"
%%% End: 

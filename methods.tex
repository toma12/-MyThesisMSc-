

\chapter{Methods\label{cha:methods}}
Data collection on flights through the airspace monitored and controlled by Isavia has been going on for several years now. This data is filtered and stored on company servers for reference and analysis. The statistical analysis for this project was done in Python and used to determine the fleet mix at BIKF, runway occupancy and landing intervals.

\section{Data Collecting and Data Filtering}
More detailed data collection commences on 21 November 2014 with the implementation of the Automatic Dependent Surveillance-Broadcast (ADS-B) system by Isavia. ADS-B was preferred to radar at Keflavik Internetional Airport because of improved signal accuracy~\cite{isavia_wiki}. The ADS-B equipped aircraft through Keflavik Airport are estimated to be around~$90\%$~\cite{isavia-rounardeild_rannsoknir_2018}. The aircraft surveillance data specifications used for measurements are in the Eurocontrol ASTERIX Category 21 standard~\cite{ASTERIX_ADS-B_specs}.
The reading from the position of the ADS-B antenna of the aircraft is used instead of the tail and nose positions. This simplification could result in a few seconds difference. Certhttps://www.overleaf.com/read/whpxcswcrgytain filtering of the ADS-B data is applied before commencing measurements and calculations~\cite{isavia_wiki}: 
\begin{itemize}
    \item All records with no latitude, longitude and/or time data are omitted.
    \item The records are filtered with regards to quality, i.e. filtered with regards to Target Surveillance status, MOPS version, NUC/NIC, NACp, SIL and PIC.
    \item An ADS-B data record within 0.5 second from another is omitted, i.e. if there are less than 0.5 seconds between two successive locations one of the records is omitted.
    \item An algorithm is used to analyse the data and determine if it originates from two different flights (the flight stops for 15 minutes or more after/before taxiing). If so the data for the second flight is omitted
    \item The velocity is calculated from ADS-B location data and ADS-B time data (velocity = distance travelled/time). The ADS-B velocity record is optional in the ASTERIX category 21 standard and is therefore unreliable in measurements. Since the velocity is calculated from measured data, it can exhibit spikes. To get a more realistic velocity curve, a Savitzky-Golay filter is applied to smooth the data.
    \item An aircraft main gear lift-off is considered to be the point where the ADS-B ground bit is removed, i.e. the ADS-B ground bit goes from a value of 1 to a value of 0.
    \item An aircraft touchdown is considered to be the point where the ADS-B ground bit is set, i.e. the ADS-B ground bit goes from a value of 0 to a value of 1.
    \item An aircraft is considered to be stationary when the velocity is below 0.5~knots and to be moving if the velocity is greater than 0.5~knots.
    
\end{itemize}

\fxnote{The ROT measurements have been checked by: comparing them to measurements made by hand.
Who measures and who checks them and how???}

\section{Data Manipulation}
The data was obtained from the database using SQL server and stored as csv files. The manipulations on the data frames were performed using the functionality of Pandas and NumPy data analysis tools and computing packages for Python. A key table containing information about 2300 aircraft models, including ICAO and RECAT-EU wake turbulence categories for each model, was also used as reference. 
The reference table was provided by Eurocontrol. 
The Isavia aircraft data was cross-referenced with the Eurocontrol key table based on ICAO aircraft type labels, the mutual characteristic in both data sets. 
After cross-referencing a RECAT-EU category was assigned to each of the aircraft arriving at BIKF in peak hours. Outliers bellow the 0,003~quantile and above the 0,997~quantile were regarded as anomalies removed from data set based on AROT values. 
The resulting data frame was used for the analysis in this project. 
It contained over 11500 arrivals for the time period of almost four years (from 01.01.2015 until 30.11.2018). The data frame contained unique information about the AROT of each aircraft, landing time and runway, along with the ICAO wake category. Consequently the time frame for the analysis was reduced to 13 months (from 04.10.2017 to 30.11 2018) because of the effect of fast-exits on AROT, which is explained in the coming section \ref{sssec:factors_arot}


\subsection{Aircraft Traffic Mix\label{ssec:traffic_mix}}
The aircraft traffic mix is one of the components that affect runway capacity as mentioned in Section~\ref{sec:runway_capacity}. Sorting the aircraft fleet arriving at BIKF into ICAO WTC reveals that the majority of flights~(85,4\%) are in the Medium wake category (Figure~\ref{fig:post_fast_exit_mix_pie_v2}) and the rest are mainly in the Heavy category~(14,2\%) with less than one percent Light aircraft. The small portion of Light flights can be explained with the policy of ATM at BIKF to avoid servicing those flights during the morning and afternoon peaks. This distribution of the three categories implies that combining the aircraft into arriving pairs would produce primarily Medium-Medium pairs. This is confirmed by the analysis of the number of arrival pairs in the ICAO categories (Table~\ref{tab:mix_to_wtc}).\\
When this same fleet is presented using the RECAT categories, the prevailing category is CAT-C (73,7\%), followed by CAT-D (22,1\%). The percentage of each of the remaining categories varies within 2\%. The expectation that this distribution of the RECAT categories would result in primarily in C-C pairs is confirmed later on by the analysis presented in Table~\ref{tab:mix_to_recat}. \\
The method of of re-categorising the traffic mix is a prerequisite for assembling the RECAT pairs later on and determining the main aircraft categories that will be considered for analysis along with estimating the inter arrival distances characteristic of each pair.
\begin{figure}[h]
    \centering
    \includegraphics[width=1\textwidth]{graphics/fig_post_fast_exit_mix_pie_v2.png}
    \caption[Traffic mix in RECAT-EU and ICAO WTC]{The traffic mix at Keflavik Airport represented in RECAT-EU categories alongside ICAO WTC categories.}
    \label{fig:post_fast_exit_mix_pie_v2}
\end{figure}
Another approach for classification of the traffic fleet mix at BIKF in peak hours is to identify the aircraft types or models. The traffic data contains ICAO aircraft type designator, that is a two-, three- or four-character code comprising of numbers and letters. This designator is unique for each aircraft type. The top fifteen ICAO types of the aircraft mix are presented in Figure~\ref{fig:traffic_mix_by_model}. Dominating the scene (56.8\% of all arrivals) is the Boeing~757-200 model (ICAO:B752), followed by the Boeing~767-300 (ICAO:B763).
\begin{figure}[h]
    \centering
    \includegraphics[width=1\textwidth]{graphics/fig_traffic_mix_by_model.png}
    \caption[Traffic mix by aircraft model.]{The traffic mix at Keflavik Airport grouped by aircraft type are shown alongside the ICAO WTC and RECAT-EU designators. The top three models comprise the major part of the Icelandair fleet.}
    \label{fig:traffic_mix_by_model}
\end{figure}
 Those models represent a major part of the Icelandair fleet and the share of the Boeing~737~MAX~8 (ICAO:B38M) is likely to increase as the Icelandair company plans to gradually add sixteen new B38M and B39M models from the beginning of 2018~\cite{icelandair_fleet}. The aircraft types are shown with their ICAO and RECAT categories and the tendency is towards increasing the share of the Medium-Medium pairs, or the C-D and D-C RECAT pairs respectively, with the addition of the new Icelandair aircraft.




% -----------------------


\subsection{Arrival Runway Occupancy Times} 

The ICAO Doc 4444 PANS-AM~\cite{doc44444} dictates that the radar separation minimum (MRS) between succeeding aircraft which are established on the same final approach track, may be reduced from 3~NM to 2,5~NM under certain conditions. One of the requirements is that the average runway occupancy time of landing aircraft is proven, by means such as data collection and statistical analysis and methods based on theoretical model, not to exceed 50 seconds. \\
Several approaches were used to look at the AROT at the airfield. First the runway occupancy was inspected based on the RECAT category of the aircraft (Figure~\ref{fig:RECAT_AROTs_boxplot}). 
\begin{figure}[h]
    \centering
    \includegraphics[width=1\textwidth]{graphics/fig_RECAT_AROTs_boxplot.png}
    \caption[AROTs boxplot for RECAT categories, all runways]{Arrival Runway Occupancy Times for the different RECAT-EU categories based on data gathered for a period of one year since October 2017. The box plot shows the AROTs for all runways at BIKF.}
    \label{fig:RECAT_AROTs_boxplot}
\end{figure}
The analysis showed that none of the average runway occupancy values fulfils the 50~seconds limit for reduced MRS. Closest to the required time were the aircraft from the CAT-D and CAT-C with mean values of 75 and 78 seconds respectively (Table~\ref{tab:AROT_RECAT_stats}). Those results point to the necessity of setting the MRS reference value at 3 NM and also suggest that the runway occupancy will be a limiting factor for the cases in which MRS is applicable (refer to Table~\ref{tab:RECAT-dist}).


\subsubsection{Factors affecting AROT\label{sssec:factors_arot}}
\fxnote{Airfield surface condition, weather conditions, fast exits, season. The effect of Specific weather conditions on ROT were not investigated in this project, Airfield surface conditions ref from Doc 4444.}







% Please add the following required packages to your document preamble:
% \usepackage{graphicx}
\begin{table}[h]
\centering
\resizebox{0.8\textwidth}{!} & \multicolumn{1}{l|}{50\%} & \multicolumn{1}{l|}{75\%} & \multicolumn{1}{l|}{max} \\ \hline
\multicolumn{1}{|l|}{SUMMER} & \multicolumn{1}{r|}{3727} & 76  & 17 & 46 & 63  & 72 & 87  & 155 \\ \hline
\multicolumn{1}{|l|}{WINTER} & \multicolumn{1}{r|}{937}  & 84  & 20 & 45 & 69  & 84 & 96  & 153 \\ \hline
\end{tabular}%
}
\caption[AROTs for the air traffic mix by season]{AROT statistics for the air traffic mix at KEF by season. The count is the number of landings in peak hours since October 2017}
\label{my-label2}
\end{table}

% Please add the following required packages to your document preamble:
% \usepackage{graphicx}
\begin{table}[h]
\centering
\resizebox{0.8\textwidth}{!} & \multicolumn{1}{l|}{50\%} & \multicolumn{1}{l|}{75\%} & \multicolumn{1}{l|}{max} \\ \hline
\multicolumn{1}{|l|}{RWY 01} & 963 & 85 & 21 & 46 & 68 & 88 & 100 & 152 \\ \hline
\multicolumn{1}{|l|}{RWY 10} & 852 & 85 & 11 & 61 & 78 & 84 & 91 & 155 \\ \hline
\multicolumn{1}{|l|}{RWY 19} & 1511 & 67 & 10 & 46 & 61 & 66 & 71 & 155 \\ \hline
\multicolumn{1}{|l|}{RWY 28} & 401 & 68 & 14 & 49 & 60 & 66 & 72 & 155 \\ \hline
\end{tabular}%
}
\caption[AROTs for the air traffic mix by runway for the summer]{AROT statistics for the air traffic mix at KEF by runway for the summer of 2018. The count is the number of landings in peak hours.}
\label{my-label3}
\end{table}
\fxnote{Include a schematic of the airport, overview for runway and taxiway reference}


% Please add the following required packages to your document preamble:
% \usepackage{graphicx}
\begin{table}[h]
\centering
\resizebox{0.8\textwidth}{!} & \multicolumn{1}{l|}{50\%} & \multicolumn{1}{l|}{75\%} & \multicolumn{1}{l|}{max} \\ \hline
\multicolumn{1}{|l|}{RWY 01} & 284 & 88 & 26 & 47 & 63 & 90 & 106 & 153 \\ \hline
\multicolumn{1}{|l|}{RWY 10} & 333 & 93 & 12 & 70 & 85 & 91 & 99 & 144 \\ \hline
\multicolumn{1}{|l|}{RWY 19} & 251 & 72 & 12 & 51 & 65 & 70 & 75 & 140 \\ \hline
\multicolumn{1}{|l|}{RWY 28} & 69 & 73 & 13 & 49 & 63 & 72 & 79 & 105 \\ \hline
\end{tabular}%
}
\caption[AROTs for the air traffic mix by runway for the winter]{AROT statistics for the air traffic mix at KEF by runway for the winter season. The count is the number of landings in peak hours since October 2017}
\label{my-label4}
\end{table}


% Please add the following required packages to your document preamble:
% \usepackage{graphicx}
% \usepackage[table,xcdraw]{xcolor}
% If you use beamer only pass "xcolor=table" option, i.e. \documentclass[xcolor=table]{beamer}
\begin{table}[h]
\centering
\resizebox{0.8\textwidth}{!} & \multicolumn{1}{l|}{50\%} & \multicolumn{1}{l|}{75\%} & \multicolumn{1}{l|}{max} \\ \hline
\rowcolor[HTML]{DAE8FC} 
\multicolumn{1}{|l|}{\cellcolor[HTML]{DAE8FC}January} & 95 & 86 & 20 & 54 & 70 & 85 & 96 & 136 \\ \hline
\rowcolor[HTML]{DAE8FC} 
\multicolumn{1}{|l|}{\cellcolor[HTML]{DAE8FC}February} & 80 & 87 & 16 & 52 & 75 & 86 & 93 & 144 \\ \hline
\rowcolor[HTML]{DAE8FC} 
\multicolumn{1}{|l|}{\cellcolor[HTML]{DAE8FC}March} & 215 & 84 & 18 & 47 & 70 & 85 & 95 & 150 \\ \hline
\rowcolor[HTML]{DAE8FC} 
\multicolumn{1}{|l|}{\cellcolor[HTML]{DAE8FC}April} & 221 & 82 & 17 & 47 & 69 & 81 & 93 & 139 \\ \hline
\rowcolor[HTML]{FFFC9E} 
\multicolumn{1}{|l|}{\cellcolor[HTML]{FFFC9E}May} & 386 & 75 & 13 & 46 & 67 & 74 & 83 & 140 \\ \hline
\rowcolor[HTML]{FFFC9E} 
\multicolumn{1}{|l|}{\cellcolor[HTML]{FFFC9E}June} & 647 & 73 & 16 & 47 & 62 & 68 & 80 & 155 \\ \hline
\rowcolor[HTML]{FFFC9E} 
\multicolumn{1}{|l|}{\cellcolor[HTML]{FFFC9E}July} & 729 & 72 & 16 & 46 & 61 & 68 & 77 & 155 \\ \hline
\rowcolor[HTML]{FFFC9E} 
\multicolumn{1}{|l|}{\cellcolor[HTML]{FFFC9E}August} & 724 & 80 & 18 & 46 & 64 & 81 & 91 & 148 \\ \hline
\rowcolor[HTML]{FFFC9E} 
\multicolumn{1}{|l|}{\cellcolor[HTML]{FFFC9E}September} & 632 & 78 & 18 & 46 & 64 & 75 & 88 & 152 \\ \hline
\rowcolor[HTML]{FFFC9E} 
\multicolumn{1}{|l|}{\cellcolor[HTML]{FFFC9E}October} & 609 & 79 & 17 & 46 & 66 & 76 & 91 & 155 \\ \hline
\rowcolor[HTML]{DAE8FC} 
\multicolumn{1}{|l|}{\cellcolor[HTML]{DAE8FC}November} & 220 & 84 & 23 & 48 & 66 & 82 & 99 & 151 \\ \hline
\rowcolor[HTML]{DAE8FC} 
\multicolumn{1}{|l|}{\cellcolor[HTML]{DAE8FC}December} & 106 & 87 & 23 & 51 & 69 & 86 & 99 & 153 \\ \hline
\end{tabular}%
}
\caption[AROTs for the air traffic mix by month]{AROT statistics for the air traffic mix at BIKF by month. The count is the number of landings in peak hours from October 2017 til November 2018. The colour fields indicate a subjective separation of the data into summer and winter season, based on mean AROT value. Months with mean AROT $\leq$ 80 seconds are classified as summer, and the remaining as winter.}
\label{my-label5}
\end{table}



\fxnote{Looked ar ROT, boxplot, seasonal and by RWY, fast exit effect}\\



% ---------------------------
\subsection{Landing Time Intervals}


% Please add the following required packages to your document preamble:
% \usepackage{multirow}
% \usepackage{graphicx}
\begin{table}[h]
\centering
\resizebox{\textwidth}{!}{%
\begin{tabular}{|c|c|c|c|c|c|c|c|}
\hline
\multicolumn{2}{|c|}{\multirow{2}{*}{RECAT-EU scheme}} & \multicolumn{6}{c|}{Follower}                   \\ \cline{3-8} 
\multicolumn{2}{|c|}{}                                 & CAT-A & CAT-B & CAT-C & CAT-D & CAT-E & CAT-F \\ \hline
\multirow{6}{*}{\rotatebox[origin=c]{90}{Leader}}             & CAT-A            &        & 100s  & 120s  & 140s  & 160s  & 180s  \\ \cline{2-8} 
                                    & CAT-B            &        &       &       & 100s  & 120s  & 140s  \\ \cline{2-8} 
                                    & CAT-C            &        &       &       & 80s   & 100s  & 120s  \\ \cline{2-8} 
                                    & CAT-D            &        &       &       &       &       & 120s  \\ \cline{2-8} 
                                    & CAT-E            &        &       &       &       &       & 100s  \\ \cline{2-8} 
                                    & CAT-F            &        &       &       &       &       & 80s   \\ \hline
\end{tabular}%
}
\caption[RECAT-EU time-based separation minima]{RECAT-EU WT time-based separation minima on approach and departure~\cite{rooseleer2015recat}\fxnote{This table is not referenced, maybe unnecessary, maybe move to LTI chapter!}}
\label{tab:RECAT-time}
\end{table}


% Please add the following required packages to your document preamble:
% \usepackage{multirow}
% \usepackage{graphicx}
% \usepackage[table,xcdraw]{xcolor}
% If you use beamer only pass "xcolor=table" option, i.e. \documentclass[xcolor=table]{beamer}
\begin{table}[h]
\centering
\resizebox{0.4\textwidth}{!}{%
\begin{tabular}{cc|r|r|r|}
\cline{3-5}
\multicolumn{1}{l}{} & \multicolumn{1}{l|}{} & \multicolumn{3}{c|}{Follower} \\ \cline{3-5} 
\multicolumn{1}{l}{} & \multicolumn{1}{l|}{} & \multicolumn{1}{c|}{H} & \multicolumn{1}{c|}{M} & \multicolumn{1}{c|}{L} \\ \hline
\multicolumn{1}{|c|}{} & H & \cellcolor[HTML]{FFCC67}56 & \cellcolor[HTML]{FE996B}332 & \cellcolor[HTML]{FFFFC7}2 \\ \cline{2-5} 
\multicolumn{1}{|c|}{} & M & \cellcolor[HTML]{FE996B}367 & \cellcolor[HTML]{FD6864}1795 & \cellcolor[HTML]{FFFFC7}7 \\ \cline{2-5} 
\multicolumn{1}{|c|}{\multirow{-3}{*}{\rotatebox[origin=c]{90}{Leader}}} & L & \cellcolor[HTML]{FFFFC7}3 & \cellcolor[HTML]{FFFFC7}9 & \cellcolor[HTML]{FFFFC7}1 \\ \hline
\end{tabular}%
}
\caption[BIKF traffic mix sorted into ICAO WTC]{Number of ICAO pairs from the traffic mix at BIKF arranged into the corresponding wake categories. It is apparent that the majority of arrival pairs are classified as Medium-Medium}
\label{tab:mix_to_wtc}
\end{table}


% Please add the following required packages to your document preamble:
% \usepackage{multirow}
% \usepackage{graphicx}
% \usepackage[table,xcdraw]{xcolor}
% If you use beamer only pass "xcolor=table" option, i.e. \documentclass[xcolor=table]{beamer}
\begin{table}[h]
\centering
\resizebox{\textwidth}{!}{%
\begin{tabular}{cc|c|c|c|c|c|c|}
\cline{3-8}
\multicolumn{1}{l}{} & \multicolumn{1}{l|}{} & \multicolumn{6}{c|}{Follower} \\ \cline{3-8} 
\multicolumn{1}{l}{} & \multicolumn{1}{l|}{} & CAT-A & CAT-B & CAT-C & CAT-D & CAT-E & CAT-F \\ \hline
\multicolumn{1}{|c|}{} & CAT-A &  &  &  & \cellcolor[HTML]{FFFFC7}1 &  &  \\ \cline{2-8} 
\multicolumn{1}{|c|}{} & CAT-B &  & \cellcolor[HTML]{FFFFC7}1 & \cellcolor[HTML]{FFFC9E}17 & \cellcolor[HTML]{FFFC9E}16 & \cellcolor[HTML]{FFFFC7}2 &  \\ \cline{2-8} 
\multicolumn{1}{|c|}{} & CAT-C &  & \cellcolor[HTML]{FFFC9E}19 & \cellcolor[HTML]{FD6864}1690 & \cellcolor[HTML]{FE996B}240 & \cellcolor[HTML]{FFCE93}42 & \cellcolor[HTML]{FFFC9E}13 \\ \cline{2-8} 
\multicolumn{1}{|c|}{} & CAT-D & \cellcolor[HTML]{FFFFC7}1 & \cellcolor[HTML]{FFFC9E}10 & \cellcolor[HTML]{FE996B}229 & \cellcolor[HTML]{FE996B}198 & \cellcolor[HTML]{FFFC9E}10 & \cellcolor[HTML]{FFFFC7}5 \\ \cline{2-8} 
\multicolumn{1}{|c|}{} & CAT-E &  & \cellcolor[HTML]{FFFFC7}1 & \cellcolor[HTML]{FFCE93}43 & \cellcolor[HTML]{FFFFC7}9 &  &  \\ \cline{2-8} 
\multicolumn{1}{|c|}{\multirow{-6}{*}{\rotatebox[origin=c]{90}{Leader}}} & CAT-F &  &  & \cellcolor[HTML]{FFFC9E}16 & \cellcolor[HTML]{FFFFC7}9 & \cellcolor[HTML]{FFFFC7}1 &  \\ \hline
\end{tabular}%
}
\caption[BIKF traffic mix sorted into RECAT-EU categories]{Number of RECAT pairs from the traffic mix at BIKF arranged into the corresponding wake categories. The majority of arrival pairs are classified as C-C.}
\label{tab:mix_to_recat}
\end{table}




\begin{figure}[h]
    \centering
    \includegraphics[width=1\textwidth]{graphics/fig_dist_separ_HH_HM_MH_MM_pairs.png}
    \caption[list of figures caption]{Caption}
    \label{fig:dist_separ_HH_HM_MH_MM_pairs}
\end{figure}

\begin{figure}
    \centering
    \includegraphics[width=1\textwidth]{graphics/fig_time_separ_HH_HM_MH_MM_pairs.png}
    \caption[list of figures caption]{Caption}
    \label{fig:time_separ_HH_HM_MH_MM_pairs}
\end{figure}




%\lipsum[14-20]
%%% Local Variables: 
%%% mode: latex
%%% TeX-master: "DEGREE-NAME-YEAR"
%%% End: 

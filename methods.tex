

\chapter{Methods\label{cha:methods}}
 
\section{Data Collecting and Data Filtering}
Isavia deploys the Automatic Dependent Surveillance-Broadcast (ADS-B) system \cite{ads_b} on 21~November~2014. ADS-B is a surveillance technology used to determine the position of aircraft via navigation satellites, combines that position with other aircraft data items, such as air speed, magnetic heading, altitude and flight number and broadcast this information simultaneously to other ADS-B capable aircraft and to ADS-B ground stations. The information is then relayed to Air Traffic Control centres in real time. The ADS-B equipped aircraft through Keflavik Airport are estimated to be around~$90\%$~\cite{isavia-rounardeild_rannsoknir_2018}. ADS-B was preferred to radar at Keflavik International Airport because of improved accuracy~\cite{isavia_wiki}. The aircraft surveillance data specifications used for measurements are in the Eurocontrol ASTERIX Category 21 standard~\cite{ASTERIX_ADS-B_specs}. The standard data items for transmissions of ADS-B are defined in Appendix~\ref{app:adsb_items}.
All the information from the ADS-B transmissions on flights passing through the airspace monitored and controlled by Isavia is stored on the company's database servers, administered by TERN Systems~\cite{Tern}, for reference and analysis.
Information on flights and aircraft used for this project was obtained by cross-referencing three data tables:

\begin{tabular}{ll}
   \textbullet \space \textit{bikf\_traffic} from \textit{airport\_traffic} database\\
   \textbullet \space  \textit{aicrafttypes} and \textit{flights}  from \textit{anscs\_isavia} database
\end{tabular}

The reading from the position of the ADS-B antenna of the aircraft is used by Isavia instead of the tail and nose positions. This simplification could result in a few seconds difference. Certain filtering criteria of the ADS-B data are applied by Isavia before commencing measurements and calculations of data items (e.g. final approach speed, runway occupancy time, location):~\cite{isavia_wiki}
\begin{itemize}
    \item All records with no latitude, longitude and/or time data are omitted.
    \item The records are filtered with regards to quality, i.e. filtered with regards to Target Surveillance status, MOPS version, NUC/NIC, NACp, SIL and PIC.
    \item An ADS-B data record within 0,5~second from another is omitted, i.e. if there are less than 0,5~seconds between two successive locations one of the records is omitted.
    \item An algorithm is used to analyse the data and determine if it originates from two different flights (the flight stops for 15 minutes or more after/before taxiing). If so the data for the second flight is omitted
    \item The velocity is calculated from ADS-B location data and ADS-B time data (velocity = distance travelled/time). The ADS-B velocity record is optional in the ASTERIX category 21 standard and is therefore unreliable in measurements. Since the velocity is calculated from measured data, it can exhibit spikes. To get a more realistic velocity curve, a Savitzky-Golay filter is applied to smooth the data.
    \item An algorithm is used to analyse the ADS-B data and estimate the runway occupancy time (ROT) from timestamp and/or location.
    \item An aircraft touchdown is considered to be the point where the ADS-B ground bit is set, i.e. the ADS-B ground bit goes from a value of 0 to a value of 1.
    \item An aircraft is considered to be stationary when the velocity is below 0,5~knots and to be moving if the velocity is greater than 0,5~knots.
    
\end{itemize}

The data relevant to this study provides information about the arriving flights at Keflavík International Airport during peak hours from the last four years, starting January~1,~2015. The information on flights includes data items such as aircraft type, wake turbulence category, final approach speed, landing time, runway and taxiway and AROT~(Table~\ref{tab:sample_df}).

\begin{table}[h]
\centering
\resizebox{\textwidth}{!}{%
\begin{tabular}{lllllllllll}
id & name & ICAO & WTC & AROT & runway & taxiway & peak & landing & threshold\_speed & final\_approach\_speed \\ \hline
13 & Boeing 767-300 & B763 & H & 42.953 & RWY 01 & TWY A-1 & Y & 2018-11-14 15:17:57 & 147.696 & 169.179 \\
13 & Boeing 767-300 & B763 & H & 75.594 & RWY 01 & TWY RWY 28 & Y & 2018-11-14 15:21:16 & 142.051 & 171.158 \\
150 & Boeing 757-200 & B752 & M & 55.852 & RWY 19 & TWY S-2 & Y & 2018-11-14 15:26:06 & 135.372 & 154.001 \\
107 & Boeing 757-300 & B753 & M & 59.288 & RWY 19 & TWY S-2 & Y & 2018-11-14 15:29:49 & 142.973 & 153.118 \\
79 & Raytheon Hawker 800 & H25B & M & 93.391 & RWY 19 & TWY S-1 & Y & 2018-11-14 17:40:12 & 119.583 & 171.378
\end{tabular}%
}
\caption[Data table sample from arrivals at BIKF during peak hours]{Data table sample (abridged) from the arrivals at BIKF during peak hours used for the analysis in this study. The units of the AROT are seconds, speed is measured in knots (nautical mile per hour).}
\label{tab:sample_df}
\end{table}

A key table containing characteristic information about 2300 aircraft models, including ICAO type designator and RECAT-EU wake turbulence categories for each model, was also used as reference (Table~\ref{tab:sample_key_RECAT}). The reference table was provided to Isavia by EUROCONTROL. 

% Please add the following required packages to your document preamble:
% \usepackage{graphicx}
\begin{table}[h]
\centering
\resizebox{0.7\textwidth}{!}{%
\begin{tabular}{llcll}
\multicolumn{1}{c}{Manufacturer} & \multicolumn{1}{c}{Model} & \multicolumn{1}{c}{ICAO Type Designator} & \multicolumn{1}{c}{WTC} & \multicolumn{1}{c}{RECAT-EU} \\ \hline
AIRBUS & A-350-1000 XWB & A35K & H & CAT-B \\
CESSNA & A-37 Dragonfly & A37 & L & CAT-F \\
AIRBUS & A-380-800 & A388 & H & CAT-A \\
SATIC & A-300ST Beluga & A3ST & H & CAT-C \\
SINGAPORE & A-4 Super Skyhawk & A4 & M & CAT-F
\end{tabular}%
}
\caption[Eurocontrol key reference table sample]{Key reference table sample (abridged) from EUROCONTROL used for assigning a RECAT-EU category to ICAO aircraft types arriving at BIKF.}
\label{tab:sample_key_RECAT}
\end{table}

The data frame of arriving flights at BIKF during peak hours was cross-referenced with the Eurocontrol key table, based on ICAO aircraft type labels -- the common item for both data frames. After cross-referencing, a RECAT-EU category was assigned to each of the aircraft models.

\section{Data Manipulation}

The statistical computations for this project were done in Python~3~\cite{python} programming language through the use of Jupyter Notebook~\cite{jupyter}, an open-source web application. The data was extracted from a MySQL~\cite{mysql} server into a data frame using Python and the manipulations on the data frame were performed using the functionality of Pandas~\cite{pandas} and NumPy~\cite{numpy} data analysis tools and computing packages.

The descriptive statistic methods of this study, examine the frequency distribution of a data set, the mean values and the standard deviation about the mean, in comparison to a certain reference value. The above measures along with the interquartile range (IQR) is used to summarise the distribution of given data points. Outliers are discarded using quantile values.

The data-set was filtered from unreliable information bits after examining the distribution of AROT -- one of the main metrics in this project. Outliers below the 0,003~quantile and above the 0,997~quantile were regarded as anomalies and removed from the data frame. All aircraft in the data-set now contained AROT values between 45 seconds and 156 seconds. The resulting data frame was used for initial analysis in this project. It contained over 11.500 arrivals for the time period of almost four years (from 01.01.2015 until 30.11.2018). The data frame contained unique information about the AROT of each aircraft, landing time and runway, along with the ICAO and RECAT-EU wake turbulence category. Within the process of developing this study, the time frame for the analysis was reduced to 14 months (from 04.10.2017 to 30.11.2018) because of the effect of rapid-exits on AROT, which is illustrated with the results in the upcoming section \ref{ssec:runway_usage_arot}.



%\lipsum[14-20]
%%% Local Variables: 
%%% mode: latex
%%% TeX-master: "DEGREE-NAME-YEAR"
%%% End: 
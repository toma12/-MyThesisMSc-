\chapter{Arrival Runway Occupancy Times}\label{cha:AROTs}


% Please add the following required packages to your document preamble:
% \usepackage{graphicx}
\begin{table}[]
\centering
\resizebox{\textwidth}{!} & \multicolumn{1}{l|}{50\%} & \multicolumn{1}{l|}{75\%} & \multicolumn{1}{l|}{max} \\ \hline
\multicolumn{1}{|l|}{Summer} & \multicolumn{1}{r|}{764} & 85 & 20 & 46 & 70 & 86 & 99 & 153 \\ \hline
\multicolumn{1}{|l|}{Winter} & \multicolumn{1}{r|}{263} & 100 & 17 & 55 & 87 & 98 & 111 & 156 \\ \hline
\end{tabular}%
}
\caption[AROTs RWY-01 pre fast exit by season]{AROTs for runway RWY-01 pre-fast-exit TWY A-1 by season}
\label{tab:season_AROT_stats_RWY01_pre_fast_exit}
\end{table}

% ----------------------------

% Please add the following required packages to your document preamble:
% \usepackage{graphicx}
\begin{table}[]
\centering
\resizebox{\textwidth}{!} & \multicolumn{1}{l|}{50\%} & \multicolumn{1}{l|}{75\%} & \multicolumn{1}{l|}{max} \\ \hline
\multicolumn{1}{|l|}{Summer} & \multicolumn{1}{r|}{1879} & 84 & 22 & 46 & 64 & 86 & 98 & 152 \\ \hline
\multicolumn{1}{|l|}{Winter} & \multicolumn{1}{r|}{284} & 88 & 26 & 47 & 63 & 90 & 106 & 153 \\ \hline
\end{tabular}%
}
\caption[AROTs RWY-01 post fast exit by season]{AROTs for runway RWY-01 post-fast-exit TWY A-1 by season}
\label{tab:season_AROT_stats_RWY01_post_fast_exit}
\end{table}

% ----------------------------

% Please add the following required packages to your document preamble:
% \usepackage{graphicx}
\begin{table}[]
\centering
\resizebox{\textwidth}{!} & \multicolumn{1}{l|}{50\%} & \multicolumn{1}{l|}{75\%} & \multicolumn{1}{l|}{max} \\ \hline
\multicolumn{1}{|l|}{Summer} & \multicolumn{1}{r|}{553} & 99 & 12 & 46 & 91 & 98 & 106 & 153 \\ \hline
\multicolumn{1}{|l|}{Winter} & \multicolumn{1}{r|}{249} & 110 & 13 & 60 & 100 & 108 & 117 & 155 \\ \hline
\end{tabular}%
}
\caption[AROTs RWY-28 pre fast exit by season]{AROTs for runway RWY-28 pre-fast-exit TWY B-1 by season}
\label{tab:season_AROT_stats_RWY28_pre_fast_exit}
\end{table}

% ----------------------------

% Please add the following required packages to your document preamble:
% \usepackage{graphicx}
\begin{table}[]
\centering
\resizebox{\textwidth}{!} & \multicolumn{1}{l|}{50\%} & \multicolumn{1}{l|}{75\%} & \multicolumn{1}{l|}{max} \\ \hline
\multicolumn{1}{|l|}{Summer} & 401 & 68 & 14 & 49 & 60 & 66 & 72 & 155 \\ \hline
\multicolumn{1}{|l|}{Winter} & 69 & 73 & 13 & 49 & 63 & 72 & 79 & 105 \\ \hline
\end{tabular}%
}
\caption[AROTs RWY-28 post fast exit by season]{AROTs for runway RWY-28 post-fast-exit TWY B-1 by season}
\label{tab:season_AROT_stats_RWY28_post_fast_exit}
\end{table}


% ----------------------------


\chapter{Code}\label{cha:code}
You can put code in your document using the listings package, which is
loaded by default in \path{custom.tex}.  Be aware that the listings
package does not put code in your document if you are in draft mode
unless you set the \texttt{forcegraphics} option.

There is an example java (Listing~\ref{src:Data_Bus.java}) and XML
file (Listing~\ref{src:AndroidManifest.xml}).  Thanks to the
\texttt{url} package, you can typeset OSX and unix paths like this:
\path{/afs/rnd.ru.is/project/thesis-template}.  Windows paths:
\path{C:\windows\temp\ }.  You can also typeset them using the menukey
package, but it tends to delete the last separator and has other
complications.\footnote{The menukey package has issues with biblatex,
  read \path{custom.tex} for more information.}

If you are trying to include multiple different languages, you should
go read the documentation and set these up in \path{custom.tex}.  You
will save yourself a lot of effort, especially if you have to fix
anything.

%I have put the source code in the \directory{src/} folder.
\lstinputlisting[language=Java, firstline=1,
lastline=40, caption={Data\_Bus.java: Setting up the class.},
label={src:Data_Bus.java}]{src/Data_Bus.java}

\lstinputlisting[language={[android]XML}, firstline=1, lastline=20,
caption={AndroidManifest.xml: Configuration for the Android UI.},
label={src:AndroidManifest.xml}]{src/AndroidManifest.xml}

%%% Local Variables: 
%%% mode: latex
%%% TeX-master: "DEGREE-NAME-YEAR"
%%% End: 
